\documentclass{article}
\usepackage{geometry}
\geometry{left=1.2in, right=1.2in, top=1.2in, bottom=1.2in}%change the margins here
\usepackage[utf8]{inputenc}
\usepackage{tikz}
\usetikzlibrary{cd}
\usetikzlibrary{shapes.geometric,arrows,positioning,fit,calc,}
\usepackage[english]{babel}
\usepackage{amsthm} %lets us use \begin{proof}
\usepackage{amssymb} %gives us the character \varnothing
\usepackage{mathtools}
\usepackage{amsmath}
\usepackage{hyperref}
\usepackage{dsfont}
\usepackage[shortlabels]{enumitem}
\usepackage{biblatex}
\addbibresource{references.bib}  % The filename of your .bib file
\usepackage{csquotes}
\usepackage{float}
\usepackage[all]{xy}
\usepackage{mathrsfs}
\usepackage{multirow}
\usepackage{adjustbox}
\usepackage{titlesec}

% Custom chapter format
\titleformat{\section}
  {\normalfont\Large\bfseries}
  {Chapter \thesection}{1em}{}

% Custom section format
\titleformat{\subsection}
  {\normalfont\large\bfseries}
  {Section \thesection.\arabic{subsection}}{1em}{}

% Custom subsection format
\titleformat{\subsubsection}
  {\normalfont\normalsize\bfseries}
  {Exercise \thesubsubsection}{0em}{}

% Make subsections numbered with respect to sections
\renewcommand{\thesubsection}{\arabic{section}.\arabic{subsection}}

% Make subsubsections numbered with respect to subsections
\renewcommand{\thesubsubsection}{\arabic{section}.\arabic{subsection}.}

\newcommand{\abs}[1]{\left| #1 \right|}
\newcommand{\norm}[1]{\left\| #1 \right\|}
\newcommand{\R}{\mathbb{R}}
\newcommand{\T}{\mathbb{T}}
\newcommand{\N}{\mathbb{N}}
\newcommand{\Z}{\mathbb{Z}}
\newcommand{\Q}{\mathbb{Q}}
\newcommand{\C}{\mathbb{C}}
\newcommand{\rddots}{\reflectbox{$\ddots$}}
\newcommand{\F}{\mathbb{F}}
\newcommand{\id}{\mathrm{id}}
\newcommand{\ctd}{\Rightarrow \Leftarrow}
\newcommand{\Ss}{\mathbb{S}}
\newcommand{\B}{\mathbb{B}}
\newcommand{\fI}{\mathscr{I}}
\newcommand{\fJ}{\mathscr{J}}
\newcommand{\fA}{\mathscr{A}}
\newcommand{\fB}{\mathscr{B}}
\newcommand{\fC}{\mathscr{C}}
\newcommand{\fD}{\mathscr{D}}
\newcommand{\fE}{\mathscr{E}}
\newcommand{\fO}{\mathscr{O}}
\newcommand{\fF}{\mathscr{F}}
\newcommand{\fG}{\mathscr{G}}
\newcommand{\fH}{\mathscr{H}}
\newcommand{\fT}{\mathscr{T}}
\newcommand{\fS}{\mathscr{S}}
\newcommand{\frakm}{\mathfrak{m}}
\newcommand{\frakn}{\mathfrak{n}}
\newcommand{\frakp}{\mathfrak{p}}
\newcommand{\nsubset}{\not \subset}
\newcommand\interior[1]{{#1}^{\circ}}
\newcommand{\Hh}{\mathbb{H}}
\newcommand{\D}{\mathbb{D}}
\DeclareMathOperator{\pre}{pre}
\DeclareMathOperator{\res}{res}
\DeclareMathOperator{\im}{im}
\DeclareMathOperator{\coim}{coim}
\DeclareMathOperator{\cok}{cok}
\DeclareMathOperator{\colim}{colim}
\DeclareMathOperator{\spn}{span}
\DeclareMathOperator{\Sym}{Sym}
\DeclareMathOperator{\Hom}{Hom}
\DeclareMathOperator{\Mor}{Mor}
\DeclareMathOperator{\Nat}{Nat}
\DeclareMathOperator{\Tr}{Tr}
\DeclareMathOperator{\Bd}{Bd}
\DeclareMathOperator{\Ann}{Ann}
\DeclareMathOperator{\Int}{Int}
\DeclareMathOperator{\Char}{char}
\DeclareMathOperator{\Aut}{Aut}
\DeclareMathOperator{\supp}{supp}
\DeclareMathOperator{\rank}{rank}
\DeclareMathOperator{\diag}{diag}
\DeclareMathOperator{\glue}{glue}
\DeclareMathOperator{\kerpre}{\ker_{\text{pre}}}
\DeclareMathOperator{\cokpre}{\cok_{\text{pre}}}
\DeclareMathOperator{\impre}{\im_{\text{pre}}}
\DeclareMathOperator{\sh}{sh}
\DeclareMathOperator{\ev}{ev}
\newcommand{\altid}{\mathds{1}}
\newcommand{\Ab}{\mathbf{Ab}} %Abelian Groups
\newcommand{\Grp}{\mathbf{Grp}} %Groups
\newcommand{\Ring}{\mathbf{Ring}} %Rings
\newcommand{\CRing}{\mathbf{CRing}} %Commutative Rings
\newcommand{\Rng}{\mathbf{Rng}} %Rings without identity
\newcommand{\Set}{\mathbf{Set}} %Sets
\newcommand{\pSet}{\mathbf{Set}_{\bullet}} %Pointed Spaces
\newcommand{\Top}{\mathbf{Top}} %Topological Spaces
\newcommand{\pTop}{\mathbf{Top}_{\bullet}} %Pointed Topological Spaces
\newcommand{\Op}{\mathbf{Op}} %Open Subsets
\newcommand{\Vect}{\mathbf{Vect}} %Vector Spaces
\newcommand{\Man}{\mathbf{Man}} %Manifolds
\newcommand{\Mod}{\mathbf{Mod}} %Modules
\newcommand{\Mon}{\mathbf{Mon}} %Monoids
\newcommand{\Cat}{\mathbf{Cat}} %Small Categories
\newcommand{\Ssubset}{\mathbf{Subset}} %Subsets
\newcommand{\Com}{\mathbf{Com}} %Complexes
\DeclareMathOperator{\Haus}{\mathbf{Haus}} %Hausdorff Spaces
\DeclareMathOperator{\Comp}{\mathbf{Comp}} %Compact Spaces
\DeclareMathOperator{\Poset}{\mathbf{Poset}} %Partially Ordered Sets
\DeclareMathOperator{\Graph}{\mathbf{Graph}} %Graphs (Not Graph Theory)
\DeclareMathOperator{\Sch}{\mathbf{Sch}} %Schemes
\DeclareMathOperator{\AffSch}{\mathbf{AffSch}} %Affine Schemes
\DeclareMathOperator{\Grph}{\mathbf{Grph}} %Graphs in Graph Theory and Graph Homomorphisms
\DeclareMathOperator{\Rel}{\mathbf{Rel}} %Sets and Relations
\DeclareMathOperator{\CW}{\mathbf{CW}} %CW Complexes and Cellular Maps
\DeclareMathOperator{\PreSh}{\mathbf{PreSh}} %Presheaves
\DeclareMathOperator{\Sh}{\mathbf{Sh}} %Sheaves
\DeclareMathOperator{\catD}{\mathbf{D}} %Derived Category
\DeclareMathOperator{\TopGrp}{\mathbf{TopGrp}} %Topological Groups
\DeclareMathOperator{\Meas}{\mathbf{Meas}} %Measurable Spaces and measurable functions
\DeclareMathOperator{\Cob}{\mathbf{Cob}} %Cobordisms
\DeclareMathOperator{\LieAlg}{\mathbf{LieAlg}} %Lie Algebras
\DeclareMathOperator{\Ban}{\mathbf{Ban}} %Banach Spaces and Bounded Linear Operators
\DeclareMathOperator{\Hilb}{\mathbf{Hilb}} %Hilbert Spaces and Bounded Linear Operators
\DeclareMathOperator{\AlgC}{\mathbf{Alg_C}} %C-Algebras where C isn't necessarily commutative
\DeclareMathOperator{\Rep}{\mathbf{Rep}} %Representations
\makeatletter
\newcommand\xtwoheadrightarrow[2][]{%
  \ext@arrow 0579{\twoheadrightarrowfill@}{#1}{#2}}
\newcommand\twoheadrightarrowfill@{%
  \arrowfill@\relbar\relbar\twoheadrightarrow}
\makeatother
\let\oldemptyset\emptyset
\let\emptyset\varnothing

\newtheorem{theorem}{Theorem}[section]
\newtheorem{corollary}{Corollary}[theorem]
\newtheorem{lemma}[theorem]{Lemma}
\newtheorem*{remark}{Remark}
\newtheorem*{lemma*}{Lemma}
\usepackage{lipsum}                     % Dummytext
\usepackage{xargs}                      % Use more than one optional parameter in a new commands
%\usepackage[pdftex,dvipsnames]{xcolor}  % Coloured text etc.
% 
\usepackage[colorinlistoftodos,prependcaption,textsize=tiny]{todonotes}
\newcommandx{\unsure}[2][1=]{\todo[linecolor=red,backgroundcolor=red!25,bordercolor=red,#1]{#2}}
\newcommandx{\change}[2][1=]{\todo[linecolor=blue,backgroundcolor=blue!25,bordercolor=blue,#1]{#2}}
\newcommandx{\info}[2][1=]{\todo[linecolor=OliveGreen,backgroundcolor=OliveGreen!25,bordercolor=OliveGreen,#1]{#2}}
\newcommandx{\improvement}[2][1=]{\todo[linecolor=Plum,backgroundcolor=Plum!25,bordercolor=Plum,#1]{#2}}
\newcommandx{\thiswillnotshow}[2][1=]{\todo[disable,#1]{#2}}
%
\title{Solutions to ``The Rising Sea"}
\author{Jack Westbrook}
\date\today
%This information doesn't actually show up on your document unless you use the maketitle command below

\begin{document}
\maketitle %This command prints the title based on information entered above

%Section and subsection automatically number unless you put the asterisk next to them.
The exercises in this document are taken from the February 21, 2024 draft of Ravi Vakil's ``The Rising Sea".
You can access the draft \href{https://math.stanford.edu/~vakil/216blog/FOAGfeb2124public.pdf}{here}.

\section*{Preliminary Results}
\subsection*{Results in Arbitrary Categories}
\begin{lemma}\label{lem:ker monic}
    If $f:A\to B$, the inclusion map $\iota:\ker f\to A$ is monic.
    \begin{proof}
        If $g_1,g_2:C\to \ker f$ are such that $\iota\circ g_1=\iota\circ g_2$, then the following diagram commutes:
        \begin{center}
            \begin{tikzcd}
                &&C\\
                &\ker f\ar{r}{\iota} \ar{ur}{0}& A \ar{u}[swap]{g\circ f}\\
                D\ar[dashed]{ur}[description]{\exists!} \ar[bend right, shift right]{urr}[swap]{\iota\circ g_1} \ar[bend right, shift left]{urr}{\iota\circ g_2}
            \end{tikzcd}
        \end{center}
        We immediately notice both $g_1$ and $g_2$ satisfy the unique arrow because $\iota\circ g_2=\iota\circ g_1$. By uniqueness, $g_1=g_2$.
    \end{proof}
\end{lemma}
\begin{lemma}\label{lem:cok epic}
    If $f:A\to B$, the projection $\pi:B \to \cok f$ is epic.
    \begin{proof}
        Suppose $g_1,g_2:\cok f\to C$ are such that $g_1\circ \pi=g_2\circ \pi$. Then the following diagram commutes:
        \begin{center}
            \begin{tikzcd}
                &&C\\
                &\cok f \ar[dashed]{ur}[description]{\exists!}&\\
                A\ar{ur}{0} \ar{r}{f}&B\ar{u}{\pi} \ar[bend right, shift right]{uur}[swap]{g_1\circ \pi} \ar[bend right, shift left]{uur}{g_2\circ \pi}&
            \end{tikzcd}
        \end{center}
        We notice immediately that $g_1$ and $g_2$ satisfy the unique arrow because $g_2\circ \pi=g_1\circ \pi$, so by uniqueness $g_1=g_2$.
    \end{proof}
\end{lemma}
\begin{lemma}\label{lem:comp epic then epic}
    If $h=g\circ f$ and $h$ is epic, then $g$ is epic.
\end{lemma}
\begin{proof}
    Suppose $\phi\circ g=\varphi \circ g$. Then it's also true that
    \[
    \phi \circ g\circ f=\varphi\circ g\circ f
    \]
    which by definition implies
    \[
    \phi \circ h=\varphi\circ h
    \]
    Because $h$ is epic, $\phi=\varphi$ as desired.
\end{proof}
\begin{lemma}\label{lem:comp monic then monic}
    If $h=g\circ f$ and $h$ is monic, then $f$ is monic.
\end{lemma}
\begin{proof}
    If $f\circ \phi=f\circ \varphi$, then
    \[
    g\circ f\circ \phi=g\circ f\circ \varphi
    \]
    which means by definition
    \[
    h\circ \phi=h\circ \varphi
    \]
    Because $h$ is monic, then $\phi=\varphi$ as desired.
\end{proof}
\begin{lemma}\label{lem:comp with monic and ker}
    If $A\xrightarrow{f} B\xhookrightarrow{g}C$, then $\ker (g\circ f)=\ker f$.
    \begin{proof}
        If $\ker f\xhookrightarrow{\iota} A$ is the inclusion, it follows that the following diagram commutes:
        \begin{center}
            \begin{tikzcd}
            & C\\
                \ker f \ar[hookrightarrow]{r}{\iota} \ar{ur}{0}& A\ar{u}{g\circ f}
            \end{tikzcd}
        \end{center}
        If there is a morphism $h:D\to A$ such that $g\circ f \circ h=0$, then we notice
        \[
        g\circ f\circ h=0=g\circ 0
        \]
        which implies, by $g$ being monic, that $f\circ h=0$. Then we obtain a unique induced morphism from the following diagram:
        \begin{center}
            \begin{tikzcd}
                &&B\\
                &\ker f\ar[hookrightarrow]{r}{\iota} \ar{ur}{0}& A \ar{u}[swap]{f}\\
                D\ar[dashed]{ur}[description]{\exists!} \ar[bend right]{urr}{h}
            \end{tikzcd}
        \end{center}
        Thus, in particular, the following diagram commutes as well:
        \begin{center}
            \begin{tikzcd}
                &&C\\
                &\ker f\ar[hookrightarrow]{r}{\iota} \ar{ur}{0}& A \ar{u}[swap]{g\circ f}\\
                D\ar[dashed]{ur}[description]{\exists!} \ar[bend right]{urr}{h}
            \end{tikzcd}
        \end{center}
    \end{proof}
\end{lemma}
\begin{corollary}\label{cor:comp with monic and coim}
        If $A\xrightarrow{f}B \xhookrightarrow{g} C$, then $\coim (g\circ f)=\coim f$.
        \begin{proof}
            $\ker (g\circ f)=\ker f$, thus 
            \[
            \coim (g\circ f)=\cok \ker (g\circ f)=\cok \ker f=\coim f
            \]
        \end{proof}
    \end{corollary}
\begin{lemma}\label{lem:comp with epic and cok}
    If $A\overset{f}\twoheadrightarrow B\xrightarrow{g}C$, then $\cok (g\circ f)=\cok g$.
    \begin{proof}
        If $C\overset{\pi}\twoheadrightarrow \cok g$ is the projection, 
        \begin{center}
            \begin{tikzcd}
            &\cok g\\
                A \ar{r}{g\circ f} \ar{ur}{0}&C \ar[two heads]{u}{\pi}
            \end{tikzcd}
        \end{center}
        commutes because $\pi \circ g=0$. If we have some $p:C\to D$ such that $p\circ g\circ f=0$, we notice
        \[
        p\circ g\circ f=0=0 \circ f
        \]
        which implies, by $f$ being epic, that $p\circ g=0$. Thus the following commutes:
        \begin{center}
            \begin{tikzcd}
            &&D\\
            &\cok g\ar[dashed]{ur}[description]{\exists!}&\\
                B \ar{r}{g} \ar{ur}{0}&C \ar[two heads]{u}{\pi}\ar[bend right]{uur}{p}&
            \end{tikzcd}
        \end{center}
        In particular, we have the following unique morphism from the above diagram such that the following diagram commutes:
        \begin{center}
            \begin{tikzcd}
            &&D\\
            &\cok g\ar[dashed]{ur}[description]{\exists!}&\\
                A \ar{r}{g\circ f} \ar{ur}{0}&C \ar[two heads]{u}{\pi}\ar[bend right]{uur}{p}&
            \end{tikzcd}
        \end{center}
    \end{proof}
\end{lemma}
    
    \begin{corollary}\label{cor:comp with epic and im}
    If $A\overset{f}\twoheadrightarrow B \xrightarrow{g} C$, then $\im (g\circ f)=\im g$.
    \begin{proof}
        $\cok (g\circ f)=\cok f$, thus
        \[
        \im (g\circ f)=\ker \cok (g\circ f)=\ker \cok g=\im g
        \]
    \end{proof}
\end{corollary}
\begin{lemma}\label{lem:double comp with monic and ker}
    If $\ker g \xhookrightarrow{\varphi} A\xhookrightarrow{f}B \xrightarrow[]{g}C$ such that $f\circ \varphi$ is the inclusion\\  $\iota:\ker g\hookrightarrow B$, then $\ker (g\circ f)=\ker g$.
    \begin{proof}
        We have the following diagram commutes:
        \begin{center}
            \begin{tikzcd}
                &C\\
                \ker g\ar[hookrightarrow]{r}{\varphi} \ar{ur}{0}& A \ar{u}[swap]{g\circ f}\\
            \end{tikzcd}
        \end{center}
        because
        \[
        g\circ f\circ \varphi=g\circ \iota=0
        \]
        Now if there exists some morphism $h:D\to A$ such that $g\circ f \circ h=0$, then
        \begin{center}
            \begin{tikzcd}
                &&C\\
                &\ker g\ar[hookrightarrow]{r}{\iota} \ar{ur}{0}& B \ar{u}[swap]{g}\\
                D\ar[dashed]{ur}[description]{\exists!} \ar[bend right]{urr}{f\circ h}
            \end{tikzcd}
        \end{center}
        If $\phi$ is the induced morphism, then 
        \[
        f\circ \varphi\circ \phi=\iota\circ \phi=f\circ h
        \]
        Because $f$ is monic, we get that $\varphi \circ \phi=h$ so that the following diagram commutes:
        \begin{center}
            \begin{tikzcd}
                &&C\\
                &\ker g\ar[hookrightarrow]{r}{\varphi} \ar{ur}{0}& A \ar{u}[swap]{g\circ f}\\
                D\ar[dashed]{ur}[description]{\phi} \ar[bend right]{urr}{h}
            \end{tikzcd}
        \end{center}
    \end{proof}
\end{lemma}
\begin{lemma}\label{lem:ker of comp into ker of first}
    If $A\xrightarrow{f}B\xrightarrow{g}C$, then there is a canonical monomorphism $\ker f\hookrightarrow \ker (g\circ f)$.
\end{lemma}
\begin{proof}
    The canonical morphism is the one induced in the following commutative diagram:
    \begin{center}
        \begin{tikzcd}
            &&C\\
            &\ker (g\circ f) \ar[hook]{r}{i} \ar{ur}{0}&A \ar{u}{g\circ f}\\
            \ker f \ar[bend right,hook]{urr}[hook]{j} \ar[dashed]{ur}[description]{\exists!}
        \end{tikzcd}
    \end{center}
    where $g\circ f\circ j=g\circ 0=0$, and the induced morphism is monic by Lemma \ref{lem:comp monic then monic}.
\end{proof}
\begin{lemma}\label{lem:cok of comp onto cok of second}
    If $A\xrightarrow{f}B\xrightarrow{g}C$, then there is a canonical epimorphism $\cok (g\circ f) \twoheadrightarrow \cok g$.
\end{lemma}
\begin{proof}
    The canonical morphism is the one induced in the following commutative diagram:
    \begin{center}
        \begin{tikzcd}
            && \cok g\\
            & \cok (g\circ f) \ar[dashed]{ur}[description]{\exists!}\\
            A \ar{r}{g\circ f} \ar{ur}{0}& C \ar[two heads]{u}{p} \ar[two heads, bend right]{uur}[swap]{q}
        \end{tikzcd}
    \end{center}
    where $q\circ g\circ f=0\circ f=0$, and the induced morphism is epic by Lemma \ref{lem:comp epic then epic}.
\end{proof}
\begin{lemma}\label{lem:im of comp into im of second}
    If If $A\xrightarrow{f}B\xrightarrow{g}C$, then there is a canonical monomorphism $\im (g\circ f) \hookrightarrow \im g$
\end{lemma}
\begin{proof}
    Using the same notation as was used in the Lemmas referenced, by Lemma \ref{lem:cok of comp onto cok of second} we get a morphism $q':\cok(g\circ f)\twoheadrightarrow \cok g$ such that $q=q'\circ p$. Then by Lemma \ref{lem:ker of comp into ker of first}, we get the desired morphism $i':\ker p\hookrightarrow \ker q$ such that $i'=i\circ q'$.
\end{proof}
\subsection*{Results in Abelian Categories}
\begin{lemma}\label{lem:monic iff im is source}
    If $\iota:A\to B$, then $\iota$ is monic if and only if $\im \iota=A$.
\end{lemma}
\begin{proof}
    The forward direction is by definition of an Abelian Category. For the reverse direction, suppose $\im \iota=A$. Then $\iota$ is a kernel of its cokernel, and by Lemma \ref{lem:ker monic} we obtain that $\iota$ is monic.
\end{proof}
\begin{lemma}\label{lem:epic iff coim is target}
    If $\pi: A\to B$, then $\pi$ is epic if and only if $\coim \pi =B$.
\end{lemma}
\begin{proof}
    The forward direction is by definition of an Abelian Category. For the reverse direction, suppose $\coim \pi=B$. Then $\pi$ is a cokernel of its kernel, and by Lemma \ref{lem:cok epic} we obtain that $\pi$ is epic.
\end{proof}
\begin{lemma}\label{lem:double factorization through im}
    If a morphism $f:A\to B$ factorizes as both $\iota \circ q$ and $\iota'\circ q'$ where $\iota'$ is monic and $\iota \circ q$ is the canonical factorization of $f$ through $\im f$, the following diagram commutes:
    \begin{center}
        \begin{tikzcd}
            A\ar{r}{q}\ar{d}{q'}& \ker \pi \ar[hook]{d}{\iota} \ar[dashed]{dl}[description]{\exists!}\\
            \ker \pi' \ar[hook]{r}{\iota'}&B
        \end{tikzcd}
    \end{center}
    \cite{FIT}
    \begin{proof}
        We will let $\pi:B\to \cok \iota$ and $\pi':B\to \cok \iota'$ be the projections. Therefore
        \[
        \pi'\circ f=\pi'\circ \iota'\circ q'=0\circ q'=0
        \]
        Because also $\cok \iota=\cok f$, we get the following commutative diagram:
        \begin{center}
            \begin{tikzcd}
                &&\cok \iota'\\
                &\cok \iota \ar[dashed]{ur}[description]{\exists! \varphi}&\\
                A\ar{ur}{0} \ar{r}{f}& B \ar[two heads]{u}{\pi} \ar[two heads, bend right]{uur}{\pi'}
            \end{tikzcd}
        \end{center}
        Therefore
        \[
        \pi'\circ \iota =\varphi \circ \pi \circ \iota=\varphi \circ 0=0
        \]
        We use the fact that $\pi'\circ \iota=0$ to get the following commutative diagram:
        \begin{center}
            \begin{tikzcd}
                 &&\cok \iota'\\
                &\ker \pi'\ar{ur}{0} \ar[hook]{r}{\iota'}&B\ar[two heads]{u}{\pi'}\\
                \ker \pi \ar[dashed]{ur}[description]{\exists! \chi} \ar[hook, bend right]{urr}{\iota}
            \end{tikzcd}
        \end{center}
         Therefore
        \[
        \iota'\circ q'=\iota \circ q=\iota' \circ \chi \circ q
        \]
        We use the fact that $\iota'$ is monic to obtain
        \[
        q'=\chi \circ q
        \]
        which shows the desired diagram does indeed commute.
    \end{proof}
\end{lemma}
\begin{theorem}\label{thm:map to im is epic}
    For every morphism $f:A\to B$, the following diagram commutes:
    \begin{center}
        \begin{tikzcd}
            &\cok f\\
            \mathrm{im} f\ar[hook]{r}{\iota}\ar{ur}{0}&B\ar[two heads]{u}{\pi}\\
            A\ar[two heads, dashed]{u}[description]{\exists!q} \ar[bend right]{ur}{f}
        \end{tikzcd}
    \end{center}
        \cite{FIT}
\end{theorem}
\begin{proof}
Existence of the morphism $q: A\to \im f$ is simply by definition of $\im f$. The main result is that $q$ is epic. To show this, suppose there are two morphism $g,h:\im f\to C$ such that $g\circ q=h\circ q$. Then 
\[
g\circ q-h\circ q=0\Rightarrow (g-h)\circ q=0 
\]
We will focus our attention on $\ker g-h$, which is the equalizer of $g$ and $h$, from which we obtain the following commutative diagram:
\begin{center}
    \begin{tikzcd}
        &&C\\
        &\ker g-h \ar{ur}{0}\ar[hook]{r}{j}&\im f \ar{u}[swap]{g-h}\\
        A \ar[dashed]{ur}[description]{\exists! p} \ar[bend right]{urr}{q}
    \end{tikzcd}
\end{center}
This implies that
\[
f=\iota \circ q=\iota \circ j\circ p
\]
We notice that $\iota \circ j$ is monic, so by the statement above, we obtain from Lemma \ref{lem:double factorization through im} the following commutative diagram:
\begin{center}
    \begin{tikzcd}
        A\ar{r}{q} \ar{d}{p}& \im f \ar[hook]{d}{\iota}\ar[hook]{dl}[description]{\chi}\\
        \ker g-h \ar[hook]{r}{\iota \circ j}&B
    \end{tikzcd}
\end{center}
Therefore
\[
\iota=\iota \circ j\circ \chi
\]
Using the fact $\iota$ is monic we obtain
\[
\id_{\im f}=j\circ \chi
\]
Using this, we also obtain
\[
j\circ \chi\circ j=\id_{\im f}\circ j=j
\]
Now using the fact $j$ is monic, we get
\[
\chi \circ j=\id_{\ker g-h}
\]
Thus $j$ is an isomorphism and is in particular epic. Then
\[
(g-h)\circ j=0=0\circ j
\]
implies that, by $j$ being epic, that $g-h=0$, or equivalently $g=h$. Thus $q$ is indeed epic.
\end{proof}
\begin{theorem}[The First Isomorphism Theorem or The 1IT]\label{thm:1IT}
    If $f:A\to B$ is a morphism, then $\im f= \coim f$.
\end{theorem}
\begin{proof}
    We have the canonical epimorphism $q:A\to \im f$. Because $f=\iota \circ q$ and $\iota$ is monic, we get by Corollary \ref{cor:comp with monic and coim} that
    \[
    \coim f=\coim (\iota \circ q)=\coim q
    \]
    From Theorem 0.7 we obtain from that $q$ is epic. By definition, in any abelian category the coimage of an epimorphism is the target, so in the case of $q:A\to \im f$
    \[
    \coim q=\im f
    \]
    Hence
    \[
    \coim f=\im f
    \]
\end{proof}
\begin{theorem}[The Third Isomorphism Theorem or the 3IT]\label{thm:3IT}
    If $A\hookrightarrow B\hookrightarrow C$, then $C/B=(C/A)/(B/A)$.
\end{theorem}
\begin{proof}
    To prove this, let $j:A\hookrightarrow B$ and $i: B\hookrightarrow C$ as well as $q=\cok (i\circ j)$ and $p=\cok i$ . We're going to show that $(C/A)/(B/A)$ satisfies the universal property of $C/B$. First, we observe there is a canonical morphism $\iota$ given below:
    \begin{center}
        \begin{tikzcd}
            &&\cok(i\circ j)\\
            &\cok j \ar[dashed]{ur}[description]{\exists!}\\
            A \ar{ur}{0} \ar[hook]{r}{j}& B \ar[two heads]{u}{p} \ar[hook]{r}{i}& C\ar[two heads]{uu}{q}
        \end{tikzcd}
    \end{center}
    We will first show that $\iota$ is monic. We will do this by first proving $\ker(q\circ i)=A$. Suppose that there is some $h:D\to B$ such that $q\circ i\circ h=0$. Then we get the following commutative diagram:
    \begin{center}
        \begin{tikzcd}
            &&\cok(i\circ j)\\
            &\ker q \ar[hook]{r}{i\circ j} \ar{ur}{0}& C \ar[two heads]{u}{q}\\
            D \ar[bend right]{urr}{i\circ h} \ar[dashed]{ur}[description]{\exists!h'}
        \end{tikzcd}
    \end{center}
    Then
    \[
    i\circ j\circ h'=i\circ h
    \]
    implies, by $i$ being monic, that $j\circ h'=h$. Thus $h'$ is the unique morphism satisfying the diagram below, where $\ker q=A$ essentially by definition:
    \begin{center}
        \begin{tikzcd}
            &&\cok(i\circ j)\\
            &A \ar[hook]{r}{j} \ar{ur}{0}& B \ar{u}{q\circ i}\\
            D \ar[bend right]{urr}{h} \ar[dashed]{ur}[description]{\exists!}
        \end{tikzcd}
    \end{center}
    This demonstrates that indeed $A=\ker(q\circ i)$. By Corollary \ref{cor:comp with epic and im}
    \[
    \im \iota=\im(\iota \circ p)
    \]
    which by commutativity is equal to $\im(q\circ i)$. By the 1IT \ref{thm:1IT}, $\im(q\circ i)=\coim(q\circ i)=B/\ker(q\circ i)$. By our work above, $B/\ker(q\circ i)=B/A$. Then we obtain that
    \[
    \im \iota=B/A
    \]
    By Lemma \ref{lem:monic iff im is source}, this shows that $\iota$ is indeed monic.
    We claim that $\cok \iota =\cok i$, where we let $\tau:C\twoheadrightarrow C/B$ be the canonical projection. Suppose that $h\circ \iota=0$ for some morphism $h:C/A\to D$. Therefore
    \begin{align*}
        0=0\circ p=h\circ \iota \circ p=h\circ q \circ i
    \end{align*}
Therefore $h\circ q$ factors uniquely through $\cok i=C/B$ as shown below:
\begin{center}
    \begin{tikzcd}
        &C/B \ar[dashed]{r}[description]{\exists!h'}&D\\
        B \ar[hook]{r}{i} \ar{ur}{0}& C \ar[two heads]{u}{\tau} \ar[two heads]{r}{q}&C/A \ar{u}{h}
    \end{tikzcd}
\end{center}
However, we can also show that $\tau$ factors through $q$, because $\tau \circ i \circ j=0\circ j=0$, so we also have the following commutative diagram:
\begin{center}
    \begin{tikzcd}
        &C/A \ar[two heads]{r}[description]{\exists! \tau'}& C/B\\
        A\ar[hook]{r}{i\circ j} \ar{ur}{0}& C \ar[two heads]{u}{q} \ar[two heads]{ur}{\tau}
    \end{tikzcd}
\end{center}
    Plugging in our result that $\tau=\tau'\circ q$ to the previous result, we obtain that
    \begin{align*}
        h\circ q=h'\circ \tau=h'\circ \tau' \circ q
    \end{align*}
    Now because $q$ is an epimorphism, we obtain that $h=h'\circ \tau'$. The final thing to show is that $\tau'\circ \iota=0$. This is because
    \begin{align*}
        \tau'\circ \iota\circ p=\tau'\circ q \circ i=\tau \circ i=0=0\circ p
    \end{align*}
    and $p$ is epic implies that indeed $\tau'\circ \iota=0$. We have shown that $h$ factors uniquely through $\tau'$ in the below commutative diagram
    \begin{center}
        \begin{tikzcd}
        &&D\\
            &C/B \ar[dashed]{ur}[description]{\exists! h'}\\
            B/A \ar[hook]{r}{\iota}\ar{ur}{0}&C/A \ar[two heads]{u}{\tau'} \ar[bend right]{uur}{h}
        \end{tikzcd}
    \end{center}
    so indeed $C/B=\cok \iota=(C/A)/(B/A)$ because $C/B$ satisfies the universal property of $\cok \iota$.
\end{proof}
\begin{lemma}\label{lem:comp with epic and ker}
    If $A\xtwoheadrightarrow{f}B\xrightarrow{g} C$ and $\ker(g\circ f)=\ker f$, then $g$ is monic.
\end{lemma}
\begin{proof}
    We obtain by taking the cokernel of each side that
    \[
    \coim (g\circ f)=\coim f
    \]
    By the 1IT \ref{thm:1IT}, we obtain that
    \[
    \im (g\circ f)=\im f
    \]
    Thus we have the following commutative diagram:
    \begin{center}
        \begin{tikzcd}
            &&\cok(g\circ f)\\
            \\
            A\ar{uurr}{0}\ar[two heads]{r}{f}&B\ar{r}{g}&C\ar[two heads]{uu}&B=\im (g\circ f) \ar{l}[swap]{g}\ar{uul}[swap]{0}
        \end{tikzcd}
    \end{center}
    By Lemma \ref{lem:ker monic}, $g$ being a kernel is monic.
\end{proof}
\begin{lemma}\label{lem:covariant right exact preserves epic}
    If $F:\fA \to \fB$ is a right exact covariant functor and $f:A\to B$ is epic, then $Ff$ is epic.
\end{lemma}
\begin{proof}
    We have the exact sequence $\ker f \xrightarrow{i} A \xrightarrow{f}B \rightarrow0 $. By right exactness of $F$, then the following is also exact:
    \begin{align*}
        F\ker f \xrightarrow{Fi} FA \xrightarrow{Ff} FB \rightarrow 0
    \end{align*}
    In particular, $\im Ff=FB$ is the target of $Ff$, so by Lemma \ref{lem:epic iff coim is target} $Ff$ is epic.
\end{proof}
\begin{lemma}\label{lem:covariant left exact preserves monic}
    If $F:\fA \to \fB$ is a left exact covariant functor and $f:A\to B$ is monic, then $Ff$ is monic.
\end{lemma}
\begin{proof}
    We have the exact sequence $0\rightarrow A \xrightarrow{f}B\xrightarrow{p}\cok f$. By left exactness of $F$, the following is also exact:
    \begin{align*}
        0\rightarrow FA \xrightarrow{Ff} FB \xrightarrow{Fp} F\cok f
    \end{align*}
    In particular, $\ker Ff=0$ so $Ff$ is monic.
\end{proof}
\begin{lemma}\label{lem:covariant right exact commutes with cok}
    If $F:\fA \to \fB$ is a right exact covariant functor and $f:A\to B$, then $\cok Ff=F\cok f$.
\end{lemma}
\begin{proof}
    We have $A\xrightarrow{f}B \xrightarrow{p}\cok f\rightarrow 0$ is an exact sequence in $\fA$. Then by right exactness,
    \begin{align*}
        &FA \xrightarrow{Ff} FB\xrightarrow{Fp}F\cok f\rightarrow0
    \end{align*}
    is exact. Then we get the following commutative diagram:
    \begin{center}
        \begin{tikzcd}
            &&F\cok f\\
            &\cok Ff \ar[dashed]{ur}[ description]{\exists!}\\
            FA \ar[r, "Ff"] \ar[ur,"0"]&FB \ar[two heads] {u} {\pi}   \ar[bend right, two heads]{uur}{Fp}
            \end{tikzcd}
    \end{center}
    where $Fp$ is epic by Lemma \ref{lem:covariant right exact preserves epic}. Thus by exactness
    \[
    \ker Fp=\im Ff=\ker \pi
    \]
    Therefore
    \begin{align*}
        \ker \pi=\ker Fp\Rightarrow \coim \pi=\coim Fg
    \end{align*}
    Because both $\pi$ and $Fg$ are epic, we also have $\coim \pi=\pi$ and $\coim Fp=Fp$, which proves $\pi=Fp$.
\end{proof}
\begin{corollary}\label{cor:contravariant right exact and ker to cok}
    If $F:\fA \to \fB$ is a right exact contravariant functor and $f:A\to B$, then $\cok Ff=F\ker f$.
\end{corollary}
\begin{proof}
    If $F:\fA \to \fB$ is contravariant, then $F:\fA^{op}\to \fB$ is an equivalent formulation. Because limits in $\fA^{op}$ are colimits in $\fA$ and vice versa,
    \begin{align*}
        \ker f=\cok (f^{op})
    \end{align*}
    implies that by Lemma \ref{lem:covariant right exact commutes with cok}
    \begin{align*}
        F\ker f=F\cok (f^{op})=\cok Ff^{op}=\cok Ff
    \end{align*}
\end{proof}
\begin{lemma}\label{lem:covariant left exact commutes with ker}
    If $F:\fA \to \fB$ is a left exact covariant functor and $f:A\to B$, then $\ker Ff=F\ker f$.
\end{lemma}
\begin{proof}
    We have $0\rightarrow \ker f\xrightarrow{i} A\xrightarrow{f}B$ is exact. By left exactness,
    \begin{align*}
        0\rightarrow F\ker f \xrightarrow{Fi} FA \xrightarrow{Ff} FB
    \end{align*}
    is exact. Therefore we have the following commutative diagram:
    \begin{center}
        \begin{tikzcd}
            &&FB\\
            &\ker Ff \ar[hook]{r}{\iota} \ar{ur}{0}& FA \ar{u}{Ff}\\
            F\ker f \ar[dashed]{ur}[description]{\exists!} \ar[bend right, hook]{urr}{Fi}
        \end{tikzcd}
    \end{center}
    where $Fi$ is monic by Lemma \ref{lem:covariant left exact preserves monic}. By exactness, $\im Fi=\ker Ff$, and because $Fi$ is monic, then $\im Fi=F\ker f$ by Lemma \ref{lem:monic iff im is source}, which proves $\ker Ff=F\ker f$.
\end{proof}
\begin{corollary}\label{cor:contravariant left exact and cok to ker}
    If $F:\fA \to \fB$ is a left exact contravariant functor and $f:A\to B$, then $\ker Ff=F\cok f$.
\end{corollary}
\begin{proof}
    If $F:\fA \to \fB$ is contravariant, then $F:\fA^{op}\to \fB$ is an equivalent formulation. Because limits in $\fA^{op}$ are colimits in $\fA$ and vice versa,
    \begin{align*}
        \cok f=\ker (f^{op})
    \end{align*}
    implies that by Lemma \ref{lem:covariant left exact commutes with ker}
    \begin{align*}
        F\cok f=F\ker (f^{op})=\ker Ff^{op}=\ker Ff
    \end{align*}
\end{proof}
\begin{lemma}\label{lem:covariant exact and commutes with im and coim}
    If $F:\fA \to \fB$ is an exact covariant functor and $f:A\to B$, then $\im Ff=F\im f$ and $\coim Ff=F\coim f$.
\end{lemma}
\begin{proof}
    By Lemmas \ref{lem:covariant left exact commutes with ker} and \ref{lem:covariant right exact commutes with cok}, we have
    \begin{align*}
        F\im f=F\ker \cok f=\ker F\cok f=\ker \cok Ff=\im Ff
    \end{align*}
    as well as
    \begin{align*}
        F\coim f=F\cok \ker f=\cok F\ker f=\cok \ker Ff=\coim Ff
    \end{align*}
\end{proof}
\begin{lemma}\label{lem:contravariant exact and im to coim and coim to im}
     If $F:\fA \to \fB$ is an exact contravariant functor and $f:A\to B$, then $\im Ff=F\coim f$ and $\coim Ff=F\im f$.
\end{lemma}
\begin{proof}
    By Corollaries \ref{cor:contravariant left exact and cok to ker} and \ref{cor:contravariant right exact and ker to cok},
    \begin{align*}
        F\coim f=F\cok \ker f=\ker F\ker f=\ker \cok Ff=\im Ff
    \end{align*}
    and
    \begin{align*}
        F\im f=F\ker \cok f=\cok F\cok f=\cok \ker Ff=\coim Ff
    \end{align*}
\end{proof}
\subsection*{Miscellaneous Results}
\begin{lemma}
        If $D:\N \to \Top$ is a diagram and $D':\N \to \Top$ is another diagram and there exists some embedding $\sigma\in \Nat(D',D)$, then $\colim (D/D') \cong \colim (D)/\colim (D')$.
    \end{lemma}
    \begin{proof}
        Let $D:\N \to \Top$ be a diagram such that $D(i)=X_i$ for every $i\in \N$ with embeddings $\iota_i:X_i\hookrightarrow X_{i+1}$, and $D':\N \to \Top$ is another diagram such that $D(i)=A_i$ for every $i\in \N$  with embeddings $j_i:A_i\hookrightarrow A_{i+1}$, and let $\sigma \in \Nat(D',D)$ be an embedding of diagrams, i.e. $\sigma_i:A_i\hookrightarrow X_i$ is a natural embedding. For each $i\in \N$, let $p_i:X_i \twoheadrightarrow X_i/A_i$ be the quotient map taking $\im \sigma_i$ to a point. For ease of notation, define $X\coloneqq \colim X_i$ and $A\coloneqq \colim A_i$. We observe the following commutative diagram, and in particular, the induced embedding $\kappa:A\hookrightarrow X$:
        \begin{center}
        \begin{tikzcd}
            &&X\\
            \\
            X_i \arrow[hook, bend left=10, crossing over, pos=0.34]{rrrr}[description]{\iota_i} \ar[swap, bend left, hook]{uurr}{f_i}&&A \ar[dashed]{uu}[description]{\exists!}&& X_{i+1} \ar[bend right,hook']{uull}{f_{i+1}}\\
            &A_i\ar[hook]{rr}{j_i}\ar[hook]{ur}{g_i} \ar[hook']{ul}{\sigma_i}&&A_{i+1}\ar[hook']{ul}[swap]{g_{i+1}} \ar[hook]{ur}[swap]{\sigma_{i+1}}
        \end{tikzcd}
    \end{center}
    which commutes because $\sigma$ was natural by assumption. Therefore we let $q:X\twoheadrightarrow X/A$ be the quotient that maps all of $\im \kappa$ to a point. We can create another functor $F:\N \to \Top$ that has objects $X_i/A_i$ and the morphism $\tau_i:X_i/A_i \to X_{i+1}/A_{i+1}$ is defined by the below universal property of $X_i/A_i:$
    \begin{center}
        \begin{tikzcd}
            X_i \ar[hook]{r}{\iota_i} \ar[two heads]{d}{p_i}&X_{i+1} \ar[two heads]{r}{p_{i+1}}&X_{i+1}/A_{i+1}\\
            X_i/A_i \ar[dashed]{urr}[description]{\exists!\tau_i}
        \end{tikzcd}
    \end{center}
    
    
    Now have our two objects of interest in the problem: $X/A$ and $\colim (X_i/A_i)$, defined by the universal properties respectively below:
    \begin{center}
        \begin{tikzcd}
            X \ar[two heads]{d}{q} \ar{r}{\sim}& \bullet\\
            X/A \ar[dashed]{ur}[description]{\exists!}
        \end{tikzcd}
    \end{center}
    \begin{center}
        \begin{tikzcd}
        &\bullet\\
            &\colim (X_i/A_i) \ar[dashed]{u}[description]{\exists!}\\
            X_i/A_i \ar{ur}{h_i} \ar[bend left]{uur} \ar[hook]{rr}{\tau_i}&& X_{i+1}/A_{i+1}\ar{ul}[swap]{h_{i+1}} \ar[bend right]{uul}
        \end{tikzcd}
    \end{center}
    Now we will begin constructing maps via universal properties, and eventually show that the two constructed maps are isomorphisms -- i.e. homeomorphisms. We first notice that 
    \[
    q\circ f_i \circ \sigma_i=q\circ \kappa \circ g_i=c_*\circ g=c_*
    \]
    where for the rest of the homework we let $c_*$ be a constant map. Thus $q\circ f_i$ is constant on $\sigma_i$, hence we obtain a morphism $\varphi_i:X_i/A_i \to X/A$ for each $i\in \N$ given as follows:
    \begin{center}
        \begin{tikzcd}
            X_i \ar{r}{f_i} \ar[two heads]{d}{p_i}& X \ar[two heads]{r}{q}& X/A\\
            X_i/A_i \ar[dashed]{urr}[description]{\exists!\varphi_i}
        \end{tikzcd}
    \end{center}
    Now we observe one more thing:
    \begin{align*}
        \varphi_{i+1}\circ \tau_i\circ p_i\\
        =\varphi_{i+1}\circ p_{i+1}\circ \iota_i\\
        =q\circ f_{i+1}\circ \iota_i\\
        =q\circ f_i\\
        =\varphi_i\circ p_i
    \end{align*}
    which implies, because each $p_i$ is an epimorphism -- i.e. surjective -- that $\varphi_i=\varphi_{i+1}\circ \tau_{i+1}$. Therefore we get the following induced morphism $\Phi:\colim (X_i/A_i)\to X/A$ below:
    \begin{center}
        \begin{tikzcd}
        &X/A\\
            &\colim (X_i/A_i) \ar[dashed]{u}[description]{\exists!\Phi}\\
            X_i/A_i \ar{ur}{h_i} \ar[bend left]{uur}{\varphi_i} \ar[hook]{rr}{\tau_i}&& X_{i+1}/A_{i+1}\ar{ul}[swap]{h_{i+1}} \ar[bend right]{uul}[swap]{\varphi_{i+1}}
        \end{tikzcd}
    \end{center}
    We eventually will show $\Phi$ is an isomorphism. For now though, we turn our attention to the following property:
    \begin{align*}
        h_{i+1}\circ q_{i+1} \circ \iota_i\\
        =h_{i+1}\circ \tau_i \circ p_i\\
        =h_i\circ p_i
    \end{align*}
    by construction of $\tau_i$. Thus we get another morphism $\phi:X\to \colim (X_i/A_i)$ given below:
    \begin{center}
        \begin{tikzcd}
            &&\colim(X_i/A_i)\\
            \\
            X_i \arrow[hook, bend left=10, crossing over, pos=0.34]{rrrr}[description]{\tau_i} \ar[swap, bend left]{uurr}{h_i}&&X \ar[dashed]{uu}[description]{\exists!\phi }&& X_{i+1} \ar[bend right]{uull}{h_{i+1}}\\
            &X_i\ar[hook]{rr}{j_i}\ar[hook]{ur}{f_i} \ar[two heads]{ul}{p_i}&&X_{i+1}\ar[hook']{ul}[swap]{f_{i+1}} \ar[two heads]{ur}[swap]{p_{i+1}}
        \end{tikzcd}
    \end{center}
    We claim that $\phi \circ \kappa=c_*$. To show this, we notice that for $a\in A$, there exists some $a'\in A_i$ for some $i$ such that $a=g_i(a')$, so if we can show that for arbitrary $g_i$ it is constant, we are done. We observe
    \begin{align*}
        \phi \circ \kappa \circ g_i\\
        = \phi \circ f_i\circ \sigma_i\\
        =h_i\circ p_i\circ \sigma_i\\
        =h_i\circ c_*\\
        =c_*
    \end{align*}
    as desired. Therefore $\phi$ descends downstairs to a map $\Psi:X/A\to \colim(X_i/A_i)$ shown below:
    \begin{center}
        \begin{tikzcd}
            X \ar{r}{\phi} \ar[two heads]{d}{q}& \colim (X_i/A_i)\\
            X/A \ar[dashed]{ur}[description]{\exists!\Psi}
        \end{tikzcd}
    \end{center}
    We now claim that $\Phi\circ \Psi=\altid_{X/A}$ and $\Psi \circ \Phi=\altid_{\colim (X_i/A_i)}$. We will use uniqueness of the maps induced by universal properties to prove both. We observe first that
    \begin{align*}
        \Phi \circ \Psi \circ q=\Phi \circ \phi
    \end{align*}
    Now, we realize that every element of $x\in X$ has the property that there exists some $i\in \N$ such that there exists some $x'\in X_i$ where $x=f_i(x')$. Therefore 
    \begin{align*}
        \Phi \circ \phi \circ f_i\\
        =\Phi \circ h_i\circ p_i\\
        =\varphi_i \circ p_i\\
        =q\circ f_i
    \end{align*}
    shows $\Phi\circ \Psi\circ q$ acts the same as $q$ on every $\im f_i$, hence indeed $\Phi \circ \Psi \circ q=q$. Then we observe the following commutative diagram, where the unique arrow is satisfied by both $\Phi \circ \Psi$ and $\altid_{X/A}$, proving, by uniqueness, the two are equal:
    \begin{center}
        \begin{tikzcd}
            X \ar[two heads]{r}{q} \ar[two heads]{d}{q}& X/A\\
            X/A \ar[dashed]{ur}[description]{\exists!}
        \end{tikzcd}
    \end{center}
    For the second claim, we observe
    \begin{align*}
        \Psi \circ \Phi \circ h_i\\
        =\Psi \circ \varphi_i
    \end{align*}
    and
    \begin{align*}
        \Psi \circ \varphi_i \circ p_i\\
        =\Psi \circ q\circ f_i\\
        =\phi \circ f_i\\
        =h_i\circ p_i
    \end{align*}
    The second equation shows, because $p_i$ is an epimorphism, that $h_i=\Psi \circ \varphi_i$, so by the first equation we get $\Psi \circ \Phi\circ h_i=h_i$, thus both $\Psi \circ \Phi$ and $\altid_{\colim(X_i/A_i)}$ satisfy the unique arrow from the universal property below:
    \begin{center}
        \begin{tikzcd}
        &\colim(X_i/A_i)\\
            &\colim (X_i/A_i) \ar[dashed]{u}[description]{\exists!}\\
            X_i/A_i \ar{ur}{h_i} \ar[bend left]{uur}{h_i} \ar[hook]{rr}{\tau_i}&& X_{i+1}/A_{i+1}\ar{ul}[swap]{h_{i+1}} \ar[bend right]{uul}[swap]{h_{i+1}}
        \end{tikzcd}
    \end{center}
    This proves that $\Phi$ (or equivalently $\Psi$) are homeomorphisms, so the claim that $\colim (X_i/A_i)\cong \colim (X_i)/\colim(A_i)$ is true.
    \end{proof}
\section{}
\subsection{}
There are no exercises in this section.
\subsection{}
\subsubsection{A}\label{1.2.A}
\begin{proof}
\begin{enumerate}[(a)]
    \item If we have a groupoid $\mathscr{C}$ with one object $X$, we could define the group of $\mathcal{C}$ to be $\Aut(X)$. On the other hand if we're given a group $G$ by the standard definition, we could define a category with one object, namely the underlying set of $G$, where the morphisms are defined by the action of the elements of $G$, and where composition of morphisms is given by multiplication of the elements.
    \item Consider the following category:
    \begin{center}
        \begin{tikzcd}
        A \arrow[r, leftrightarrow] & B
    \end{tikzcd}
    \end{center}
    This is not a group because it has two objects, or by interpreting the morphisms as the elements of the set, we cannot compose the morphism $A\to B$ with itself, so our operation is not always defined.
\end{enumerate}
\end{proof}
\subsubsection{B}\label{1.2.B}
\begin{proof}
    Since the subcategory of $\fC$ consisting of the single object $A$ and the morphisms $\Aut(A)$ all have inverses, we have a monoid that is also a groupoid, a.k.a. a group.\\
    \newline
    For Example 1.2.2, given any set $S$, $\Aut(S)$ is the set of all bijections from $S$ to itself, a.k.a. the permutation group of $S$.\\
    \newline
    For Example 1.2.3, given any $k$ vector space $V$, $\Aut(V)$ is the set of all bijective linear transformations from $V$ to itself. For $V$ with dimension $n$, these can be interpreted as the group of $n\times n$ matrices with entries in $k$.\\
    \newline
    Suppose $A,B\in \mathscr{C}$ are isomorphic, and let $\varphi\in \Mor(A,B)$ be an isomorphism. For any $f\in \Aut(A)$, we can define a map $\phi:\Aut(A)\to \Aut(B)$ that acts by
    \begin{equation*}
        f\mapsto \varphi \circ f \circ \varphi^{-1}
    \end{equation*}
    To demonstrate $\phi$ is an isomorphism, we need to show it has an inverse. We do this by letting $\Tilde{\phi}:\Aut(B)\to \Aut(A)$ that acts by
    \begin{equation*}
        g\mapsto \varphi^{-1}\circ g\circ \varphi
    \end{equation*}
    Then
    \begin{align*}
        \phi \circ \Tilde{\phi}(g)=\phi(\varphi^{-1}\circ g \circ \varphi)=\varphi \circ (\varphi^{-1}\circ g \circ \varphi)\circ \varphi^{-1}=\id_B \circ g \circ \id_B=g
    \end{align*}
    and 
    \begin{align*}
        \Tilde{\phi}\circ \phi(f)=\Tilde{\phi}(\varphi \circ f \circ \varphi^{-1})=\varphi^{-1}\circ (\varphi \circ f \circ \varphi^{-1}) \circ \varphi=\id_A \circ f \circ \id_A=f
    \end{align*}
    so indeed $\Tilde{\phi}=\phi^{-1}$. Also
    \begin{align*}
        \phi(f\circ g)=\varphi \circ f\circ g \circ \varphi^{-1}=\varphi\circ f \circ \varphi^{-1}\circ \varphi \circ g \circ \varphi^{-1}=\phi(f)\circ \phi(g)
    \end{align*}
    Therefore $\phi$ (and similarly $\phi^{-1}$) preserve compositions of morphisms.
\end{proof}

\subsubsection{C}\label{1.2.C}
\begin{proof}
    We wish to show that the following diagram commutes for all $V,U\in f.d.Vec_k$ and $T\in \Mor(V,U)$:
    \begin{center}
        \begin{tikzcd}
            V \arrow[r, "T"] \arrow [d, "m_V"]& U\arrow[d, "m_U"]\\
            V^{\lor \lor} \arrow[r, "T^{\lor \lor}"]& U^{\lor \lor}
        \end{tikzcd}
    \end{center}
    as well as that $m_V$ is an isomorphism. We first define $m_V:V\to V^{\lor \lor}$ as
    \begin{align*}
        m_V(x)(f)=f(x)
    \end{align*}
    for any $f\in V^\lor$ and any $x\in V$. Then 
    \begin{align*}
        m_V(x+y)(f)=f(x+y)=f(x)+f(y)=(m_V(x)+m_V(y))(f)
    \end{align*}
    and
    \begin{align*}
        m_V(cx)(f)=f(cx)=cf(x)=cm_V(x)(f)
    \end{align*}
    so indeed $m_V\in \Mor(V,V^{\lor \lor})$. 

    
    To construct an inverse to $m_V$, we simultaneously fix bases for all finite dimensional vector spaces so that if $\{e_1,\dots, e_n\}$ be a basis for $V$, we let $\{\epsilon_1, \dots, \epsilon_n\}$ be the corresponding dual basis for $V^{\lor}$, meaning that for each $1\le i,j\le n$,
    \begin{align*}
        \epsilon_i(e_i)=\left\{
    \begin{array}{lr}
        1, & \text{if } i =j\\
        0, & \text{if } i\ne j
    \end{array}
\right\}
    \end{align*}
    %For an arbitrary $\varphi\in V^{\lor \lor}$ we define $c_i\coloneqq \varphi(\epsilon_i)$ for each $1\le i \le n$. Then
    We define $\Tilde{m_V}:V^{\lor \lor}\to V$ as
    \begin{align*}
        \Tilde{m_V}(\varphi)=\sum_{k=1}^n\varphi(\epsilon_i) e_i
    \end{align*}
    Then
    \begin{align*}
        m_V\circ \Tilde{m_V}(\varphi)(\sum_i a_i \epsilon_i)=m_V(\sum_i \varphi(\epsilon_i)e_i)(\sum_i a_i \epsilon_i)=\sum_i a_i \epsilon_i(\sum_j \varphi(\epsilon_j)e_j)\\
        =\sum_i a_i \varphi(\epsilon_i)=\varphi(\sum_i a_i \epsilon_i)
    \end{align*}
    which implies that $m_V\circ \Tilde{m_V}=\id_{V^{\lor \lor}}$. On the other hand for any $\epsilon_j$
    \begin{align*}
        m_V(\sum_i b_ie_i)(\epsilon_j)=\epsilon_j(\sum_i b_ie_i)=\sum_i b_i \epsilon_j(e_i)=b_j
    \end{align*}
    Then we clearly see that 
    \begin{align*}
        \Tilde{m_V}\circ m_V(\sum_i b_ie_i)=\sum_i b_i e_i
    \end{align*}
    so additionally $\Tilde{m_V}\circ m_V=\id_V$ implies that as desired $\Tilde{m_V}=m_V^{-1}$ and that $m_V$ is an isomorphism. Now to prove
    \begin{center}
        \begin{tikzcd}
            V \arrow[r, "T"] \arrow [d, "m_V"]& U\arrow[d, "m_U"]\\
            V^{\lor \lor} \arrow[r, "T^{\lor \lor}"]& U^{\lor \lor}
        \end{tikzcd}
    \end{center}
    commutes, we first observe that for any $\varphi\in V^{\lor \lor}$ and any $g\in U^\lor$, $T^{\lor \lor}(\varphi)(g)=\varphi(g\circ T)$ which makes sense because $g\circ T:V\to k$. Therefore if $\{d_1,\dots, d_m\}$ is a basis for $U$ with dual basis $\{\delta_1,\dots, \delta_m\}$
    \begin{align*}
        m_U\circ T(\sum_i a_i e_i)=m_U(\sum_i a_i Te_i)
    \end{align*}
    and
    \begin{align*}
        m_U(\sum_i a_i Te_i)(\sum_j \alpha_j \delta_j)=\sum_j\alpha_j \delta_j(\sum_i a_i Te_i)\\
        =\sum_j \alpha_j\delta_j(\sum_i a_i \sum_k c_k^i d_k)=\sum_j \alpha_j \sum_i a_i c^i_j
    \end{align*}
    where we have rewritten each
    \[
    Te_i=\sum_{k=1}^m c^i_k d_i
    \]
    On the other hand,
    \begin{align*}
        T^{\lor \lor}(m_V(\sum_i a_i e_i))(\sum_j \alpha_j \delta_j)=m_V(\sum_i a_ie_i)(\sum_j \alpha_j \delta_j\circ T)
        =\sum_j \alpha_j \delta_j \circ T(\sum_i a_ie_i)\\
        =\sum_j \alpha_j \delta_j (\sum_i a_i Te_i)=\sum_j \alpha_j\delta_j (\sum_i a_i \sum_k c^i_k d_k)=\sum_j \alpha_j \sum_i a_i c^i_j
    \end{align*}
    which proves the diagram does indeed commute.
\end{proof}
\subsubsection{D}\label{1.2.D}
\begin{proof}
    First we will simultaneously fix bases for all vector spaces. Then the inverse functor $G:f.d.Vec_k\to \mathscr V$ will map any $V\in f.d.Vec_k$ with dimension $n$ to $k^n$. If $W\in f.d.Vec_k$ has dimension $m$ with fixed basis $\{w_1,\dots,w_m\}$ and $V$ has basis $\{v_1,\dots,v_n\}$, then for any $T:V\to W$ we define $GT\in \Mor(k^n,k^m)$ such that if for each $1\le i \le n$
    \[
    Tv_i=\sum_{j=1}^m c^i_j w_j
    \]
    then
    \[
    GTk_i=\sum_{j=1}^m c^i_j k_j
    \]
    where $\{k_1,\dots, k_n\}$ is a basis for $k^n$ constructed inductively such that for any $k^{n'}\subset k^n$, the basis $\{k_1,\dots, k_{n'}\}$ is a subset of the basis $\{k_1,\dots, k_n\}$.\\
    \newline
    To show $F\circ G$ is naturally isomorphic to $id_{f.d.Vec_k}$, we want to show the following diagram commutes:
    \begin{center}
    \begin{tikzcd}
        V \arrow[r, "T"] \arrow[d,"m_V"]& W \arrow[d,"m_W"]\\
        F\circ G(V) \arrow[r, "F\circ G(T)"] &F\circ G(W)
    \end{tikzcd}
    \end{center}
    where we define $m_V(v_i)=k_i$ for each $1\le i\le n$. $m_V$ is then an isomorphism because its inverse $m_V^{-1}$ is described how you would think: it is the linear map that sends each $k_i$ to $v_i$.  Following the diagram on the bottom,
    \begin{align*}
        F\circ G(T)(m_V(\sum_i a_iv_i))=F\circ G(T)(\sum_i a_ik_i)\\
        =F(\sum_i a_i \sum_j c^i_j k_j)=\sum_i a_i \sum_j c^i_j k_j
    \end{align*}
    On the other hand,
    \begin{align*}
        m_V(T\sum_i a_iv_i)=m_V(\sum_i a_i \sum_j c^i_j w_j)=\sum_i a_i \sum_j c^i_jk_j
    \end{align*}
    so the diagram commutes. To show $G\circ F$ is naturally isomorphic to $id_\mathscr V$, we want to show
    \begin{center}
        \begin{tikzcd}
            k^n \arrow[d, "m_{k^n}"] \arrow [r, "T"]& k^m \arrow[d, "m_{k^m}"]\\
            G\circ F(k^n)\arrow[r, "G\circ F(T)"]& G\circ F(k^m)
        \end{tikzcd}
    \end{center}
    where here $m_{k^n}=\id_{k^n}$. Because $G\circ F(k^n)=k^n$ and preserves bases, the diagram trivially commutes because also $G\circ F(T)=T$.
\end{proof}
\subsection{}
\subsubsection{A}\label{1.3.A}
\begin{proof}
    Suppose both $A$ and $B$ as objects of a category $\fC$ are initial. Then we have
    \begin{center}
        \begin{tikzcd}
            A \arrow[r, "\exists! f", shift left]& B\arrow[l, "\exists! g",shift left]
        \end{tikzcd}
    \end{center}
    because by $A$ being initial $f$ exists and by $B$ being initial $g$ exists. But now we observe
    \begin{center}
        \begin{tikzcd}
            A \arrow[r, "\exists!"]& A
        \end{tikzcd}
    \end{center}
    and because by definition of $\fC$ being a category, $id_A\in \Mor(A,A)$, so the only morphism from $A$ to itself is $\id_A$ by uniqueness. But $f\circ g\in \Mor(A,A)$ implies that $f\circ g=\id_A$. Similarly $g\circ f=\id_B$, so $A\cong B$.\\
    \newline
    If $A,B\in \fC$ are final, then the same diagrams exist but now because $A,B$ are final instead. The same argument holds here.
\end{proof}
\subsubsection{B}\label{1.3.B}
\begin{proof}
    \begin{center}
\begin{tabular}{||c|c|c||} 
 \hline
 Category & Initial Object & Final Object\\ [0.5ex] 
 \hline\hline
 $\Set$ & $\emptyset$ & $\{*\}$ \\ 
 \hline
 $\Ring$ & $\Z$ & $0$ \\
 \hline
 $\Top$& $\emptyset$ & $\{*\}$\\
 \hline
 $\Ssubset(X)$& $\emptyset$ & $X$\\
 \hline
 $\Op(X)$& $\emptyset$& $X$\\
 \hline
\end{tabular}
\end{center}
\end{proof}
\subsubsection{C}\label{1.3.C}
\begin{proof}
    \begin{enumerate}[($\Rightarrow$)]
        \item Assuming $A\hookrightarrow S^{-1}A$, we want to prove $S$ has no zero divisors. Assuming for a contradiction that $s\in S$ is a zero divisor, let $as=0$ for some $a\in A$. Noting that $0\mapsto 0/1$, we also observe that $a\mapsto a/1=0/1$ because $s(1\cdot a-1\cdot 0)=sa=0$. This contradicts the mapping being an injection.
        \item[($\Leftarrow$)]
        Now we assume $S$ has no zero divisors. If $a,b\in A$ are mapped to the same element of $S^{-1}A$, then $a/1=b/1$. This is true if and only if there exists some $s\in S$ such that
        \[
        s(a1-b1)=0\iff s(a-b)=0
        \]
        But $s$ being a non-zero divisor implies that $a-b=0$, hence $a=b$ proving that the canonical map is injective.
    \end{enumerate}
\end{proof}
\subsubsection{D}\label{1.3.D}
\begin{proof}
    Suppose we have an $A$-algebra $B$ such that every element of $A$ is mapped to an invertible element of $B$ via the map $g$. We want to make the following diagram commute:
    \begin{center}
        \begin{tikzcd}
            A \arrow[r] \arrow[rd,"g"]& S^{-1}A \arrow[d, dashed, "\exists!"]\\
            & B
        \end{tikzcd}
    \end{center}
    If we're constructing the unique map $f:S^{-1}A\to B$, by commutativity we have $f(a/1)=g(a)$ for all $a\in A$. Also notice that
    \begin{align*}
        1_B=f(1/1)=f(s/s)=f(s/1)f(1/s)=g(s)f(1/s)
    \end{align*}
    so $f(1/s)=g(s)^{-1}$, which exists since $g$ maps elements of $A$ to invertible elements in $B$. Then
    \[
    f(a/s)=f(a/1)f(1/s)=g(a)g(s)^{-1}
    \]
    means that if $f$ is a morphism, it is uniquely determined by the line above. To show $f$ is linear,
    \begin{align*}
        f(a_1/s_1+a_2/s_2)=f(\frac{s_2a_1+s_1a_2}{s_1s_2})=g(s_2a_1+s_1a_2)g(s_1s_2)^{-1}\\
        =[g(s_2)g(a_1)+g(s_1)g(a_2)]g(s_1)^{-1}g(s_2)^{-1}\\
        =g(a_1)g(s_1)^{-1}+g(a_2)g(s_2)^{-1}=f(a_1/s_1)+f(a_2/s_2)
    \end{align*}
    and
    \begin{align*}
        f(\frac{a_1}{s_1} \frac{a_2}{s_2})=f(\frac{a_1a_2}{s_1s_2})=g(a_1a_2)g(s_1s_2)^{-1}\\
        =g(a_1)g(s_1)^{-1}g(a_2)g(s_2)^{-1}=f(\frac{a_1}{s_1})f(\frac{a_2}{s_2})
    \end{align*}
    which concludes the proof.
\end{proof}
\subsubsection{E}\label{1.3.E}
\begin{proof}
    We will take the construction given in the hint to be $S^{-1}M$ and define the map $\phi:M\to S^{-1}M$ as $m\mapsto \frac{m}{1}$. Clearly this map is an $A$-module map that sends elements of $S$ to invertible elements. We want to show that the following diagram commutes for all $\alpha$ that map elements of $S$ to invertible elements of $N$:
    \begin{center}
        \begin{tikzcd}
            M \arrow[r, "\phi"] \arrow[rd,"\alpha"]& S^{-1}M \arrow[d, dashed, "\exists!"]\\
            & N
        \end{tikzcd}
    \end{center}
    For any such map $\beta:S^{-1}M\to N$, by commutativity we have $\beta(m/1)=\alpha(m)$. We will let $\sigma_s$ to be the isomorphism $s\times \cdot:N\to N$. Then
    \[
    \alpha(m)=\beta(\frac{m}{1})=\beta(s \frac{m}{s})=s\beta(\frac{m}{s})
    \]
    Then applying the isomorphism $\sigma_s^{-1}$ to either side, we get
    \[
    \sigma_s^{-1}\circ \alpha(m)=\beta(\frac{m}{s})
    \]
    which means that if $\beta$ is an $A$-module morphism, then it is uniquely determined by the line above. To show $\beta$ is linear, we see
    \begin{align*}
        \beta(\frac{m_1}{s_1}+\frac{m_2}{s_2})=\beta(\frac{s_2m_1+s_1m_2}{s_1s_2})=\sigma_{s_1s_2}^{-1}\circ \alpha(s_2m_1+s_1m_2)\\
        =\sigma_{s_1s_2}^{-1}(s_2\alpha(m_1)+s_1\alpha(m_2))=\sigma_{s_1}^{-1}\alpha(m_1)+\sigma_{s_2}^{-1}\alpha(m_2)=\beta(\frac{m_1}{s_1})+\beta(\frac{m_2}{s_2})
    \end{align*}
    where we used the fact that $\sigma_{s_1}\circ \sigma_{s_2}=\sigma_{s_1s_2}=\sigma_{s_2s_1}=\sigma_{s_2}\circ \sigma_{s_1}$. Also
    \begin{align*}
        \beta(\frac{a}{s_1} \frac{m}{s_2})=\beta(\frac{am}{s_1s_2})=\sigma_{s_1s_2}^{-1}\alpha(am)=a\sigma_{s_1}^{-1}\circ \sigma_{s_2}^{-1}\alpha(m)=\frac{a}{s_1}\beta(\frac{m}{s})
    \end{align*}
    so $\beta$ is $S^{-1}A$-linear and uniquely satisfies the commutative diagram.
\end{proof}
\subsubsection{F}\label{1.3.F}
\begin{proof}

\begin{enumerate}[(a)]
    \item This is just a special case of the following part of the question.
    \item We define a map $f:S^{-1}\bigoplus M_i\to \bigoplus S^{-1}M_i$ that acts as
    \[
    \frac{(m_i)}{s}\mapsto (\frac{m_i}{s})
    \]
    To show $f$ is linear, we observe
    \begin{align*}
        f(\frac{(m_i)}{s}+\frac{(m_i')}{s'})=f(\frac{(s'm_i+sm_i')}{ss'})=(\frac{s'm_i+sm_i'}{ss'})=(\frac{m_i}{s}+\frac{m_i'}{s'})\\
        =(\frac{m_i}{s})+(\frac{m_i'}{s})=f(\frac{(m_i)}{s})+f(\frac{(m_i')}{s'})
    \end{align*}
    and
    \begin{align*}
        f(\frac{a}{s}\frac{(m_i)}{s'})=f(\frac{(am_i)}{ss'})=(\frac{am_i}{ss'})=\frac{a}{s}(\frac{m_i}{s'})=\frac{a}{s}f(\frac{(m_i)}{s'})
    \end{align*}
    To be completely thorough we should show that $f$ is well defined, but I will not do this for brevity. Now we see that if
    \[
    \frac{(m_i)}{s}\mapsto 0
    \]
    then for each $i$, $\frac{m_i}{s}=0$. This means that for each $m_i$, there exists some $r_i\in S$ such that $r_im_i=0$. But because there are only finitely many $i$, we take $\prod r_i\in S$, and then $(m_i)/s=0$ because
    \[
    \prod r_i(m_i)=(0)=0
    \]
    so $f$ is injective. To show $f$ is surjective, fix any $(\frac{m_i}{s_i})\in \bigoplus S^{-1}M_i$. Then again using the fact that only finitely many $m_i$ are nonzero, we define for each $m_i$ an element $c_i$ of $S$, namely $c_i\coloneqq\prod_{j\ne i} s_j$. Then 
    \[
    f(\frac{(c_i m_i)}{\prod s_i})=(\frac{c_im_i}{\prod s_i})=(\frac{m_i}{s_i})
    \]
    as desired so $f$ is surjective, thus proving $f$ is an isomorphism.
    \item 
    If we let each $M_i=\Z$ and $S=\Z\setminus\{0\}$, then $S^{-1}M_i= \Q$ where we are considering these as $\Z$ modules. Letting $\iota$ be the canonical embedding of $\prod \Z_i \to \prod \Q_i$, then we have
    \begin{center}
        \begin{tikzcd}
            \prod \Z_i \arrow[r, "\phi", hookrightarrow] \arrow[rd, "\iota", hookrightarrow]& S^{-1}\prod \Z_i\arrow[d, dashed, "\exists! \varphi"]\\
            & \prod \Q_i
        \end{tikzcd}
    \end{center}
    However, $\varphi$ does not map to the element $(1,\frac{1}{2},\frac{1}{3},\frac{1}{4},\dots)$. To prove this, we suppose
    \[
    \frac{(n_1,n_2,\dots)}{s}\mapsto (1,\frac{1}{2},\frac{1}{3},\dots)
    \]
    Then
    \[
    s\varphi(\frac{(n_1,n_2,\dots)}{s})=\varphi(s\frac{(n_1,n_2,\dots)}{s})=\varphi(\frac{(n_1,n_2,\dots)}{1})=(n_1,n_2,\dots)
    \]
    But on the other hand, by hypothesis $\varphi(\frac{(n_1,n_2,\dots)}{s})=(1,\frac{1}{2},\dots)$ so also
    \[
    s\varphi(\frac{(n_1,n_2,\dots)}{s})=s(1,\frac{1}{2},\dots)
    \]
    which implies for some nonzero integer $s$, $(s,\frac{s}{2},\frac{s}{3},\dots)=(n_1,n_2,n_3,\dots)$ where each $n_i\in \Z$. This would imply that every prime number $p_i$ divides $s$ because $\frac{s}{p_i}$ would be in the sequence and would have to equal $n_i\in \Z$. But this is a contradiction because there are infinitely many primes, hence no such $s$ can exist. Thus $\varphi$ is not surjective, but $\varphi$ is the unique morphism that preserves the structure of $\prod \Z_i$ which embeds into both $S^{-1}\prod \Z_i$ and $\prod \Q_i$.
\end{enumerate}



\end{proof}
\subsubsection{G}\label{1.3.G}
\begin{proof}
    We have
    \[
    6(1\otimes 2)=1\otimes 12=1\otimes 0=0
    \]
    On the other hand,
    \[
    5(1\otimes 2)=10(1\otimes 1)=10\otimes 1=0\otimes 1=0
    \]
    Therefore
    \[
    1\otimes 1+1\otimes 1=1\otimes 2=(6-5)(1\otimes 2)=6(1\otimes 2)-5(1\otimes 2)=0-0=0
    \]
    We can now see that we only have two elements in $\Z/(10)\otimes_\Z \Z/(12)$, being $0$ and $1\otimes 1$. To show $1\otimes 1\ne 0$, we can show that there is a bilinear map from $\Z/(10)\otimes_\Z \Z/(12)$ to $\Z/(2)$, given by first noticing that any $a\otimes b=ab\otimes 1$, which then we just map $ab\mapsto ab\mod 2$. It's readily verified this is bilinear, and we notice $1\otimes 1\mapsto 1\mod 2$, which is a nonzero element, hence $1\otimes 1\ne 0$ either by the universal property.
\end{proof}
\subsubsection{H}\label{1.3.H}
    For simplicity we will use the facts that the Hom functor is left exact--proven in Exercise \ref{1.6.G}G-- and that for all $A$ modules $M,N,P$
    \[
    \Hom_A(M\otimes N,P)\cong \Hom_A(M,\Hom_A(N,P))
    \]
by Exercise \ref{1.5.D}D. 
\begin{lemma*}
If the sequence $\Hom(C,P)\xrightarrow{g^*}\Hom(B,P)\xrightarrow{f^*}\Hom(A,P)$ is exact for all $P\in \Mod_A$, then $A\xrightarrow{f}B\xrightarrow{g}C$ is exact.
\cite{tensorRexact}
\end{lemma*}
\begin{proof}
    First, we let $P=\cok f=B/ \im f$ and let $\pi$ be the projection from $B$ onto $\cok f$. Then $\pi\in \ker f^*$ because $f^*(\pi)=\pi\circ f=0$, and by exactness $\pi\in \im g^*$. Let $h\in \Hom(C,P)$ such that $g^*(h)=\pi$, or equivalently $h\circ g=\pi$. Then we observe that
    \[
    \ker g\subset \ker \pi=\im f
    \]
    which demonstrates $\ker g\subset \im f$.\\
    To prove the reverse inclusion, we now let $P=C$ and we trace $\id_C$ through the diagram to see
    \[
    0=f^*\circ g^*(\id_C)=f^*(\id_C\circ g)=g\circ f
    \]
    Then clearly $\im f\subset \ker g$. Therefore
    \[
    A\xrightarrow{f}B\xrightarrow{g}C
    \]
    is exact.
\end{proof}
\begin{proof}[Main Result]
    Given $M'\xrightarrow{f} M\xrightarrow{g} M''\to 0$ is exact, we first fix an arbitrary $P\in \Mod_A$ and subsequently apply $\Hom(\cdot,\Hom_A(N,P))$. By left exactness of the Hom functor, the following is exact:
    \[
    0\to \Hom(M'',\Hom_A(N,P))\xrightarrow{g^*} \Hom(M,\Hom_A(N,P))\xrightarrow{f^*} \Hom(M',\Hom_A(N,P))
    \]
    Now we use the fact that $\cdot \otimes N$ is left adjoint to $\Hom(N,\cdot)$ by Exercise \ref{1.5.D}D so that
    \[
    0\to \Hom(M''\otimes N,P))\xrightarrow{(g\otimes N)^*} \Hom(M\otimes N,P))\xrightarrow{(f\otimes N)^*} \Hom(M'\otimes N,P))
    \]
    is exact for all $P$. The lemma yields that
    \[
    M'\otimes N\xrightarrow{f\otimes N} M\otimes N\xrightarrow{g\otimes N}M''\otimes N
    \]
    is exact. Now to show $g\otimes N$ is surjective given $g$ is, fix any $m''\otimes n\in M''\otimes N$. Because $g$ is surjective, let $g(m)=m''$. Then $g\otimes N(m\otimes n)=g(m)\otimes n=m''\otimes n$ proving that $g\otimes N$ is surjective. This completes the proof.
\end{proof}
\subsubsection{I}\label{1.3.I}
\begin{proof}
    In this category, the objects are pairs $(T,t:M\times N\to T)$ such that $t$ is bilinear, and a morphism $f:T\to T'$ is a morphism of $A$ modules such that $f\circ t=t'$. Defining the tensor product to be the initial objects of this category, by the fact that any initial object in a category is unique up to unique isomorphism, we get the desired result. But for a more concrete proof, suppose we have $(T,t)$ and $(T',t')$ both satisfying the definition of tensor product. Then
    \begin{center}
        \begin{tikzcd}
            M\times N\arrow[rr,"t"] \arrow[dr,"t'"]& & T \arrow[dl, "\exists!f", dashed]\\
            & T'&
        \end{tikzcd}
    \end{center}
    and also
    \begin{center}
        \begin{tikzcd}
            M\times N\arrow[rr,"t'"] \arrow[dr,"t"]& & T' \arrow[dl, "\exists!g", dashed]\\
            & T&
        \end{tikzcd}
    \end{center}
    commute. On the other hand,
    \begin{center}
        \begin{tikzcd}
            M\times N\arrow[rr,"t"] \arrow[dr,"t"]& & T \arrow[dl, "\exists!", dashed]\\
            & T&
        \end{tikzcd}
    \end{center}
    means that $\id_T$ satisfies the definition, as well as $g\circ f$ because $g\circ f\circ t=g\circ t'=t$, so by uniqueness $\id_T=g\circ f$, and a similar argument shows $f\circ g=\id_{T'}$.\\
    \newline
    We could define the product in any category $\fC$ to be the final object in a category whose objects are pairs $(P,p_M,p_N)$ where $P\in \fC$, $p_M\in \Mor_\fC(P,M)$ and $p_N\in \Mor_\fC(P,N)$. The morphisms of the category are morphisms $f\in \Mor_\fC(P',P)$ such that $p_M'=p_M\circ f$ and $p_N'=p_N\circ f$. Again, any final object in a category is unique up to unique isomorphism, so the product is defined up to unique isomorphism.
\end{proof}
\subsubsection{J}\label{1.3.J}
\begin{proof}
    Suppose we have some pair $(T,t)$ as in the previous exercise. To show that
    \begin{center}
        \begin{tikzcd}
            M\times N\arrow[rr,"\phi"] \arrow[dr,"t"]& & M\otimes N \arrow[dl, "\exists!", dashed]\\
            & T&
        \end{tikzcd}
    \end{center}
    If any such $\varphi:M\otimes N\to T$ exists that makes the diagram commute, then by definition
    \[
    \varphi(m\otimes n)=t(m,n)
    \]
    Notice this proves that if $\varphi$ exists, it is unique. To show this $\varphi\in \Hom_A(M\otimes N,T)$ we first need to show that it is well defined. Letting $R$ be the linear subspace of the free module $F(M\times N)$ spanned by all elements of the form
    \begin{align*}
        (m_1+m_2,n)-(m_1,n)-(m_2,n)\\
        (m,n_1+n_2)-(m,n_1)-(m,n_2)\\
        (am,n)-a(m,n)\\
        (m,an)-a(m,n)
    \end{align*}
    we formally have any tensor $m\otimes n=(m,n)+R$ as a coset. But for each basis element $x$ of $R$, we notice that $f(x)=0$ because $f$ is bilinear, so $f\equiv 0$ on $R$. Thus it doesn't matter which representative of the coset $(m,n)+R$ we pick, so $f$ is well defined. 
    To check linearity,
    \[
    \varphi(m_1+m_2\otimes n)=f(m_1+m_2,n)=f(m_1,n)+f(m_2,n)=\varphi(m_1\otimes n)+\varphi(m_2\otimes n)
    \]
    and similarly
    \[
    \varphi(m\otimes n_1+n_2)=f(m,n_1+n_2)=f(m,n_1)+f(m,n_2)=\varphi(m\otimes n_1)+\varphi(m\otimes n_2)
    \]
    and
    \[
    \varphi(am\otimes n)=f(am,n)=af(m,n)=a\varphi(m\otimes n)
    \]
    There are no other linearity relations on $M\otimes N$, so $\varphi$ must be linear on other sums of tensors; indeed $\varphi$ is an $A$-module homomorphism and is the unique one making the diagram commute.
\end{proof}
\subsubsection{K}\label{1.3.K}
\begin{proof}
    \begin{enumerate}[(a)]
        \item 

        We define scalar multiplication by first constructing a bilinear map $\varphi_b:B\times M\to B\otimes_A M$ for each $b\in B$ given by 
    \[
    \varphi_b(b',m)=bb'\otimes m
    \]
    To prove $\varphi_b$ is bilinear,
    \[
    \varphi_b(b_1+b_2,m)=b(b_1+b_2)\otimes m=bb_1+bb_2\otimes m=bb_1\otimes m+bb_2\otimes m=\varphi_b(b_1,m)+\varphi_b(b_2,m)
    \]
    and
    \[
    \varphi_b(b',m_1+m_2)=bb'\otimes m_1+m_2=bb'\otimes m_1+bb'\otimes m_2=\varphi_b(b',m_1)+\varphi_b(b',m_2)
    \]
    as well as
    \[
    \varphi_b(ab',m)=bab'\otimes m=bb'\otimes am=\varphi_b(b',am)=a\varphi_b(b',m)
    \]
    Thus
    \begin{center}
        \begin{tikzcd}
            B\times M \arrow[r] \arrow[dr, "\varphi_b"]& B\otimes_A M \arrow[d, dashed, "\exists!\phi_b"]\\
            & B\otimes_A M
        \end{tikzcd}
    \end{center}
    commutes and we define scalar multiplication by $b$ as the function $\phi_b$. It's easy to verify that
    \[
    \phi_{b_1}\circ \phi_{b_2}(b\otimes m)=\phi_{b_1b_2}(b\otimes m)
    \]
    Thus
    \[
    b_1(b_2 b\otimes m)=(b_1 b_2) b\otimes m
    \]
    By $\phi_b$ being $A$-linear,
    \[
    b(b_1\otimes m_1+b_2\otimes m_2)=b b_1\otimes m_1+b b_2\otimes m_2
    \]
    Also
    \begin{align*}
        \phi_{b_1+b_2}(b\otimes m)=(b_1+b_2)b\otimes m=b_1b+b_2b\otimes m=b_1b\otimes m+b_2b\otimes m\\
        =\phi_{b_1}(b\otimes m)+\phi_{b_2}(b\otimes m)
    \end{align*}
    And finally
    \[
    \phi_1(b\otimes m)=1b\otimes m=b\otimes m
    \]
    so we have indeed defined a $B$-module structure on $B\otimes_A M$. To see that this defines a functor, we want to show that for any $X,Y,Z\in mod_A$, $f\in \Hom_A(X,Y)$ and $g\in \Hom_A(Y,Z)$, that $B\otimes g\circ f=B\otimes g\circ B\otimes f$ where we define $B\otimes f$ as the induced map in the following commutative diagram:
    \begin{center}
        \begin{tikzcd}
            B\times X\arrow[r,"\phi_X"] \arrow[d,"\id_B\times f"]&B\otimes_A X \arrow[d, dashed, "\exists!"]\\
            B\times Y\arrow[r, "\phi_Y"]&B\otimes_A Y
        \end{tikzcd}
    \end{center}
    To be thorough we should prove that $\phi_Y\circ \id_B\times f$ is bilinear, but it is readily verifiable because $f$ is $A$-linear. Now to show that this functor respects compositions, we see
    \[
    B\otimes g\circ B\otimes f(b\otimes x)=B\otimes g(b\otimes f(x))=b\otimes g\circ f(x)=B\otimes g\circ f(b\otimes x)
    \]
    so we do have a functor from $mod_A\to mod_B$.

    \item 

    We have a similar approach for the construction of multiplication: for all $b\in B$ and $c\in C$, we define a map $\varphi_{b,c}:B\times C\to B\otimes_A C$ as
    \[
    \varphi_{b,c}(b',c')=bb'\otimes cc'
    \]
    To show $\varphi_{b,c}$ is $A$-bilinear, we see
    \[
    \varphi_{b,c}(b_1+b_2,c')=b(b_1+b_2)\otimes cc'=bb_1\otimes cc'+bb_2\otimes cc'=\varphi_{b,c}(b_1,c)+\varphi_{b,c}(b_2,c')
    \]
    and
    \[
    \varphi_{b,c}(b',c_1+c_2)=bb'\otimes c(c_1+c_2)=bb'\otimes cc_1+bb'\otimes cc_2=\varphi_{b,c}(b',c_1)+\varphi_{b,c}(b',c_2)
    \]
    as well as
    \[
    \varphi_{b,c}(ab',c')=bab'\otimes cc'=bb'\otimes cac'=\varphi_{b,c}(b',ac')=a\varphi_{b,c}(b',c')
    \]
    \end{enumerate}
    Then we get a commutative diagram induced by the universal property:
    \begin{center}
        \begin{tikzcd}
            B\times C \arrow[r] \arrow[dr, "\varphi_{b,c}" below]& B\otimes_A C \arrow[d, dashed, "\exists!\phi_{b,c}"]\\
            & B\otimes_A C
        \end{tikzcd}
    \end{center}
    Then we use the action of $\phi_{b,c}$ to be multiplication by $b\otimes c$. Therefore
    \begin{align*}
        \phi_{b,c}\circ \phi_{b',c'}(b''\otimes c'')=\phi_{b,c}(b'b''\otimes c'c'')=bb'b''\otimes cc'c''=\phi_{bb',cc'}(b''\otimes c'')
    \end{align*}
    so multiplication is associative. The multiplicative identity is
    \[
    \phi_{1,1}(b\otimes c)=1b\otimes 1c=b\otimes c
    \]
    To show multiplication is distributive,
    \begin{align*}
        \phi_{b,c}(b_1\otimes c_1+b_2\otimes c_2)=\phi_{b,c}(b_1\otimes c_1)+\phi_{b,c}(b_2\otimes c_2)
    \end{align*}
    because $\phi_{b,c}$ is $A$-linear. We notice that because $B,C$ are commutative rings,
    \begin{align*}
        (b_1\otimes c_1)(b_2\otimes c_2)=b_1b_2\otimes c_1c_2=b_2b_1\otimes c2c_1=(b_2\otimes c_2)(b_1\otimes b_1)
    \end{align*}
    implying that multiplication is also right distributive since
    \[
    (b_1\otimes c_1+b_2\otimes c_2)(b\otimes c)=(b\otimes c)(b_1\otimes c_1+b_2\otimes c_2)=(b\otimes c)(b_1\otimes c_1)+(b\otimes c)(b_2\otimes c_2)
    \]
    \[
    =(b_1\otimes c_1)(b\otimes c)+(b_2\otimes c_2)(b\otimes c)
    \]
    This completes the verification of the ring axioms, so indeed $B\otimes_A C$ is a ring.
\end{proof}
\subsubsection{L}\label{1.3.L}
\begin{proof}
    We will use the universal property of tensor products to construct a map $\beta:S^{-1}A\otimes_A M\to S^{-1}M$. We define a map $\alpha:S^{-1}A\times M\to S^{-1}M$ given by $\alpha(\frac{a}{s},m)=\frac{am}{s}$. To show $\alpha$ is $A$ bilinear, we see
    \begin{align*}
        \alpha(\frac{a_1}{s_1}+\frac{a_2}{s_2},m)=\alpha(\frac{s_2a_1+s_1a_2}{s_1s_2},m)=\frac{(s_2a_1+s_1a_2)m}{s_1s_2}=\frac{a_1m}{s_1}+\frac{a_2m}{s_2}=\alpha(\frac{a_1}{s_1},m)+\alpha(\frac{a_2}{s_2},m)
    \end{align*}
    and
    \begin{align*}
        \alpha(\frac{a}{s},m_1+m_2)=\frac{a(m_1+m_2)}{s}=\frac{am_1}{s}+\frac{am_2}{s}=\alpha(\frac{a}{s},m_1)+\alpha(\frac{a}{s},m_2)
    \end{align*}
    as well as
    \begin{align*}
        \alpha(a' \frac{a}{s},m)=\alpha(\frac{a'a}{s},m)=\frac{a'am}{s}=a'\frac{am}{s}=a'\alpha(\frac{a}{s},m)=\frac{aa'm}{s}=\alpha(\frac{a}{s},a'm)
    \end{align*}
    Then $\alpha$ is $A$ bilinear, and hence we get an induced map $\beta:S^{-1}A\otimes_A M\to S^{-1}M$ from the below diagram:
    \begin{center}
        \begin{tikzcd}
            S^{-1}A\times M \arrow[r] \arrow[dr, "\alpha" below]& S^{-1}A\otimes_A M \arrow[d, dashed, "\exists!"]\\
            & S^{-1}M
        \end{tikzcd}
    \end{center}
    We will now construct an inverse to $\beta$. Let $\phi(m)=1\otimes m$. This is clearly $A$ bilinear as well so we obtain a unique $\varphi:S^{-1}M\to S^{-1}A\otimes_A M$
    \begin{center}
        \begin{tikzcd}
            M \arrow[r] \arrow[dr, "\phi" below]& S^{-1}M \arrow[d, dashed, "\exists!"]\\
            & S^{-1}A\otimes_A M
        \end{tikzcd}
    \end{center}
    Then
    \[
    \varphi \circ \beta(\frac{a}{s}\otimes m)=\varphi(\frac{am}{s})=\frac{1}{s}\otimes am=\frac{a}{s}\otimes m
    \]
    and
    \[
    \beta \circ \varphi(\frac{m}{s})=\beta(\frac{1}{s}\otimes m)=\frac{m}{s}
    \]
    so indeed $S^{-1}M\cong S^{-1}A\otimes_A M$ as $A$-modules. We can extend the $A$-module structure into a $S^{-1}A$ module structure by the previous exercise, and in fact the same morphisms we just constructed can be considered to be $S^{-1}A$ linear as well. We can see this by
    \[
    \beta(\frac{a}{s}\frac{a'}{s'}\otimes m)=\frac{a'am}{ss'}=\frac{a'}{s'}\frac{am}{s}=\frac{a'}{s'}\beta(\frac{a}{s}\otimes m)
    \]
    and
    \[
    \varphi(\frac{a}{s}\frac{m}{s'})=\varphi(\frac{am}{ss'})=\frac{1}{ss'}\otimes am=\frac{a}{s}\frac{1}{s'}\otimes m=\frac{a}{s}\varphi(\frac{m}{s'})
    \]
    Therefore they are also isomorphic as $S^{-1}A$ modules as well.
\end{proof}
\subsubsection{M}\label{1.3.M}
\begin{proof}
    We will use the universal property to construct our desired map. We define $\alpha:M\times \bigoplus_{i\in I} N_i\to \bigoplus_{i\in I} M\otimes N_i$ where
    \[
    \alpha(m,\sum_i n_i)=\sum_i m\otimes n_i
    \]
    To verify $\alpha$ is $A$-bilinear,
    \begin{align*}
        \alpha(m_1+m_2,\sum_i n_i)=\sum_i m_1+m_2\otimes n_i=\sum_i m_1\otimes n_i+\sum_i m_2\otimes n_i\\
        =\alpha(m_1,\sum_i n_i)+\alpha(m_2,\sum_i n_i)
    \end{align*}
    and
    \begin{align*}
        \alpha(m,\sum_i n_i+\sum_i n_i')=\alpha(m,\sum_i n_i+n_i')=\sum_i m\otimes n_i+n_i'\\
        =\sum m\otimes n_i+\sum_i m\otimes n_i'=\alpha(m,\sum_i n_i)+\alpha(m,\sum_i n_i')
    \end{align*}
    as well as
    \begin{align*}
        \alpha(am,\sum_i n_i)=\sum_i am\otimes n_i=a\sum_i m\otimes n_i=\sum_i m\otimes an_i=\alpha(m,a\sum_i n_i)
    \end{align*}
Then let $\varphi$ be the unique induced map below:
\begin{center}
        \begin{tikzcd}
            M\times \bigoplus_i N_i \arrow[r] \arrow[dr, "\alpha" below]& M\otimes_A \bigoplus_i N_i \arrow[d, dashed, "\exists!"]\\
            & \bigoplus_i M\otimes_A N_i
        \end{tikzcd}
    \end{center}
    Then $\varphi(m\otimes \sum_i n_i)=\sum_i m\otimes n_i$, and the inverse map $\phi$ is defined as $\phi(\sum_i m\otimes n_i)=m\otimes \sum_i n_i$. The construction of $\phi$ follows from the diagram below:
    \begin{center}
        \begin{tikzcd}
            \bigoplus M\otimes N_i \arrow[r,"\pi_j"] &M\otimes N_j \arrow[r, dashed, "\exists! \varphi_j"] & M\otimes \bigoplus N_i \\
            & M\times N_j\arrow[u]\arrow[ur,"\alpha_j", right]&
        \end{tikzcd}
    \end{center}
    and then defining $\phi=\sum_i \varphi_j\circ \pi_j$ which is well defined because all but finitely many of the $\pi_j$ are nonzero for any given element. These are clearly inverses, hence $M\otimes \bigoplus N_i\cong \bigoplus M\otimes N_i$ as $A$-modules.
\end{proof}
\subsubsection{N}\label{1.3.N}
\begin{proof}
    Letting $S=\{(x,y)\in X\times Y: \alpha(x)=\beta(y)\}$ with the obvious projection maps $\pi_X$ and $\pi_Y$, it is immediate that
    \begin{center}
        \begin{tikzcd}
            S \arrow[r,"\pi_Y"] \arrow[d, "\pi_X"]& Y\arrow[d,"\beta"]\\
            X\arrow[r,"\alpha"]& Z
        \end{tikzcd}
    \end{center}
    commutes by construction of $S$. Now suppose we're given the following commutative diagram:
    \begin{center}
        \begin{tikzcd}
            W \arrow[r,"p_Y"] \arrow[d, "p_X"]& Y\arrow[d,"\beta"]\\
            X\arrow[r,"\alpha"]& Z
        \end{tikzcd}
    \end{center}
    We want to show that
    \begin{center}
        \begin{tikzcd}
            W \arrow[dr, dashed, "\exists!"] \arrow[ddr, bend right=30, "p_X"] \arrow[drr, bend left=30, "p_Y"]&&\\
            &S \arrow[r,"\pi_Y"] \arrow[d, "\pi_X"]& Y\arrow[d,"\beta"]\\
            &X\arrow[r,"\alpha"]& Z
        \end{tikzcd}
    \end{center}
    commutes for some unique map $\varphi:W\to S$. Any such map $\varphi$ that makes the diagram commute has $\pi_X\circ \varphi=p_X$ and $\pi_Y\circ \varphi=p_Y$. It's then clear that if $\varphi(w)=(\varphi_X(w), \varphi_Y(w))$ for all $w\in W$, that then $\varphi_X=p_X$ and $\varphi_Y=p_Y$. Thus uniqueness is proven, and the fact that $\varphi$ makes the diagram commute is trivial so indeed $S=X\times_Z Y$.
\end{proof}
\subsubsection{O}\label{1.3.O}
\begin{proof}
    We claim that if $A,B,C\in Op(X)$ such that $A,B\subset C$, then $A\times_C B=A\cap B$. In $Op(X)$, we observe that there is at most one arrow from any object to any other object, so we needn't prove uniqueness in the universal property argument. It is clear that
    \begin{center}
        \begin{tikzcd}
            A\cap B \arrow[r] \arrow[d]& B\arrow[d]\\
            A\arrow[r]& C
        \end{tikzcd}
    \end{center}
    commutes--notice that commutativity here is just saying that every element of $A\cap B$ is an element of $C$ because the morphisms are inclusions. If we have another open set $W$ such that $W\subset A$ and $W\subset B$, it's clear that every element of $W$ must be an element of $A\cap B$ by definition, which proves that
    \begin{center}
        \begin{tikzcd}
            W \arrow[dr, dashed, "\exists!"] \arrow[ddr, bend right=30] \arrow[drr, bend left=30]&&\\
            &A\cap B \arrow[r] \arrow[d]& B\arrow[d]\\
            &A\arrow[r]& C
        \end{tikzcd}
    \end{center}
    commutes as well, thus proving $A\times_C B=A\cap B$.
\end{proof}
\subsubsection{P}\label{1.3.P}
\begin{proof}
    First of all, given the following data
    \begin{center}
        \begin{tikzcd}
            &W\arrow[d,dashed, "\exists!"]\arrow[bend right]{ddl}[swap]{p_X} \arrow[ddr, bend left=30, "p_Y"]&\\
            &X\times Y\arrow[dl, "\pi_X"] \arrow{dr}[swap]{\pi_Y}&\\
            X& &Y
        \end{tikzcd}
    \end{center}
    we have our unique morphism $\varphi:W\to X\times Y$ that makes the diagram above commute. But by $Z$ being final, there is only one morphism from any object to $Z$, hence the entire diagram below commutes:
    \begin{center}
        \begin{tikzcd}
W\arrow[bend left]{drr}{p_X}
\arrow[bend right]{ddr}[swap]{p_Y}
\arrow[dashed]{dr}[description]{\exists!} & & \\
& X \times Y \arrow{r}[swap]{\pi_Y} \arrow{d}{\pi_X} & Y \arrow{d} \\
& X \arrow{r} & Z
\end{tikzcd}
    \end{center}
    Therefore $X\times Y$ satisfies the definition of $X\times_ZY$, and by the standard universal property argument, are defined up to unique isomorphism.
\end{proof}
\subsubsection{Q}\label{1.3.Q}
\begin{proof}
    The path traced in red below
    \begin{center}
        \begin{tikzcd}
            U\ar[r, red] \ar[d]&V\ar[d,red]\\
            W\ar[r] \ar[d]&X \ar[d,red]\\
            Y\ar[r]&Z
        \end{tikzcd}
    \end{center}
    is equal to
    \begin{center}
        \begin{tikzcd}
            U\ar[r] \ar[d,red]&V\ar[d]\\
            W\ar[r,red] \ar[d]&X \ar[d,red]\\
            Y\ar[r]&Z
        \end{tikzcd}
    \end{center}
    by commutativity of the top square, and by commutativity of the bottom square 
    \begin{center}
        \begin{tikzcd}
            U\ar[r] \ar[d,red]&V\ar[d]\\
            W\ar[r] \ar[d, red]&X \ar[d]\\
            Y\ar[r, red]&Z
        \end{tikzcd}
    \end{center}
    Now we need to show 
    \begin{center}
        \begin{tikzcd}
P\arrow[bend left]{drr}{\chi_2}
\arrow[bend right]{ddr}[swap]{\chi_1}
\arrow[dashed]{dr}[description]{\exists!} & & \\
& U \arrow{r} \arrow{d} & V \arrow{d} \\
& Y \arrow{r} & Z
\end{tikzcd}
    \end{center}
    commutes. We will first use the fact that the lower square is universal to get
    \begin{center}
        \begin{tikzcd}
P\arrow{rr}{\chi_2}
\arrow[bend right]{ddr}[swap]{\chi_1}
\arrow[dashed]{dr}[description]{\exists!\varphi} & & V\ar[d] \\
& W \arrow{r} \arrow{d} & X \arrow{d} \\
& Y \arrow{r} & Z
\end{tikzcd}
    \end{center}
    Now we use this $\varphi:P\to W$ with the universal property of the top square to get
    \begin{center}
        \begin{tikzcd}
P\arrow[bend left]{drr}{\chi_2}
\arrow[bend right]{ddr}[swap]{\varphi}
\arrow[dashed]{dr}[description]{\exists!\phi} & & \\
& U \arrow{r} \arrow{d} & V \arrow{d} \\
& W \arrow{r} & X
\end{tikzcd}
    \end{center}
    It is easily checked that $\phi$ is the desired morphism that makes the original diagram commute, so we have shown that the tower is indeed a Cartesian diagram.
\end{proof}
\subsubsection{R}\label{1.3.R}
\begin{proof}
    We have
    \begin{center}
        \begin{tikzcd}
            X_1\times_Y X_2\ar[r,"\iota_2"] \ar[d,"\iota_1"]& X_2\ar[d,"g"]&\\
            X_1\ar[r,"f"]&Y\ar[dr,"h"]&\\
            && Z
        \end{tikzcd}
    \end{center}
    commuting, then
    \begin{center}
        \begin{tikzcd}
X_1\times_Y X_2\arrow[bend left]{drr}{\iota_2}
\arrow[bend right]{ddr}[swap]{\iota_1}
\arrow[dashed]{dr}[description]{\exists!} & & \\
& X_1\times_ZX_2 \arrow[r] \arrow{d} & X_2 \arrow{d}{h\circ g} \\
& X_1 \arrow{r}{h\circ f} & Z
\end{tikzcd}
    \end{center}
    induces the unique natural morphism demonstrated here.
\end{proof}
\subsubsection{S}\label{1.3.S}
\begin{proof}
    From the previous exercise, the following diagram commutes
    \begin{center}
        \begin{tikzcd}
X_1\times_Y X_2\arrow[bend left]{drr}{\tau_2}
\arrow[bend right]{ddr}[swap]{\tau_1}
\arrow[dashed]{dr}[description]{\exists! \varphi} & & \\
& X_1\times_ZX_2 \arrow[r,"\pi_2"] \arrow{d}{\pi_1} & X_2 \arrow{d}{h\circ g} \\
& X_1 \arrow{r}{h\circ f} & Z
\end{tikzcd}
    \end{center}
    because
    \begin{center}
        \begin{tikzcd}
            X_1\times_Y X_2\arrow{r}{\tau_2} \arrow{d}{\tau_1}& X_2 \arrow{d}{g}\\
            X_1 \arrow{r}{f}& Y
        \end{tikzcd}
    \end{center}
    commutes.
    There is another natural map $\alpha:Y\to Y\times_ZY$ given by
    \begin{center}
        \begin{tikzcd}
Y\arrow[bend left]{drr}{\id_Y}
\arrow[bend right]{ddr}[swap]{\id_Y}
\arrow[dashed]{dr}[description]{\exists! \alpha} & & \\
& Y\times_ZY \arrow{r}{\mu_2} \arrow{d}{\mu_1} & Y \arrow{d}{h} \\
& Y \arrow{r}{h} & Z
\end{tikzcd}
    \end{center}
    Additionally, there is a map $\theta:X_1\times_Z X_2\to Y\times_Z Y$ given by
    \begin{center}
        \begin{tikzcd}
X_1\times_Z X_2\arrow[bend left]{drr}{g\circ \pi_2}
\arrow[bend right]{ddr}[swap]{f\circ \pi_1}
\arrow[dashed]{dr}[description]{\exists!\theta} & & \\
& Y\times_ZY \arrow[r, "\mu_2"] \arrow{d}{\mu_1} & Y \arrow{d}{h} \\
& Y \arrow{r}{h} & Z
\end{tikzcd}
    \end{center}
    Then we would first like to show the following diagram commutes:
    \begin{center}
        \begin{tikzcd}
            X_1\times_YX_2 \arrow{r}{\varphi}\arrow{d}{f\circ \tau_1}& X_1\times_Z X_2 \arrow{d}{\theta}\\
            Y \arrow{r}{\alpha}& Y\times_ZY
        \end{tikzcd}
    \end{center}
    To do this, we will turn to the following commutative diagram:
    \begin{center}
        \begin{tikzcd}
X_1\times_Y X_2\arrow[bend left]{drr}{g\circ \pi_2\circ \varphi}
\arrow[bend right]{ddr}[swap]{f\circ \pi_1\circ \varphi}
\arrow[dashed]{dr}[description]{\exists!} & & \\
& Y\times_ZY \arrow[r, "\mu_2"] \arrow{d}{\mu_1} & Y \arrow{d}{h} \\
& Y \arrow{r}{h} & Z
\end{tikzcd}
    \end{center}
    On one hand, we will show that $\alpha\circ f\circ \tau_1$ makes the diagram commute. We observe
    \[
    \mu_1\circ \alpha\circ f\circ \tau_1=f\circ \tau_1=f\circ \pi_1\circ \varphi
    \]
    as well as
    \[
    \mu_2\circ \alpha \circ f \circ \tau_1=\mu_2\circ \alpha\circ g\circ \tau_2=g\circ \tau_2=g\circ \pi_2\circ \varphi
    \]
    On the other hand, we will show that $\theta\circ \varphi$ makes the diagram commute. We have
    \[
    \mu_1\circ \theta\circ \varphi=f\circ \pi_1\circ \varphi
    \]
    and also
    \[
    \mu_2\circ \theta \circ \varphi=g\circ \pi_2\circ \varphi
    \]
    By uniqueness of the induced map, we obtain that indeed $\theta\circ \varphi=\alpha \circ f\circ \tau_1$. Now we need to show the square is universal. Suppose the following diagram commutes:
    \begin{center}
        \begin{tikzcd}
P\arrow[bend left]{drr}{p_2}
\arrow[bend right]{ddr}[swap]{p_1}
 & & \\
& X_1\times_Y X_2 \arrow[r, "\varphi"] \arrow{d}{f\circ \tau_1} & X_1\times_ZX_2 \arrow{d}{\theta} \\
& Y \arrow{r}{\alpha} & Y\times_Z Y
\end{tikzcd}
    \end{center}
    Then 
    \[
    \alpha\circ p_1=\theta\circ p_2  
    \]
    implies, by applying $\mu_1$ or $\mu_2$ to the left of each equation,
    \[
    p_1=\mu_1\circ \theta \circ p_2=\mu_2\circ \theta \circ p_2
    \]
    which is true by definition if and only if
    \[
    p_1=f\circ \pi_1\circ p_2=g\circ \pi_2\circ p_2
    \]
    Therefore the following diagram commutes:
    \begin{center}
        \begin{tikzcd}
P\arrow[bend left]{drr}{\pi_2\circ p_2}
\arrow[bend right]{ddr}[swap]{\pi_1\circ p_2}
\arrow[dashed]{dr}[description]{\exists!} & & \\
& X_1\times_Y X_2 \arrow[r, "\tau_2"] \arrow{d}{\tau_1} & X_2 \arrow{d}{g} \\
& X_1 \arrow{r}{f} & Y
\end{tikzcd}
    \end{center}
    Let $\chi$ be the induced map, which proves uniqueness. We will show that $\chi$ makes the following diagram commute:
    \begin{center}
        \begin{tikzcd}
P\arrow[bend left]{drr}{p_2}
\arrow[bend right]{ddr}[swap]{p_1}\arrow{dr}{\chi}
 & & \\
& X_1\times_Y X_2 \arrow[r, "\varphi"] \arrow{d}{f\circ \tau_1} & X_1\times_ZX_2 \arrow{d}{\theta} \\
& Y \arrow{r}{\alpha} & Y\times_Z Y
\end{tikzcd}
    \end{center}
    To show $f\circ \tau_1\circ \chi=p_1$, we have
    \[
    f\circ \tau_1\circ \chi=f\circ \pi_1\circ p_2=p_1
    \]
    To show $\varphi \circ \chi=p_2$, we will show that both $\varphi\circ \chi$ and $p_2$ satisfies the following induced map:
    \begin{center}
        \begin{tikzcd}
P\arrow[bend left]{drr}{\pi_2\circ p_2}
\arrow[bend right]{ddr}[swap]{\pi_1\circ p_2} \arrow[dashed]{dr}[description]{\exists!}
 & & \\
& X_1\times_Z X_2 \arrow[r, "\pi_2"] \arrow{d}{\pi_1} & X_2 \arrow{d}{h\circ g} \\
& X_1 \arrow{r}{h\circ f} & Y\times_Z Y
\end{tikzcd}
    \end{center}
    It's obvious that $p_2$ makes the diagram commute. On the other hand,
    \[
    \pi_1\circ \varphi \circ \chi=\tau_1\circ \chi=\pi_1\circ p_2
    \]
    as well as
    \[
    \pi_2\circ \varphi\circ \chi=\tau_2\circ \chi=\pi_2\circ p_2
    \]
    Since both make the diagram commute, by uniqueness, $\varphi\circ \chi=p_2$ which completes the proof.
\end{proof}
\subsubsection{T}\label{1.3.T}
\begin{proof}
    Given an indexed family of sets $A_i$ for $i\in I$, the disjoint union is the set 
    \[
    \coprod_{i\in I} A_i=\bigcup_{i\in I}\{(x,i):x\in A_i\}
    \]
    where each $A_i$ is equipped with a map $\iota_i:A_i\to \coprod_i A_i$ such that
    \[
    \iota_i(x)=(x,i)
    \]
    Now we suppose we have a set $P$ such that for each $i\in I$, there is a map $p_i:A_i\to P$. Then
    
    \begin{center}
        \begin{tikzcd}
            P\\
            \coprod_i A_i\arrow[dashed]{u}{\exists!}\\
            A_i \arrow{u}{\iota_i} \arrow[bend left=50]{uu}{p_i}
        \end{tikzcd}
    \end{center}
    where the unique map $\varphi$ is defined by $\varphi(x,i)=p_i(x)$. This definition is given to us by commutativity, so uniqueness is proven, and the construction is well defined because $\iota_i$ is an injection for each $i\in I$, which proves existence so indeed the disjoint union is the coproduct in $Sets$.
\end{proof}
\subsubsection{U}\label{1.3.U}
\begin{proof}
    Suppose $\beta:A\to B$ and $\gamma:A\to C$ are ring morphisms and \\$\varphi:B\to B\otimes C$ and $\phi:C\to B\otimes C$ are as defined in the exercise. To show $\varphi$ is a ring morphism, we recall that $B$ can be considered an $A$-module where scalar multiplication is defined as $a\cdot b\coloneqq \beta(a) b$. We immediately get $\varphi(1)=1\otimes 1$ which is the identity on $B\otimes C$, so $\varphi$ preserves identities. To show $\varphi$ is linear,
    \begin{align*}
        \varphi(b_1+b_2)=b_1+b_2\otimes 1=b_1\otimes 1+b_2\otimes1=\varphi(b_1)+\varphi(b_2)
    \end{align*}
    and
    \[
    \varphi(b_1b_2)=b_1b_2\otimes 1=(b_1\otimes 1)(b_2\otimes 1)=\varphi(b_1)\varphi(b_2)
    \]
    An almost identical argument shows $\phi$ is a ring morphism as well. Lastly, we suppose we have a ring $P$ with morphisms $f:B\to P$ and $g:C\to P$ such that the following diagram commutes:
    \begin{center}
        \begin{tikzcd}
            P& C\arrow[l, "g"]\\
            B\arrow{u}{f}& A \arrow{l}{\beta} \arrow{u}{\gamma}
        \end{tikzcd}
    \end{center}
    To show
    \begin{center}
        \begin{tikzcd}
            P&&\\
            & B\otimes_A C \arrow[dashed]{ul}[description]{\exists!}&C\arrow{l}{\phi}\arrow[bend right]{ull}[swap]{g}\\
            &B \arrow{u}{\varphi}\arrow[bend left]{uul}{f}&A\arrow{l}{\beta} \arrow{u}{\gamma}
        \end{tikzcd}
    \end{center}
    commutes, we see that if any such map $\chi:B\otimes_A C\to P$ exists that satisfies the diagram, $\chi \circ \varphi=f$ and $\chi\circ \phi=g$. This equivalently says $\chi(b\otimes 1)=f(b)$ and $\chi(1\otimes c)=g(c)$. This actually determines the action of $\chi$ entirely because $\chi$ is a ring morphism and thus
    \[
    \chi(b\otimes c)=\chi((b\otimes 1)(1\otimes c))=\chi(b\otimes 1)\chi(1\otimes c)=f(b)g(c)
    \]
    This proves that $\chi$ is unique. To prove existence, we need to show that $\chi$ is a ring morphism. We can use the universal property of tensor products to do so. Define $\alpha:B\times C\to P$ as $\alpha(b,c)=f(b)g(c)$. To show $\alpha$ is $A$ bilinear, we observe
    \begin{align*}
        \alpha(ab,c)=f(ab)g(c)=f(\beta(a)b)g(c)=f\circ \beta(a)f(b)g(c)=g\circ \gamma(a)f(b)g(c)\\
        =f(b)g(\gamma(a)c)=f(b)g(ac)=\alpha(b,ac)=a\alpha(b,c)
    \end{align*}
    and
    \begin{align*}
    \alpha(b_1+b_2,c)=f(b_1+b_2)g(c)=f(b_1)g(c)+f(b_2)g(c)=\alpha(b_1,c)+\alpha(b_2,c)
    \end{align*}
    and
    \[
    \alpha(b,c_1+c_2)=f(b)g(c_1+c_2)=f(b)g(c_1)+f(b)g(c_2)=\alpha(b,c_1)+\alpha(b,c_2)
    \]
    Then by the universal property of tensor products, we get our induced map $\chi$ defined exactly as we require it to be. This proves existence of $\chi$ and completes the proof.
\end{proof}
\subsubsection{V}\label{1.3.V}
\begin{proof}
    Suppose $\pi_1:X\to Y$ and $\pi_2:X\to Z$ are both monomorphisms and suppose we have two morphisms $f,g:W\to X$. We want to show that $\pi_2\circ \pi_1\circ f=\pi_2\circ \pi_1\circ g\Rightarrow f=g$. Supposing $\pi_2\circ \pi_1\circ f=\pi_2\circ \pi_1\circ g$, by $\pi_2$ being monic we have $\pi_1\circ f=\pi_1\circ g$. Now we use the fact that $\pi_1$ is monic to get $f=g$ as desired.
\end{proof}
\subsubsection{W}\label{1.3.W}
\begin{proof}
    \begin{enumerate}
        \item[($\Rightarrow)$]
        We suppose $\pi:X\to Y$ is monic. To prove $X\times_YX$ exists, we claim that $X$ satisfies the definition of $X\times_Y X$. To show this, we want to show
        \begin{center}
        \begin{tikzcd}
            P\arrow[dashed]{dr}[description]{\exists!} \arrow[bend right]{ddr}[swap]{p_1} \arrow[bend left]{drr}{p_2}&&\\
            &X \arrow{r}{\id_X} \arrow{d}{\id_X}& X \arrow{d}{\pi}\\
            &X\arrow{r}{\pi}& Y
        \end{tikzcd}
    \end{center}
    holds. Using the fact that $\pi$ is monic and the fact that $\pi\circ p_1=\pi\circ p_2$ by commutativity to get that $p_1=p_2$, so we just need to show that
    \begin{center}
        \begin{tikzcd}
            P\arrow[dashed]{dr}[description]{\exists!} \arrow[bend right]{ddr}[swap]{p} \arrow[bend left]{drr}{p}&&\\
            &X \arrow{r}{\id_X} \arrow{d}{\id_X}& X \arrow{d}{\pi}\\
            &X\arrow{r}{\pi}& Y
        \end{tikzcd}
    \end{center}
    commutes. The unique morphism is clearly $p$. Thus $X$ satisfies the definition of $X\times_Y X$, and thus the induced morphism is $\id_X$ which is, in particular, an isomorphism.
    \item[$(\Leftarrow)$]
    Now supposing that there is a unique isomorphism $\varphi:X\to X\times_YX$ and that $X\times_Y X$ exists, we will furthermore suppose that $\pi\circ f=\pi\circ g$ for some $f,g:Z\to X$. Notice that 
    \begin{center}
        \begin{tikzcd}
            X\arrow[dashed]{dr}[description]{\exists! \varphi} \arrow[bend right]{ddr}[swap]{\id_X} \arrow[bend left]{drr}{\id_X}&&\\
            &X\times_YX \arrow{r}{\chi_1} \arrow{d}{\chi_2}& X \arrow{d}{\pi}\\
            &X\arrow{r}{\pi}& Y
        \end{tikzcd}
    \end{center}
    commuting and $\varphi$ being an isomorphism implies that $\chi_2=\varphi^{-1}=\chi_1$.
    Then we obtain a map $\phi$ from the below commutative diagram:
    \begin{center}
        \begin{tikzcd}
            Z\arrow[dashed]{dr}[description]{\exists!} \arrow[bend right]{ddr}[swap]{f} \arrow[bend left]{drr}{g}&&\\
            &X\times_YX \arrow{r}{\varphi^{-1}} \arrow{d}{\varphi^{-1}}& X \arrow{d}{\pi}\\
            &X\arrow{r}{\pi}& Y
        \end{tikzcd}
    \end{center}
    Therefore $f=\varphi^{-1}\circ \phi$ and $g=\varphi^{-1}\circ \phi$. This implies that $f=g$ as desired so $\pi$ is monic.
    \end{enumerate}
\end{proof}
\subsubsection{X}\label{1.3.X}
\begin{proof}
    We will use the same variables in this exercise as in Exercise \ref{1.3.S}S. Let $\varphi:X_1\times_YX_2\to X_1\times_ZX_2$ be induced in the following diagram, using the fact $\pi$ is monic here so that $\pi\circ f \circ \tau_1=\pi\circ g \circ \tau_2\Rightarrow f\circ \tau_1=g\circ \tau_2$:
    \begin{center}
        \begin{tikzcd}
            X_1\times_Y X_2\arrow[dashed]{dr}[description]{\exists!} \arrow[bend right]{ddr}[swap]{\tau_1} \arrow[bend left]{drr}{\tau_2}&&&\\
            &X_1\times_ZX_2 \arrow{r}{\pi_2} \arrow{d}{\pi_1}& X_2 \arrow{d}{g}&\\
            &X_1\arrow{r}{f}& Y\arrow{dr}{\pi}&\\
            &&&Z
        \end{tikzcd}
    \end{center}
    We also use the fact that $\pi$ is monic implies both $\mu_1,\mu_2$ from Exercise \ref{1.3.S}S are equal to $\alpha^{-1}$, where $\alpha$ is an isomorphism by Exercise \ref{1.3.W}W. Then
    \[
    f\circ \pi_1=\mu_1\circ \theta=\alpha^{-1}\circ \theta\Rightarrow \alpha\circ f\circ \pi_1=\theta
    \]
    Now using the magic diagram from Exercise \ref{1.3.S}S, we have
    \begin{center}
        \begin{tikzcd}
            X_1\times_Z X_2 \arrow[dashed]{dr}[description]{\exists!} \arrow[bend right]{ddr}[swap]{f\circ \pi_1} \arrow[bend left]{drr}{\id_{X_1\times_Z X_2}}&&\\
            &X_1\times_YX_2 \arrow{r}{\varphi} \arrow{d}{f\circ \tau_1}& X_1\times_Z X_2 \arrow{d}{\alpha\circ f\circ \pi_1}\\
            &Y\arrow{r}{\alpha}& Y\times_ZY 
        \end{tikzcd}
    \end{center}
    Let $\phi$ be the map induced in the above diagram, where it is immediate that $\varphi\circ \phi=\id_{X_1\times_Z X_2}$. To show $\phi\circ \varphi=\id_{X_1\times_Y X_2}$, we will show it satisfies the diagram below, which suffices because clearly $\id_{X_1\times_Y X_2}$ also does:
    \begin{center}
        \begin{tikzcd}
            X_1\times_Y X_2 \arrow[dashed]{dr}[description]{\exists!} \arrow[bend right]{ddr}[swap]{\tau_1} \arrow[bend left]{drr}{\tau_2}&&\\
            &X_1\times_YX_2 \arrow{r}{\tau_2} \arrow{d}{\tau_1}& X_2\arrow{d}{g}\\
            &X_1\arrow{r}{f}& Y 
        \end{tikzcd}
    \end{center}
    We need to show that $\tau_1=\tau_1\circ \phi\circ \varphi$ and that $\tau_2=\tau_2\circ\phi\circ \varphi$. Recalling that $\phi$ is a section of $\varphi$ and that $\tau_1=\pi_1\circ \varphi$ and $\tau_2=\pi_2\circ \varphi$, we have
    \[
    \tau_1\circ \phi\circ \varphi=\pi_1\circ \varphi \circ \phi \circ \varphi =\pi_1\circ \varphi=\tau_1
    \]
    as well as
    \[
    \tau_2 \circ \phi \circ \varphi=\pi_2\circ \varphi \circ \phi \circ \varphi=\pi_2\circ \varphi=\tau_2
    \]
    which proves that $\phi=\varphi^{-1}$ and $\varphi$ is an isomorphism.
\end{proof}
\subsubsection{Y}\label{1.3.Y}
\begin{proof}
    \begin{enumerate}[(a)]
        \item We have the following diagram is commutative for all $f:C\to B$:
        \begin{center}
            \begin{tikzcd}
                \Mor(B,A)\arrow{r}{f^*} \arrow{d}{\iota_B}& \Mor(C,A)\arrow{d}{\iota_C}\\
                \Mor(B,A')\arrow{r}{f^*}&\Mor(C,A')
            \end{tikzcd}
        \end{center}
        Now because $B$ was arbitrary, we let $B=A$, and then we have
        \begin{center}
            \begin{tikzcd}
                \Mor(A,A)\arrow{r}{f^*} \arrow{d}{\iota_A}& \Mor(C,A)\arrow{d}{\iota_C}\\
                \Mor(A,A')\arrow{r}{f^*}&\Mor(C,A')
            \end{tikzcd}
        \end{center}
        Now we track $\id_A$ through the bottom portion of the diagram, letting $g=\iota_A(\id_A)$ and $f\in \Mor(C,A)$ arbitrary to get
        \[
        f^*\circ \iota_A(\id_A)=f^*(g)=g\circ f
        \]
        On the top side of the diagram, we get
    \[
    \iota_C\circ f^*(\id_A)=\iota_C(\id_A\circ f)=\iota_C(f)
    \]
    By commutativity, these two are equal, hence $\iota_C(f)=g\circ f$, determining $\iota_C$ entirely.
    \item 
    Now assuming all of the $\iota_C$ are isomorphisms, we get the following diagram where $g\in \Mor(A,A')$ is as defined in the previous part:
    \begin{center}
            \begin{tikzcd}
                \Mor(A',A)\arrow{r}{g^*} \arrow{d}{\iota_{A'}}& \Mor(A,A)\arrow{d}{\iota_A}\\
                \Mor(A',A')\arrow{r}{g^*}&\Mor(A,A')
            \end{tikzcd}
        \end{center}
        By surjectivity of $\iota_A$, for each $f\in \Mor(A,A')$, there exists a unique $f'\in \Mor(A,A)$ such that $\iota_A(f')=f\iff g\circ f'=f$. On the other side of the diagram, for every $\alpha\in \Mor(A',A')$, there exists a unique $\alpha'\in \Mor(A',A)$ such that $\iota_{A'}(\alpha')=\alpha \iff g\circ \alpha'=\alpha$.\\
        \newline
        Thus if $\alpha=\id_{A'}$, we obtain a section $\alpha'$ of $g$. Now by uniqueness of the first statement, there exists a unique $f'\in \Mor(A,A)$ such that $g\circ f'=g$. But $\id_A$ and $\alpha' \circ g$ both satisfy this requirement, which proves that $\alpha'\circ g=\id_A$, proving that $\alpha'=g^{-1}$.
    \end{enumerate}
    
\end{proof}
\subsubsection{Z}\label{1.3.Z}
\begin{proof}
    \begin{enumerate}[(a)]
        \item If we're given some $f\in \Mor(B,A)$, we want to give a natural transformation $m_C:\Mor(A,C)\to \Mor(B,C)$. We define for every $C\in \fC$, we define
        \[
        m_C(\phi)=\phi\circ f
        \]
        To prove $m$ is indeed a natural transformation, we need to show for every $g:C\to C'$, the following diagram commutes:
        \begin{center}
            \begin{tikzcd}
                \Mor(A,C)\arrow{r}{g_*} \arrow{d}{m_C}& \Mor(A,C')\arrow{d}{m_{C'}}\\
                \Mor(B,C)\arrow{r}{g_*}&\Mor(B,C')
            \end{tikzcd}
        \end{center}
        On the bottom side of the diagram, for any $\phi\in \Mor(A,C)$, we observe
        \[
        g_*\circ m_C(\phi)=g_*(\phi\circ f)=g\circ \phi\circ f
        \]
        on the other hand,
        \[
        m_{C'}\circ g_*(\phi)=m_{C'}(g\circ \phi)=g\circ \phi \circ f
        \]
        so $m$ is a natural transformation.\\
        \newline
        Now if we're given a natural transformation $m$, we get the following commutative diagram for arbitrary $C\in \fC$ and $g\in \Mor(A,C)$:
        \begin{center}
            \begin{tikzcd}
                \Mor(A,A)\arrow{r}{g_*} \arrow{d}{m_A}& \Mor(A,C)\arrow{d}{m_{C}}\\
                \Mor(B,A)\arrow{r}{g_*}&\Mor(B,C)
            \end{tikzcd}
        \end{center}
        Tracking $\id_A$ on the bottom and defining $f\coloneqq m_A(\id_A)\in \Mor(B,A)$, we get
        \[
        g_*\circ m_A(\id_A)=g_*(f)=g\circ f
        \]
        On the top, we get
        \[
        m_C\circ g_*(\id_A)=m_C(g\circ \id_A)=m_C(g)
        \]
        By commutativity, $m_C(g)=g\circ f$ for all $g\in \Mor(A,C)$. Because $f$ uniquely defines $m$, we have obtained a unique morphism from every natural transformation.\\
        \newline
        We define a map $\varphi:\Mor(B,A)\to \Nat(h^A,h^B)$ given as $\varphi(f)=\circ f$, and another map $\phi:\Nat(h^A,h^B)\to \Mor(B,A)$ as $\phi(m)=m_A(\id_A)$. To show these are inverse maps, 
        \[
        \phi \circ \varphi(f)=\phi(\circ f)=\id_A\circ f=f
        \]
        and
        \[
        \varphi \circ \phi(m)=\varphi(m_A(\id_A))=\circ m_A(\id_A)=m
        \]
        by our previous work. Thus $\phi=\varphi^{-1}$ and we have given the desired bijection.
        \item 
        Given any $f\in \Mor(A,B)$, define $\varphi(f)=f\circ$ where $\varphi:\Mor(A,B)\to \Nat(h_A,h_B)$. Similarly to part (a), one can readily check that this defines a natural transformation. We can also do a similar process of tracking the identity to realize that any for natural transformation $m$ and any $g\in \Mor(C,A)$, $m_C(g)=m_A(\id_A)\circ g$. We then define $\phi:\Nat(h_A,h_B)\to \Mor(A,B)$ given by $\phi(m)=m_A(\id_A)$. In a very similar manner to part (a), $\phi=\varphi^{-1}$ so we obtain the bijection we want.
        \item 
        If we're given any natural transformation $m$ from $h^A\to F$, we have that for all $f\in \Mor(B,C)$, the following diagram commutes:
        \begin{center}
            \begin{tikzcd}
                \Mor(A,B)\arrow{r}{f_*} \arrow{d}{m_B}& \Mor(A,C)\arrow{d}{m_{C}}\\
                F(B)\arrow{r}{Ff}&F(C)
            \end{tikzcd}
        \end{center}
        Letting $B=A$, we have
        \begin{center}
            \begin{tikzcd}
                \Mor(A,A)\arrow{r}{f_*} \arrow{d}{m_A}& \Mor(A,C)\arrow{d}{m_{C}}\\
                F(A)\arrow{r}{Ff}&F(C)
            \end{tikzcd}
        \end{center}
        Yet again, we track $\id_A$ on the bottom to get $Ff\circ m_A(\id_A)$, and on the top we get $m_C\circ f_*(\id_A)=m_C(f\circ \id_A)=m_C(f)$. Thus by commutativity, for any $f\in \Mor(A,C)$, 
        \[
        m_C(f)=Ff(m_A(\id_A))
        \]
        We now notice then that $m_C$ is determined entirely by $m_A(\id_A)$, so we define $\varphi:\Nat(h^A,F)\to F(A)$ given by $\varphi(m)=m_A(\id_A)$. On the other hand we define $\phi:F(A)\to \Nat(h^A,F)$ to act as $\phi(\chi)_C(f)=Ff(\chi)$ for any $C\in \fC$ and $f\in \Mor(A,C)$.\\
        \newline
        Then
        \[
        \varphi\circ \phi(\chi)=\varphi( \phi(\chi))=\phi(\chi)_A(\id_A)=F(\id_A)(\chi)=\id_{F(A)}(\chi)=\chi
        \]
        and for any $f\in \Mor(A,C)$,
        \[
        \phi\circ \varphi(m)_C(f)=\phi(m_A(\id_A))_C(f)=Ff(m_A(\id_A))=m_C(f)
        \]
        so indeed $\varphi$ is a bijection.
    \end{enumerate}
\end{proof}

\subsection{}
\subsubsection{A}\label{1.4.A}
\begin{proof}
    We claim that if $F:\mathscr{I}\to \fC$ is a functor and $e\in \fI$ is an initial object, then $\varprojlim A_i=A_e$. Because $e$ is initial, there exists a unique morphism into every $i\in \fI$, so there exists a unique morphism $f_i:A_e\to A_i$ for each $i$. If $W$ is another object in $\fC$ with maps $p_i$ for each $i$ that commutes with everything, there exists a morphism $p_e:W\to A_e$ because $e\in \fI$. We also know that by assumption the following diagram must commute:
    \begin{center}
        \begin{tikzcd}
            &W\arrow{dl}[swap]{p_e}\arrow{dr}{p_i}&\\
            A_e \arrow{rr}{\exists! f_i}&&A_i
        \end{tikzcd}
    \end{center}
    so in particular the following diagram commutes for all $f:i\to j$ and all $i,j\in \fI$.
    \begin{center}
        \begin{tikzcd}
            &W\arrow{d}{p_e} \arrow[bend right]{ddl}[swap]{p_i} \arrow[bend left]{ddr}{p_j}&\\
            &A_e \arrow{dl}{f_i} \arrow{dr}[swap]{f_j}&\\
            A_i \arrow{rr}{Ff}&&A_j
        \end{tikzcd}
    \end{center}
    Uniqueness comes from the fact that any morphism $g:W\to A_e$ that makes the diagram commute in particular makes the following subdiagram commute:
    \begin{center}
        \begin{tikzcd}
            W\arrow{d}{g} \arrow[bend right=40]{dd}[swap]{p_e}\\
            A_e \arrow{d}{\id_{A_e}}\\
            A_e
        \end{tikzcd}
    \end{center}
    so that $g=p_e$.
\end{proof}

\subsubsection{B}\label{1.4.B}
\begin{proof}
    To show $X_1\times_Y X_2$ is the limit of the diagram, we need to show
    \begin{center}
        \begin{tikzcd}
            &X_1\arrow{dr}{f}&&\\
            X_1\times_Y X_2\arrow{ur}{\tau_1} \arrow{dr}{\tau_2}&&Y\arrow{r}{h}&Z\\
            &X_2\arrow{ur}{g}&&
        \end{tikzcd}
    \end{center}
    commutes given the Cartesian square below:
    \begin{center}
        \begin{tikzcd}
            X_1\times_YX_2\arrow{r}{\tau_2} \arrow{d}{\tau_1}& X_2\arrow{d}{g}\\
            X_1\arrow{r}{f}&Y
        \end{tikzcd}
        \end{center}
        But the commutativity of the first diagram is trivial then. Now to show the first diagram is universal, suppose we have the following commutative diagram:
        \begin{center}
        \begin{tikzcd}
            &X_1\arrow{dr}{f}&&\\
            P\arrow{ur}{p_1} \arrow{dr}{p_2}&&Y\arrow{r}{h}&Z\\
            &X_2\arrow{ur}{g}&&
        \end{tikzcd}
    \end{center}
Then we get an induced map from the following diagram:
\begin{center}
    \begin{tikzcd}
        P\arrow[bend right]{ddr}{p_1} \arrow[bend left]{drr}{p_2}\arrow[dashed]{dr}[description]{\exists!}&&\\
        & X_1\times_Y X_2 \arrow{d}{\tau_1}\arrow{r}{\tau_2}& X_2\arrow{d}{g}\\
        &X_1 \arrow{r}{f}&Y
    \end{tikzcd}
\end{center}
    This proves uniqueness. This map $\gamma$ makes the following diagram commute trivially
    \begin{center}
        \begin{tikzcd}
            &X_1\arrow{dr}{f}&&\\
            P\arrow{ur}{p_1} \arrow{dr}{p_2}\arrow{r}{\gamma}&X_1\times_Y X_2\arrow{u}{\tau_1}\arrow{d}{\tau_2}&Y\arrow{r}{h}&Z\\
            &X_2\arrow{ur}{g}&&
        \end{tikzcd}
    \end{center}
    which proves existence. Thus $X_1\times_Y X_2$ is the limit of the diagram.\\
    \newline
    To show $Y\times_{(Y\times_ZY)}X_1\times_Z X_2$ is also the limit of the diagram, we first need to show that the following diagram commutes:
    \begin{center}
        \begin{tikzcd}
            &X_1\arrow{dr}{f}&&\\
            Y\times_{(Y\times_ZY)}X_1\times_Z X_2\arrow{ur}{\pi_1\circ \iota_2} \arrow{dr}{\pi_2\circ \iota_2}&&Y\arrow{r}{h}&Z\\
            &X_2\arrow{ur}{g}&&
        \end{tikzcd}
    \end{center}
    where we're given the following Cartesian diagrams:
    \begin{center}
        \begin{tikzcd}
            Y\times_{(Y\times_ZY)}X_1\times_Z X_2\arrow{r}{\iota_2} \arrow{d}{\iota_1}& X_1\times_Z X_2 \arrow{d}{\theta}\\
            Y\arrow{r}{\alpha}& Y\times_Z Y
        \end{tikzcd}
    \end{center}
    and
    \begin{center}
        \begin{tikzcd}
            X_1\times_Z X_2\arrow{r}{\pi_2} \arrow{d}{\pi_1}&X_2 \arrow{d}{h\circ g}\\
            X_1\arrow{r}{h\circ f}&Z
        \end{tikzcd}
    \end{center}
    and the induced map $\alpha$ below:
\begin{center}
    \begin{tikzcd}
        Y\arrow[bend right]{ddr}{\id_Y} \arrow[bend left]{drr}{\id_Y}\arrow[dashed]{dr}[description]{\exists!\alpha }&&\\
        & Y\times_Z Y \arrow{d}{\mu_1}\arrow{r}{\mu_2}& Y\arrow{d}{h}\\
        &Y \arrow{r}{h}&Z
    \end{tikzcd}
\end{center}
    as well as the induced map $\theta$ from the following:
    \begin{center}
    \begin{tikzcd}
        X_1\times_Z X_2\arrow[bend right]{ddr}{f\circ \pi_1} \arrow[bend left]{drr}{g\circ \pi_2}\arrow[dashed]{dr}[description]{\exists!}&&\\
        & Y\times_Z Y \arrow{d}{\mu_1}\arrow{r}{\mu_2}& Y\arrow{d}{h}\\
        &Y \arrow{r}{h}&Z
    \end{tikzcd}
\end{center}
Now to show
    \begin{center}
        \begin{tikzcd}
            &X_1\arrow{dr}{f}&&\\
            Y\times_{(Y\times_ZY)}X_1\times_Z X_2\arrow{ur}{\pi_1\circ \iota_2} \arrow{dr}{\pi_2\circ \iota_2}&&Y\arrow{r}{h}&Z\\
            &X_2\arrow{ur}{g}&&
        \end{tikzcd}
    \end{center}
    does indeed commute, we observe
    \begin{align*}
        \iota_1=\iota_1\\
        \Rightarrow \mu_1\circ \alpha\circ \iota_1=\mu_2\circ \alpha\circ \iota_1\\
        \Rightarrow \mu_1\circ \theta\circ \iota_2=\mu_2\circ \theta\circ \iota_2\\
        \Rightarrow f\circ \pi_1\circ \iota_2=g\circ \pi_2\circ \iota_2
    \end{align*}
    just by recalling the definitions of each. Now that this diagram commutes, we suppose we have the following commutative diagram to prove universality:
    \begin{center}
        \begin{tikzcd}
            &X_1\arrow{dr}{f}&&\\
            P\arrow{ur}{p_1} \arrow{dr}{p_2}&&Y\arrow{r}{h}&Z\\
            &X_2\arrow{ur}{g}&&
        \end{tikzcd}
    \end{center}
    Then we can use the universal property of $X_1\times_Z X_2$ to get a unique map $\beta$ in the following diagram:
    \begin{center}
    \begin{tikzcd}
        P\arrow[bend right]{ddr}{p_1} \arrow[bend left]{drr}{p_2}\arrow[dashed]{dr}[description]{\exists!}&&\\
        & X_1\times_Z X_2 \arrow{d}{\pi_1}\arrow{r}{\pi_2}& X_2\arrow{d}{h\circ g}\\
        &X_1 \arrow{r}{h\circ f}&Z
    \end{tikzcd}
\end{center}
With this map $\beta$, we claim the following diagram commutes:
\begin{center}
    \begin{tikzcd}
        P\arrow{r}{\beta}\arrow{d}{f\circ p_1}& X_1\times_Z X_2 \arrow{d}{\theta}\\
        Y\arrow{r}{\alpha}&Y\times_Z Y
    \end{tikzcd}
\end{center}
To prove this, we turn to the universal property of $Y\times_ZY$ shown below:
\begin{center}
    \begin{tikzcd}
        P\arrow[bend right]{ddr}{f\circ p_1} \arrow[bend left]{drr}{g\circ p_2}\arrow[dashed]{dr}[description]{\exists!}&&\\
        & Y\times_ZY \arrow{d}{\mu_1}\arrow{r}{\mu_2}& Y\arrow{d}{h}\\
        &Y \arrow{r}{h}&Z
    \end{tikzcd}
\end{center}
We will show that both $\alpha \circ f\circ p$ and $\theta \circ \beta$ satisfy the unique arrow.\\
\newline
To show $\alpha \circ f\circ p_1$ satisfies the diagram, we see
\[
\mu_1\circ \alpha\circ f \circ p_1=f\circ p_1
\]
and
\[
\mu_2\circ \alpha \circ f \circ p_1=f\circ p_1=g\circ p_2
\]
Now to show $\theta\circ \beta$ satisfies the diagram,
\[
\mu_1\circ \theta\circ \beta=f\circ \pi_1\circ \beta=f\circ p_1
\]
as well as
\[
\mu_2\circ \theta\circ \beta=g\circ \pi_2\circ \beta=g\circ p_2
\]
This proves that by uniqueness of the arrow, that $\alpha\circ f\circ p_1=\theta\circ \beta$.
Thus we get an induced map $\chi$ in the following commutative diagram:
\begin{center}
    \begin{tikzcd}
        P\arrow[bend left]{drr}{\beta}\arrow[bend right]{ddr}[swap]{f\circ p_1} \arrow[dashed]{dr}[description]{\exists!}&&\\
        & Y\times_{(Y\times_Z Y)} X_1\times_Z X_2\arrow{r}{\iota_2}\arrow{d}{\iota_1}&X_1\times_Z X_2\arrow{d}{\theta}\\
        &Y\arrow{r}{\alpha}&Y\times_ZY
    \end{tikzcd}
\end{center}
Therefore the following diagram commutes as well:
\begin{center}
        \begin{tikzcd}
            &X_1\arrow{dr}{f}&&\\
            P\arrow{ur}{p_1} \arrow{dr}[swap]{p_2}\arrow{r}{\chi}&Y\times_{(Y\times_Z Y)} X_1\times_Z X_2\arrow{u}{\pi_2\circ \iota_2}\arrow{d}[swap]{\pi_1\circ \iota_2}&Y\arrow{r}{h}&Z\\
            &X_2\arrow{ur}{g}&&
        \end{tikzcd}
    \end{center}
    because
    \[
    \pi_2\circ \iota_2\circ \chi=\pi_2\circ \beta=p_1
    \]
    and
    \[
    \pi_1\circ \iota_2\circ \chi=\pi_1\circ \beta=p_2
    \]
    This proves that the limit of the diagram
        \begin{center}
        \begin{tikzcd}
            X_1\arrow{dr}{f}&&\\
            &Y\arrow{r}{h}&Z\\
            X_2\arrow{ur}{g}&&
        \end{tikzcd}
    \end{center}
    is simultaneously $X_1\times_Y X_2$ and $Y\times_{(Y\times_Z Y)} X_1\times_Z X_2$, meaning they are defined up to unique isomorphism, so in particular the following diagram is Cartesian:
    \begin{center}
        \begin{tikzcd}
            X_1\times_Y X_2 \ar{r} \ar{d}& X_1\times_Z X_2 \ar{d}\\
            Y\ar{r}& Y\times_Z Y
        \end{tikzcd}
    \end{center}
\end{proof}

\subsubsection{C}\label{1.4.C}
\begin{proof}
    Let $S$ be the defined set and $\pi_i:S\to A_i$ are the projections. It's clear that for any $i,j\in \fI$ and $m:i\to j$, the following diagram commutes:
    \begin{center}
        \begin{tikzcd}
            &S\ar{dl}[swap]{\pi_i}\ar{dr}{\pi_j}&\\
            A_i\ar{rr}{Fm}&&A_j
        \end{tikzcd}
    \end{center}
    by construction of $S$. Now suppose
    \begin{center}
        \begin{tikzcd}
            &W\ar{dl}[swap]{g_i}\ar{dr}{g_j}&\\
            A_i\ar{rr}{Fm}&&A_j
        \end{tikzcd}
    \end{center}
    commutes under the same hypotheses. If there were a map $\varphi: W\to S$ such that
    \begin{center}
        \begin{tikzcd}
            &W\ar[dashed]{d}{\varphi} \ar[bend right]{ddl}[swap]{g_i} \ar[bend left]{ddr}{g_j}&\\
            &S\ar{dl}[swap]{\pi_i}\ar{dr}{\pi_j}&\\
            A_i\ar{rr}{Fm}&&A_j
        \end{tikzcd}
    \end{center}
    commutes, then for each $w\in W$, $\varphi(w)=s$ where $\pi_i(s)=g_i(w)$. This element $s$ is uniquely defined to be $(g_i(w))_{i\in \fI}$. This demonstrates that $\varphi$ exists and is unique, so indeed $S=\varprojlim_{\fI} A_i$.
\end{proof}
\subsubsection{D}\label{1.4.D}
\begin{proof}
    \begin{enumerate}[(a)]
        \item I'm not entirely sure if the question wants to describe $\Q$ as an object of $\Ring$ or $\Mod_\Z$, but I will assume we want $\Q\in \Ring$. We take the index set to be the set of positive integers with a unique arrow $n\to m$ if and only if there exists some positive integer $k$ such that $m=nk$. If this is the case, we define a ring morphism $\phi_{n,k}:\Z_n \to \Z_{nk}$ defined by $\frac{x}{n^i}\mapsto \frac{k^ix}{(nk)^i}$. Here, is the ring given by localization by the multiplicative subset generated by $n$. We define maps $\iota_n:\Z_n\to \Q$ as $\frac{x}{n^i}\mapsto \frac{x}{n^i}$. Then by the construction, the following diagram commutes:
        \begin{center}
            \begin{tikzcd}
                \Q&\\
                \Z_{nk} \arrow{u}{\iota_{nk}}& \Z_n \ar{ul}[swap]{\iota_n} \ar{l}{\phi_{n,k}}
            \end{tikzcd}
        \end{center}
        because
        \[
        \iota_{nk}\circ \phi_{n,k}(\frac{x}{n^i})=\iota_{nk}(\frac{xk^i}{(nk)^i})=\frac{xk^i}{(nk)^i}=\frac{x}{n^i}=\iota_n(\frac{x}{n^i})
        \]
        If we have another ring $R$ with maps $f_n:\Z_n\to W$ satisfying the commutativity hypotheses, we want to show
        \begin{center}
            \begin{tikzcd}
                R\\
                \Q\ar[dashed]{u}{\exists!}&\\
                \Z_{nk}\ar[bend left=70]{uu}{f_{nk}} \arrow{u}{\iota_{nk}}& \Z_n \ar[bend right]{luu}[swap]{f_n} \ar{ul}[swap]{\iota_n} \ar{l}{\phi_{n,k}}
            \end{tikzcd}
        \end{center}
        By commutativity alone, we would require the unique map $\varphi$ to act as $\frac{x}{n^i}\mapsto f_n(\frac{x}{n^i})$ which shows that the map $\varphi$ is unique. To be precise we should show that $\varphi$ is indeed a ring morphism by showing that it's well defined for different choices of $\frac{x}{n}$, i.e. if $\frac{x}{y}=\frac{p}{q}$ then their images are the same. Notice that $\frac{x}{y}=\frac{p}{q}$ if and only if $\frac{xq}{yq}=\frac{yp}{yq}$. Therefore
        \[
        \varphi(\frac{x}{y})=\varphi\circ \iota_y(\frac{x}{y})=\varphi\circ \iota_{yq}(\frac{xq}{yq})=\varphi\circ \iota_{q}(\frac{p}{q})=\varphi(\frac{p}{q})
        \]
        and the other ring morphism axioms can be easily verified.
        \item 
        For any set $X$, we have the category $\Ssubset(X)$ in which we can define $A_1\cup A_2=\varinjlim_{\fI}A_i$ where $\fI$ is the discrete category
        \begin{center}
            \begin{tikzcd}
                1&2
            \end{tikzcd}
        \end{center}
        Explicitly, we are defining $A_1\cup A_2$ as the coproduct of $A_1$ and $A_2$. If
        \begin{center}
            \begin{tikzcd}
                &B&\\
                A_1\ar[ur]&&A_2\ar[ul]
            \end{tikzcd}
        \end{center}
        commutes, then $A_1\subset B$ and $A_2\subset B$ which directly implies that the standard definition of $A_1\cup A_2\subset B$. Therefore there is a morphism $A_1\cup A_2\to B$, and uniqueness is by uniqueness of arrows in $\Ssubset(X)$.
    \end{enumerate}
\end{proof}
\subsubsection{E}\label{1.4.E}
\begin{proof}
    Let $S$ be the defined set and let $\iota_i(a)=[(a,i)]$ where $\iota_i:A_i\to S$. If $m:i\to j$, then
    \[
    \iota_i(a)=[(a,i)]
    \]
    and
    \[
    \iota_j\circ Fm(a)=[(Fm(a),j)]
    \]
    Also notice that $(a,i)\sim (Fm(a),j)$ because $Fm:A_i\to A_j$ and $\id_{A_j}=F\id_j$ are two maps such that $Fm(a)=\id_{A_j}(f(a))$. This shows that $S$ satisfies the required definition.\\
    \newline
    To show that $S$ is universal, suppose we have another set $W$ equipped with maps $g_i:A_i\to W$ that satisfy the definition. We want to show
    \begin{center}
        \begin{tikzcd}
            &W&\\
            &S\ar[dashed]{u}[description]{\exists!}\\
            A_i \ar[bend left]{uur}{g_i} \ar{rr}{Fm} \ar{ur}{\iota_i}&& A_j \ar{ul}[swap]{\iota_j} \ar[bend right]{uul}[swap]{g_j}
        \end{tikzcd}
    \end{center}
    We define $\varphi:S\to W$ as $\varphi([(a,i)])=g_i(a)$, which proves uniqueness because this condition comes directly from commutativity. To prove existence, we just need to show $\varphi$ is well defined. In other words, we need to show that if $(a_i,i)\sim (a_j,j)$ then $g_i(a_i,i)=g_j(a_j,j)$. If $(a_i,i)\sim (a_j,j)$, then for some $\alpha:i\to k$ and some $\beta:j\to k$,
    \[
    F\alpha(a_i)=F\beta(a_j)
    \]
    Then the following diagram must commute:
    \begin{center}
        \begin{tikzcd}
            &W&\\
            A_i \ar{ur}{g_i} \ar{r}[swap]{F\alpha}& A_k \ar{u}{g_k}& A_j \ar{l}{F\beta}\ar{ul}[swap]{g_j}
        \end{tikzcd}
    \end{center}
    We now observe that
    \[
    g_i(a_i)=g_k\circ F\alpha(a_i)=g_k\circ F\beta(a_j)=g_j(a_j)
    \]
    so $\varphi$ is well defined, which proves existence.
\end{proof}
\subsubsection{F}\label{1.4.F}
\begin{proof}
    For the problem, let $m_i$ denote $(m_i,i)\in \coprod_\fI M_i$ as well as the element in $M_i$ depending on the context for convenience. To prove addition is well defined, suppose $m_i\sim m_{i'}$ and $m_j\sim m_{j'}$ for some $i,i',j,j'\in \fI$. Also pick some $l$ and $l'$ such that we have 
    \begin{center}
        \begin{tikzcd}
            i\ar{r}{u}&l&j\ar{l}[swap]{v}
        \end{tikzcd}
    \end{center}
    and
    \begin{center}
        \begin{tikzcd}
            i'\ar{r}{u'}&l'&j'\ar{l}[swap]{v'}
        \end{tikzcd}
    \end{center}
    Then there exists some $f:i\to n_i$ and some $f':{i'}\to n_i$ such that $Ff(m_i)=Ff'(m_{i'})$ and some $g:j\to n_j$ and $g':j'\to n_j$ such that $Fg(m_j)=Fg'(m_{j'})$. By the first filtered hypothesis, we have the following set of arrows in $\fI$:
    \begin{center}
        \begin{tikzcd}
        i\ar{d}&&j\ar[d]\\
        n_i\ar[r]&n&n_j\ar[l]\\
        i'\ar[u]&& j'\ar[u]
    \end{tikzcd}
    \end{center}
    as well as the other set of arrows
    \begin{center}
        \begin{tikzcd}
        i\ar{r}&l\ar[d]&j\ar[l]\\
        &k&\\
        i'\ar[r]&l'\ar[u]& j'\ar[l]
    \end{tikzcd}
    \end{center}
    We can get another set of arrows
    \begin{center}
        \begin{tikzcd}
            n\ar[r]& m& k\ar[l]
        \end{tikzcd}
    \end{center}
    Therefore we have the paths:
    \begin{center}
        \begin{tikzcd}
            1.&i\ar[r]&n_i\ar[r]&n\ar[r]&m\\
            2.&i\ar[r]&l\ar[r]&k \ar[r]&m\\
            3.&j\ar[r]&n_j\ar[r]&n\ar[r]&m\\
            4.&j\ar[r]&l\ar[r]&k\ar[r]&m\\
            5.&i'\ar[r]&n_i\ar[r]&n\ar[r]&m\\
            6.& i'\ar[r]&l'\ar[r]&k\ar[r]&m\\
            7.& j'\ar[r]&n_j\ar[r]&n\ar[r]&m\\
            8.& j'\ar[r]&l'\ar[r]&k\ar[r]&m
        \end{tikzcd}
    \end{center}
    Then by the second requirement of $\fI$ being filtered, there exists 
    \\$m_1,m_2,m_3,m_4\in \fI$ and arrows such that the following diagram commutes:
    \begin{center}
        \begin{tikzcd}
            &l'\ar[r]&k\ar[r]&m\ar{dr}&\\
            i'\ar{ur}\ar[dr]&&&&m_3\\
            &n_i\ar[r]&n\ar[r]&m\ar{dr}\ar[ur]&\\
            i\ar{ur}\ar[dr]&&&&m_1\\
            &l\ar[r]&k\ar[r]&m\ar[ur]\ar[dr]&\\
            j\ar[ur]\ar[dr]&&&&m_2\\
            &n_j\ar[r]&n\ar[r]&m\ar[ur]\ar[dr]&\\
            j'\ar[ur]\ar[dr]&&&&m_4\\
            &l'\ar[r]&k\ar[r]&m\ar[ur]&
        \end{tikzcd}
    \end{center}
    We will add on to this diagram by obtaining the following commutative diagrams:
    \begin{center}
        \begin{tikzcd}
            &m_3\ar[r]&m_{13}'\ar[dr]&\\
            m\ar[ur]\ar[dr]&&&m_{13}\\
            &m_1\ar[r]&m_{13}'\ar[ur]&\\
            &&&\\
            &m_2\ar[r]&m_{24}'\ar[dr]&\\
            m\ar[ur]\ar[dr]&&&m_{24}\\
            &m_4\ar[r]&m_{24}'\ar[ur]&
        \end{tikzcd}
    \end{center}
    and we will add on to these commutative diagrams one final time to obtain the following commutative diagram:
    \begin{center}
        \begin{tikzcd}
            &m_{13}\ar[r]&m_{0}'\ar[dr]&\\
            m\ar[ur]\ar[dr]&&&m_{0}\\
            &m_{24}\ar[r]&m_{0}'\ar[ur]&
        \end{tikzcd}
    \end{center}
    Adding all of our newest constructions to the large diagram, we get the following commutative diagram:
    \begin{center}
        \begin{tikzcd}
            &l'\ar[r]&k\ar[r]&m\ar{dr}&&&&&\\
            i'\ar{ur}\ar[dr]&&&&m_3\ar[r]&m_{13}'\ar[dr]&&&\\
            &n_i\ar[r]&n\ar[r]&m\ar{dr}\ar[ur]&&&m_{13}\ar[r]&m_0'\ar[ddr]&\\
            i\ar{ur}\ar[dr]&&&&m_1\ar[r]&m_{13}'\ar[ur]&&&\\
            &l\ar[r]&k\ar[r]&m\ar[ur]\ar[dr]&&&&&m_0\\
            j\ar[ur]\ar[dr]&&&&m_2\ar[r]&m_{24}'\ar[dr]&&&\\
            &n_j\ar[r]&n\ar[r]&m\ar[ur]\ar[dr]&&&m_{24}\ar[r]&m_0'\ar[uur]&\\
            j'\ar[ur]\ar[dr]&&&&m_4\ar[r]&m_{24}'\ar[ur]&&&\\
            &l'\ar[r]&k\ar[r]&m\ar[ur]&&&&&
        \end{tikzcd}
    \end{center}
    Now just think of this commutative diagram in $\Mod_A$ with the $A$-modules indexed by the elements in $\fI$ above because it's tedious to relabel the entire diagram. Because all of the morphisms are linear, if we want to show that $Fu(m_i)+Fv(m_j)\sim Fu'(m_{i'})+Fv'(m_{j'})$, it suffices to show that there exists morphisms $\chi_2:l'\to m_0$ and $\chi_1:l\to m_0$ such that $F(\chi_1\circ u)(m_i)= F(\chi_2\circ u')(m_{i'})$ and $F(\chi_1\circ v)(m_j)=F(\chi_2\circ v')(m_{j'})$. We claim that $\chi_1$ is the path from $l\to m_0$ in the above diagram and $\chi_2$ is the path from $l'\to m_0$. Recalling that $Fu(m_i)=Fu'(m_{i'})$, when tracking our elements $m_i,m_{i'},m_j,m_{j'}$, we have that the path $i'\to n_i=i\to n_i$ and $j'\to n_j=j\to n_j$. Therefore we track the path of $m_{i'}$ as
    \begin{center}
        \begin{tikzcd}
            &l'\ar[r,red]&k\ar[r,red]&m\ar[dr,red]&&&&&\\
            i'\ar[ur,red]\ar[dr]&&&&m_3\ar[r,red]&m_{13}'\ar[dr,red]&&&\\
            &n_i\ar[r]&n\ar[r]&m\ar{dr}\ar[ur]&&&m_{13}\ar[r,red]&m_0'\ar[ddr,red]&\\
            i\ar{ur}\ar[dr]&&&&m_1\ar[r]&m_{13}'\ar[ur]&&&\\
            &l\ar[r]&k\ar[r]&m\ar[ur]&&&&&m_0\\
        \end{tikzcd}
    \end{center}
    equals
    \begin{center}
        \begin{tikzcd}
            &l'\ar[r]&k\ar[r]&m\ar[dr]&&&&&\\
            i'\ar[ur]\ar[dr, red]&&&&m_3\ar[r,red]&m_{13}'\ar[dr,red]&&&\\
            &n_i\ar[r,red]&n\ar[r,red]&m\ar[dr]\ar[ur,red]&&&m_{13}\ar[r,red]&m_0'\ar[ddr,red]&\\
            i\ar{ur}\ar[dr]&&&&m_1\ar[r]&m_{13}'\ar[ur]&&&\\
            &l\ar[r]&k\ar[r]&m\ar[ur]&&&&&m_0\\
        \end{tikzcd}
    \end{center}
    equals
    \begin{center}
        \begin{tikzcd}
            &l'\ar[r]&k\ar[r]&m\ar[dr]&&&&&\\
            i'\ar[ur]\ar[dr]&&&&m_3\ar[r,red]&m_{13}'\ar[dr,red]&&&\\
            &n_i\ar[r,red]&n\ar[r,red]&m\ar[dr]\ar[ur,red]&&&m_{13}\ar[r,red]&m_0'\ar[ddr,red]&\\
            i\ar[ur,red]\ar[dr]&&&&m_1\ar[r]&m_{13}'\ar[ur]&&&\\
            &l\ar[r]&k\ar[r]&m\ar[ur]&&&&&m_0\\
        \end{tikzcd}
    \end{center}
    equals
    \begin{center}
        \begin{tikzcd}
            &l'\ar[r]&k\ar[r]&m\ar[dr]&&&&&\\
            i'\ar[ur]\ar[dr]&&&&m_3\ar[r]&m_{13}'\ar[dr]&&&\\
            &n_i\ar[r,red]&n\ar[r,red]&m\ar[dr,red]\ar[ur]&&&m_{13}\ar[r,red]&m_0'\ar[ddr,red]&\\
            i\ar[ur,red]\ar[dr]&&&&m_1\ar[r,red]&m_{13}'\ar[ur,red]&&&\\
            &l\ar[r]&k\ar[r]&m\ar[ur]&&&&&m_0\\
        \end{tikzcd}
    \end{center}
    equals
    \begin{center}
        \begin{tikzcd}
            &l'\ar[r]&k\ar[r]&m\ar[dr]&&&&&\\
            i'\ar[ur]\ar[dr]&&&&m_3\ar[r]&m_{13}'\ar[dr]&&&\\
            &n_i\ar[r]&n\ar[r]&m\ar[dr]\ar[ur]&&&m_{13}\ar[r,red]&m_0'\ar[ddr,red]&\\
            i\ar[ur]\ar[dr,red]&&&&m_1\ar[r,red]&m_{13}'\ar[ur,red]&&&\\
            &l\ar[r,red]&k\ar[r,red]&m\ar[ur,red]&&&&&m_0\\
        \end{tikzcd}
    \end{center}
    which demonstrates that $F(\chi_1\circ u)(m_i)=F(\chi_2\circ u')(m_{i'})$. On the other hand, tracking $m_{j'}$,
    \begin{center}
        \begin{tikzcd}
            &l\ar[r]&k\ar[r]&m\ar[dr]&&&&&m_0\\
            j\ar[ur]\ar[dr]&&&&m_2\ar[r]&m_{24}'\ar[dr]&&&\\
            &n_j\ar[r]&n\ar[r]&m\ar[ur]\ar[dr]&&&m_{24}\ar[r,red]&m_0'\ar[uur,red]&\\
            j'\ar[ur]\ar[dr,red]&&&&m_4\ar[r,red]&m_{24}'\ar[ur,red]&&&\\
            &l'\ar[r,red]&k\ar[r,red]&m\ar[ur,red]&&&&&
        \end{tikzcd}
    \end{center}
    equals
    \begin{center}
        \begin{tikzcd}
            &l\ar[r]&k\ar[r]&m\ar[dr]&&&&&m_0\\
            j\ar[ur]\ar[dr]&&&&m_2\ar[r]&m_{24}'\ar[dr]&&&\\
            &n_j\ar[r,red]&n\ar[r,red]&m\ar[ur]\ar[dr,red]&&&m_{24}\ar[r,red]&m_0'\ar[uur,red]&\\
            j'\ar[ur,red]\ar[dr]&&&&m_4\ar[r,red]&m_{24}'\ar[ur,red]&&&\\
            &l'\ar[r]&k\ar[r]&m\ar[ur]&&&&&
        \end{tikzcd}
    \end{center}
    equals
    \begin{center}
        \begin{tikzcd}
            &l\ar[r]&k\ar[r]&m\ar[dr]&&&&&m_0\\
            j\ar[ur]\ar[dr,red]&&&&m_2\ar[r]&m_{24}'\ar[dr]&&&\\
            &n_j\ar[r,red]&n\ar[r,red]&m\ar[ur]\ar[dr,red]&&&m_{24}\ar[r,red]&m_0'\ar[uur,red]&\\
            j'\ar[ur]\ar[dr]&&&&m_4\ar[r,red]&m_{24}'\ar[ur,red]&&&\\
            &l'\ar[r]&k\ar[r]&m\ar[ur]&&&&&
        \end{tikzcd}
    \end{center}
    equals
    \begin{center}
        \begin{tikzcd}
            &l\ar[r]&k\ar[r]&m\ar[dr]&&&&&m_0\\
            j\ar[ur]\ar[dr,red]&&&&m_2\ar[r,red]&m_{24}'\ar[dr,red]&&&\\
            &n_j\ar[r,red]&n\ar[r,red]&m\ar[ur,red]\ar[dr]&&&m_{24}\ar[r,red]&m_0'\ar[uur,red]&\\
            j'\ar[ur]\ar[dr]&&&&m_4\ar[r]&m_{24}'\ar[ur]&&&\\
            &l'\ar[r]&k\ar[r]&m\ar[ur]&&&&&
        \end{tikzcd}
    \end{center}
    equals
    \begin{center}
        \begin{tikzcd}
            &l\ar[r,red]&k\ar[r,red]&m\ar[dr,red]&&&&&m_0\\
            j\ar[ur,red]\ar[dr]&&&&m_2\ar[r,red]&m_{24}'\ar[dr,red]&&&\\
            &n_j\ar[r]&n\ar[r]&m\ar[ur]\ar[dr]&&&m_{24}\ar[r,red]&m_0'\ar[uur,red]&\\
            j'\ar[ur]\ar[dr]&&&&m_4\ar[r]&m_{24}'\ar[ur]&&&\\
            &l'\ar[r]&k\ar[r]&m\ar[ur]&&&&&
        \end{tikzcd}
    \end{center}
    which shows that $F(\chi_1 \circ v)(m_j)=F(\chi_2\circ v')(m_{j'})$. This proves that
    \[
    F\chi_1(Fu(m_i)+Fv(m_j))=F\chi_2(Fu'(m_{i'})+Fv'(m_{j'}))
    \]
    so that indeed
    \[
    [(Fu(m_i)+Fv(m_j),l)]=[(Fu'(m_{i'})+Fv'(m_{j'}),l')]
    \]
    so addition is independent of choice of $u,v,$ and $l$.\\
    \newline
    We define multiplication as $a[(m_i,i)]=[(am_i,i)]$. To show multiplication is well defined, suppose $(m_i,i)\sim (m_j,j)$. Then for some $f:i\to k$ and some $g:j\to k$, $Ff(m_i)=Fg(m_j)$. Then
    \[
    Ff(am_i)=aFf(m_i)=aFg(m_j)=Fg(am_j)
    \]
    implies 
    \[
    (m_i,i)\sim(m_j,j)\Rightarrow (am_i,i)\sim (am_j,j)\Rightarrow a[(m_i,i)]=a[(m_j,j)]
    \]
    demonstrating multiplication is well defined.\\
    \newline
    The module axioms are readily verifiable. Now suppose we have an $A$-module $W$ that satisfies the commutativity of the diagram indexed by $\fI$ equipped with morphisms $\alpha_i$. We want to show
    \begin{center}
        \begin{tikzcd}
            &W&\\
            &\varinjlim_{\fI} M_i\ar[dashed]{u}[description]{\exists!}\\
            M_i \ar[bend left]{uur}{\alpha_i} \ar{ur} \ar{rr}{Ff}&&M_j\ar{ul} \ar[bend right]{uul}[swap]{\alpha_j}
        \end{tikzcd}
    \end{center}
    We will construct such a unique map. By commutativity, we are required that $\varphi([(m_i,i)])=\alpha_i(m_i)$. This proves $\varphi$ is unique. To prove $\varphi$ is well defined, suppose $(m_i,i)\sim (m_j,j)$. Then there exists some $f:i\to k$ and some $g:j\to k$ such that $Ff(m_i)=Fg(m_j)$. Then the following diagram must commute:
    \begin{center}
        \begin{tikzcd}
            &W&\\
            M_i \ar[bend left]{ur}{\alpha_i} \ar{r}{Ff}&M_k\ar{u}{\alpha_k}&M_j\ar{l}[swap]{Fg} \ar[bend right]{ul}[swap]{\alpha_j}
        \end{tikzcd}
    \end{center}
    Therefore
    \[
    \alpha_i(m_i)=\alpha_k\circ Ff(m_i)=\alpha_k\circ Fg(m_j)=\alpha_j(m_j)
    \]
    so $\varphi$ is well defined. To show $\varphi$ is linear, we have
    \begin{align*}
        \varphi([(m_i,i)]+[(m_j,j)])=\varphi([Ff(m_i)+Fg(m_j),k])=\alpha_k(Ff(m_i)+Fg(m_j))\\
        =\alpha_k\circ Ff(m_i)+\alpha_k\circ Fg(m_j)=\alpha_i(m_i)+\alpha_j(m_j)=\varphi([(m_i,i)])+\varphi([(m_j,j)])
    \end{align*}
    by the below commutative diagram:
    \begin{center}
        \begin{tikzcd}
            &W&\\
            &\varinjlim_{\fI} M_i\ar[dashed]{u}[description]{\exists!}\\
            M_i \ar[bend left]{uur}{\alpha_i} \ar{ur} \ar{r}{Ff}&M_k\ar{u}&M_j\ar{l}[swap]{Fg}\ar{ul} \ar[bend right]{uul}[swap]{\alpha_j}
        \end{tikzcd}
    \end{center}
    Additionally,
    \begin{align*}
        \varphi(a[(m_i,i)])=\varphi([(am_i,i)])=\alpha_i(am_i)=a\alpha_i(m_i)=a\varphi([(m_i,i)])
    \end{align*}
    which proves existence.
\end{proof}
\subsubsection{G}\label{1.4.G}
\begin{proof}
    We take the index category to be the elements of $S$ where there is an arrow $s:s_1\to s_2$ if and only if $s_2=ss_1$ for some $s\in S$. In this case, we define the map $Fs:\frac{1}{s_1}A\to \frac{1}{ss_1}$ as
    \[
    \frac{a}{s_1}\mapsto \frac{sa}{ss_1}
    \]
    Our index category is filtered as for any $s_1,s_2\in S$, there is an arrow $s_1\to s_1s_2$ and an arrow $s_2\to s_1s_2$. By construction, there is at most one arrow from any object to any other object, so the second condition is trivially true. To show that $\varinjlim \frac{1}{s}A$ is isomorphic to $S^{-1}A$, we will first show the existence of a morphism $\varphi:\varinjlim \frac{1}{s}A\to S^{-1}A$. For each $s\in S$, we define a map $\iota_s: \frac{1}{s}A\to S^{-1}A$ given by $\frac{a}{s}\mapsto \frac{a}{s}$. To induce the map $\varphi$, we want to show the following diagram commutes:
    \begin{center}
        \begin{tikzcd}
            &S^{-1}A&\\
            \frac{1}{ss_1}A\ar{ur}{\iota_{ss_1}}&&\frac{1}{s_1}A \ar{ll}[swap]{F(s)} \ar{ul}[swap]{\iota_{s_1}}
        \end{tikzcd}
    \end{center}
    We have
    \[
    \iota_{ss_1}\circ F(s)(\frac{a}{s_1})=\iota_{ss_1}(\frac{as}{s_1s})=\frac{as}{s_1s}=\frac{a}{s}=\iota_s(\frac{a}{s})
    \]
    which proves the diagram commutes, hence we obtain the induced morphism $\varphi:\varinjlim \frac{1}{s}A\to S^{-1}A$. Now to find the inverse morphism, we will use the universal property of $S^{-1}A$. We construct a map $\alpha:A\to \varinjlim \frac{1}{s}A$ given by $\alpha(a)=\frac{a}{1}$. To show that for any $s\in S$, multiplication by $s$ is an automorphism of $\varinjlim \frac{1}{s}A$, if we take any $\frac{a}{s'}\in \varinjlim \frac{1}{s}A$ such that\[
    s\frac{a}{s'}=0\iff
    \frac{sa}{s'}=0
    \]
    if and only if there exists some $F(r)$ such that $F(r)(\frac{sa}{s'})=0$, which by definition means
    \[
    \frac{rsa}{s'r}=0
    \]
    which is true if and only if there exists some $r'\in S$ such that
    \[
    r'rsa=0
    \]
    Assuming that $0\notin S$, using the fact that $A$ is an integral domain and $r'rs\in S$ implies
    \[
    a=0
    \]
   Therefore multiplication is injective. To show multiplication by $s$ is surjective, fix any $\frac{a}{s'}\in \varinjlim \frac{1}{s}A$. Then
   \[
   s \frac{a}{s's}=\frac{as}{ss'}=(\frac{as}{ss'})=F(s)(\frac{a}{s'})=(\frac{a}{s'})=\frac{a}{s'}
   \]
   so indeed multiplication by $s$ is an automorphism. Therefore we use the following universal property to get a map $\phi:S^{-1}A\to \varinjlim \frac{1}{s}A$:
   \begin{center}
       \begin{tikzcd}
           A\ar{r} \ar{dr}{\alpha}& S^{-1}A\ar[dashed]{d}[description]{\exists!}\\
           & \varinjlim \frac{1}{s}A
       \end{tikzcd}
   \end{center}
   These can easily checked to be inverses of each other, which proves the isomorphism. Equivalently, we could have just used the previous exercise to look at the structure of $\varinjlim \frac{1}{s}A$, and observed that the underlying sets are the same, and then proved the structures are isomorphic as well.
\end{proof}
\subsubsection{H}\label{1.4.H}
\begin{proof}
    The commutativity of 
    \begin{center}
        \begin{tikzcd}
            & \bigoplus_{i\in \fI} M_i/\sim\\
            M_i\ar{ur}{\iota_i} \ar{r}{F(n)}&M_j\ar{u}{\iota_j}
        \end{tikzcd}
    \end{center}
    commutes by definition of $\sim$. Now suppose
    \begin{center}
        \begin{tikzcd}
            & W\\
            M_i\ar{ur}{g_i} \ar{r}{F(n)}&M_j\ar{u}{g_j}
        \end{tikzcd}
    \end{center}
    commutes. If 
    \begin{center}
        \begin{tikzcd}
        &W\\
            & \bigoplus M_i/\sim\ar[dashed]{u}[description]{\exists!}\\
            M_i\ar[bend left]{uur}{g_i}\ar{ur}{\iota_i} \ar{r}{F(n)}&M_j\ar{u}{\iota_j}\ar[bend right=80]{uu}[swap]{g_j}
        \end{tikzcd}
    \end{center}
    the induced morphism $\varphi$ would have to satisfy $\varphi\circ \iota_i=g_i$. By linearity, this determines $\varphi$ completely as
    \[
    \varphi(\sum_i \iota_i(m_i))=\sum_i \varphi\circ \iota_i(m_i)=\sum_i g_i(m_i)
    \]
    because also every element of the direct sum is in the image of one of the $\iota$'s.
    This proves uniqueness. To show $\varphi$ is well defined on equivalence classes, suppose $F(n)(m_i)=m_j\iff m_j\sim m_i$. Then
    \[
    \varphi\circ \iota_i(m_i)=g_i(m_i)=g_j\circ F(n)(m_i)=g_j(m_j)=\varphi\circ \iota_j(m_j)
    \]
    so $\varphi$ is well defined. $\varphi$ is $A$-linear as 
    \[
    \varphi(\sum_i\iota_i(m_i)+\sum_j\iota_j(n_j))=\varphi(\sum_i \iota_i(m_i+n_i)=\sum_i \varphi \circ \iota_i(m_i+n_i)=\sum_i\varphi\circ \iota_i(m_i)+\sum_j \varphi\circ \iota_j(n_j)
    \]
    and 
    \[
    \varphi(a\iota_i(m_i))=\varphi(\iota_i(am_i))=g_i(am_i)=ag_i(m_i)=a\varphi(\iota_i(m_i))
    \]
    Then indeed the construction is the colimit.
\end{proof}
\subsection{}
\subsubsection{A}\label{1.5.A}
\begin{proof}
The diagram is below where $g_*=g\circ$ and $Gg_*=Gg\circ $:
    \begin{center}
        \begin{tikzcd}
            \Mor_\fB(F(A),B)\ar{r}{g_*} \ar{d}{\tau_{AB}}& \Mor_\fB(F(A),B')\ar{d}{\tau_{AB'}}\\
            \Mor_\fA(A,G(B))\ar{r}{Gg_*}&\Mor_\fA(A,G(B')) 
        \end{tikzcd}
    \end{center}
    
\end{proof}
\subsubsection{B}\label{1.5.B}
\begin{proof}
    We define $\eta_A$ to be $\tau_{AF(A)}(\id_{F(A)})$ and $\epsilon_B$ as $\tau_{FG(B)B}^{-1}(\id_{G(B)})$. Tracking $\eta_A$ on the bottom of the diagram below: 
    \begin{center}
        \begin{tikzcd}
            \Mor_\fB(F(A),F(A))\ar{r}{g_*} \ar{d}{\tau_{AF(A)}}& \Mor_\fB(F(A),B)\ar{d}{\tau_{AB}}\\
            \Mor_\fA(A,GF(A))\ar{r}{Gg_*}&\Mor_\fA(A,G(B)) 
        \end{tikzcd}
    \end{center}
    we see
    \[
    Gg_*(\eta_A)=Gg\circ \eta_A
    \]
    while on the top we get
    \[
    \tau_{AB}\circ g_*(\id_{F(A)})=\tau_{AB}(g\circ \id_{F(A)})=\tau_{AB}(g)
    \]
    By commutativity, the two must be equal, and $g\in \Mor_\fB(F(A),B)$ was arbitrary.\\
    \newline
    For $\epsilon_B$, we will use the following diagram:
    \begin{center}
        \begin{tikzcd}
            \Mor_\fB(FG(B),B)\ar{r}{Ff^*} \ar{d}{\tau_{G(B)B}}& \Mor_\fB(F(A),B)\ar{d}{\tau_{AB}}\\
            \Mor_\fA(G(B),G(B))\ar{r}{f^*}&\Mor_\fA(A,G(B)) 
        \end{tikzcd}
    \end{center}
    On one hand, we get
    \[
    Ff^*(\epsilon_B)=\epsilon_B\circ Ff
    \]
    while on the other hand we have
    \[
    \tau_{AB}^{-1}\circ f^*\circ \tau_{G(B)B}(\eta_B)=\tau_{AB}^{-1}\circ f^*(\id_{G(B)})=\tau_{AB}^{-1}(\id_{G(B)}\circ f)=\tau_{AB}^{-1}(f)
    \]
    and by commutativity the two are equal, where $f\in \Mor_\fA(A,G(B))$ was arbitrary.
\end{proof}
\subsubsection{C}\label{1.5.C}
\begin{proof}
    We will use the following universal property:
    \begin{center}
        \begin{tikzcd}
            M\times N \ar{r} \ar{dr}{\alpha}& M\otimes N\ar[dashed]{d}[description]{\exists!}\\
            &P
        \end{tikzcd}
    \end{center}
    For an arbitrary $\phi\in \Hom(M, \Hom(N,P))$, we let $\alpha(m,n)=\phi(m)(n)$. Then
    \[
    \alpha(m_1+m_2,n)=\phi(m_1+m_2)(n)=\phi(m_1)(n)+\phi(m_2)(n)=\alpha(m_1,n)+\alpha(m_2,n)
    \]
    and
    \[
    \alpha(m,n_1+n_2)=\phi(m)(n_1+n_2)=\phi(m)(n_1)+\phi(m)(n_2)=\alpha(m,n_1)+\alpha(m,n_2)
    \]
    as well as
    \[
    \alpha(am,n)=\phi(am)(n)=a\phi(m)(n)=\phi(m)(an)=\alpha(m,an)
    \]
    which proves $\alpha$ is bilinear, hence we get our induced map $\beta_\phi \in \Hom(M\otimes N, P)$. Then we can define a map $\varphi:\Hom(M, \Hom(N,P))\to \Hom(M\otimes N, P)$ as $\varphi(\phi)=\beta_\phi$.\\
    \newline
    On the other hand, if we have some $\phi \in \Hom(M\otimes N,P)$, we will define some $\gamma \in \Hom(M, \Hom(N,P))$ where $\gamma(m)(n)=\phi(m\otimes n)$. Then indeed for any arbitrary $m\in M$, $\gamma(m)\in \Hom(N,P)$ because
    \[
    \gamma(m)(n_1+n_2)=\phi(m\otimes n_1+n_2)=\phi(m\otimes n_1)+\phi(m\otimes n_2)=\gamma(m)(n_1)+\gamma(m)(n_2)
    \]
    and
    \[
    \gamma(m)(an)=\phi(m\otimes an)=a\phi(m\otimes n)=a\gamma(m)(n)
    \]
    Also $\gamma \in \Hom(M, \Hom(N,P))$ because
    \[
    \gamma(m_1+m_2)(n)=\phi(m_1+m_2\otimes n)=\phi(m_1\otimes n)+\phi(m_2\otimes n)=\gamma(m_1)(n)+\gamma(m_2)(n)
    \]
    and
    \[
    \gamma(am)(n)=\phi(am\otimes n)=a\phi(m\otimes n)=a\gamma(m)(n)
    \]
    Therefore we can define a map $\tilde \varphi:\Hom(M\otimes N,P)\to \Hom(M, \Hom(N,P))$ given by $\tilde \varphi(\phi)=\gamma_\phi$. To show that $\varphi$ is a bijection, we observe
    \[
    \varphi \circ \tilde \varphi(\phi)(m\otimes n)=\varphi(\gamma_\phi)(m\otimes n)=\gamma_\phi(m)(n)=\phi(m\otimes n)
    \]
    and
    \[
    \tilde \varphi\circ \varphi(\phi)(m)(n)=\tilde \varphi(\beta_\phi)(m)(n)=\beta_\phi(m\otimes n)=\phi(m)(n)
    \]
    so indeed $\tilde \varphi=\varphi^{-1}$ and $\varphi$ is a bijection.
\end{proof}
\subsubsection{D}\label{1.5.D}
\begin{proof}
    We fix arbitrary $f\in \Hom(A',A)$ and $g\in \Hom(B,B')$ and define \\$\tau_{AB}:\Hom(A\otimes N,B)\to \Hom(A, \Hom(N,B))$ as $\varphi^{-1}$ in the previous exercise, which we proved was a bijection. We first want to show that the following diagram commutes:
    \begin{center}
        \begin{tikzcd}
            \Hom(A\otimes N,B)\ar{r}{f\otimes N^*} \ar{d}{\tau_{AB}}& \Hom(A'\otimes N,B)\ar{d}{\tau_{A'B}}\\
            \Hom(A, \Hom(N,B))\ar{r}{f^*}& \Hom(A', \Hom(N,B))
        \end{tikzcd}
    \end{center}
    Fixing any $\phi \in \Hom(A\otimes N,B)$ and any $a'\in A'$ and $n\in N$, we get on one hand that
    \begin{align*}
        \tau_{A'B}\circ f\otimes N^*(\phi)(a')(n)=\tau_{A'B}(\phi \circ f\otimes N)(a')(n)=\phi\circ f\otimes N(a'\otimes n)=\phi(f(a')\otimes n)
    \end{align*}
    On the other side of the diagram, we get
    \begin{align*}
        f^*\circ \tau_{AB}(\phi)(a')(n)=\tau_{AB}(\phi)\circ f(a')(n)=\phi(f(a')\otimes n)
    \end{align*}
    which proves the diagram does commute. Now we want to show the below diagram commutes as well:
    \begin{center}
        \begin{tikzcd}
            \Hom(A\otimes N,B)\ar{r}{g_*} \ar{d}{\tau_{AB}}& \Hom(A\otimes N,B')\ar{d}{\tau_{AB'}}\\
            \Hom(A, \Hom(N,B))\ar{r}{g_{**}}& \Hom(A, \Hom(N,B'))
        \end{tikzcd}
    \end{center}
    where $g_*$ is as usual and $g_{**}=(g_*)_*$. Fixing any $a\in A, n\in N$ and $\phi\in \Hom(A\otimes N,B)$, we get on the top that
    \[
    \tau_{AB'}\circ g_*(\phi)(a)(n)=\tau_{AB'}(g\circ \phi)(a)(n)=g\circ \phi(a\otimes n)
    \]
    On the bottom, we get
    \[
    g_{**}\circ \tau_{AB}(\phi)(a)(n)=g_*\circ \tau_{AB}(\phi)(a)(n)=g\circ \tau_{AB}(\phi)(a)(n)=g\circ \phi(a\otimes n)
    \]
    which shows this diagram commutes as well, proving that $\cdot \otimes N$ and $\Hom(N,\cdot)$ are adjoint functors.
\end{proof}
\subsubsection{E}\label{1.5.E}
\begin{proof}
    We want to first show that the following diagram commutes:
    \begin{center}
        \begin{tikzcd}
            \Hom(N\otimes_B A,M)\ar{r}{f\otimes A^*} \ar{d}{\tau_{NM}}& \Hom(N'\otimes_B A,M)\ar{d}{\tau_{N'M}}\\
            \Hom(N,M_B)\ar{r}{f^*}& \Hom(N',M_B)
        \end{tikzcd}
    \end{center}
    To do this, we first need to define what $\tau_{NM}$ is. Given any $\phi \in \Hom(N\otimes_B A,M)$, let $\varphi(\phi)\in \Hom(N,M_B)$ act as
    \[
    \varphi(\phi)(n)=\phi(n\otimes 1)
    \]
    To show $\varphi(\phi)$ is actually $B$-linear, we observe
    \[
    \varphi(\phi)(n_1+n_2)=\phi(n_1+n_2\otimes 1)=\phi(n_1\otimes 1)+\phi(n_2\otimes 1)=\varphi(\phi)(n_1)+\varphi(\phi)(n_2)
    \]
    as well as
    \[
    \varphi(\phi)(bn)=\phi(bn\otimes 1)=b\phi(n\otimes 1)=b\varphi(\phi)(n)
    \]
    On the other hand if we have some $\phi'\in \Hom(N,M_B)$, let $\tilde \varphi(\phi')\in \Hom(N\otimes_BA,M)$ act as
    \[
    \tilde \varphi(\phi')(n\otimes a)=a \phi'(n)
    \]
    where we can consider elements of $M$ as elements of $M_B$ and vice versa. To show $\tilde \varphi(\phi')$ is well defined, we define for $\phi'$ an $\alpha:N\times A\to M$ as $\alpha(n,a)=a\phi'(n)$. Then
    \begin{align*}
        \alpha(n_1+n_2,a)=a \phi'(n_1+n_2)=a\phi'(n_1)+a \phi'(n_2)=\alpha(n_1,a)+\alpha(n_2,a)
    \end{align*}
    and
    \[
    \alpha(n,a_1+a_2)=(a_1+a_2)\phi'(n)=a_1 \phi'(n)+a_2 \phi'(n)=\alpha(n,a_1)+\alpha(n,a_2)
    \]
    as well as
    \[
    \alpha(bn,a)=a\phi'(bn)=ba\phi'(n)=\alpha(n,ba)
    \]
    which demonstrates $\tilde \varphi$ satisfies the universal property below:
    \begin{center}
        \begin{tikzcd}
            N\times A \ar{r} \ar{dr}{\alpha}& N\otimes_B A\ar[dashed]{d}[description]{\exists!}\\
            &M
        \end{tikzcd}
    \end{center}
    Now to show $\tilde \varphi=\varphi^{-1}$ and $\varphi$ is a bijection, we have
    \[
    \varphi \circ \tilde \varphi(\phi')(n)=\tilde \varphi(\phi')(n\otimes 1)=\phi'(n)
    \]
    as well as
    \[
    \tilde \varphi \circ \varphi(\phi)(n\otimes a)=a\varphi(\phi)(n)=a\phi(n\otimes 1)=\phi(n\otimes a)
    \]
    which proves the two are inverses and are bijective.\\
    \newline
    Therefore we define $\tau_{NM}$ as $\varphi$ was above, so we've already shown $\tau_{NM}$ is a bijection. Now back to the diagram below,
    \begin{center}
        \begin{tikzcd}
            \Hom(N\otimes_B A,M)\ar{r}{f\otimes A^*} \ar{d}{\tau_{NM}}& \Hom(N'\otimes_B A,M)\ar{d}{\tau_{N'M}}\\
            \Hom(N,M_B)\ar{r}{f^*}& \Hom(N',M_B)
        \end{tikzcd}
    \end{center}
    we fix any $\phi\in \Hom(N\otimes_B A,M)$ and any $n'\in N'$, then
    \[
    \tau_{N'M}\circ f\otimes A^*(\phi)(n')=f\otimes A^*(\phi)(n'\otimes 1)=\phi \circ f\otimes A(n'\otimes 1)=\phi(f(n')\otimes 1)
    \]
    On the bottom side, we get
    \[
    f^*\circ \tau_{NM}(\phi)(n')=\tau_{NM}(\phi)\circ f(n')=\tau_{NM}(\phi)(f(n'))=\phi(f(n')\otimes 1)
    \]
    so the diagram commutes. To show
\begin{center}
        \begin{tikzcd}
            \Hom(N\otimes_B A,M)\ar{r}{g_*} \ar{d}{\tau_{NM}}& \Hom(N\otimes_B A,M')\ar{d}{\tau_{NM'}}\\
            \Hom(N,M_B)\ar{r}{g_*}& \Hom(N,M'_B)
        \end{tikzcd}
    \end{center}
    the above diagram commutes where the $g_*$ on the bottom is now considered to be $B$-linear, fix any $\phi\in \Hom(N\otimes_B A,M)$ and any $n\in N$. Then
    \[
    \tau_{NM'}\circ g_*(\phi)(n)=\tau_{NM'}(g\circ \phi)(n)=g\circ \phi(n\otimes 1)
    \]
    On the bottom,
    \[
    g_*\circ \tau_{NM}(\phi)(n)=g\circ \tau_{NM}(\phi)(n)=g\circ \phi(n\otimes 1)
    \]
    This proves that $\cdot_B$ is right adjoint to $\cdot \otimes_B A$.
\end{proof}
\subsubsection{F}\label{1.5.F}
\begin{proof}
    If $G$ is an abelian group, then we claim that the following diagram commutes for every map of abelian semigroups $\varphi$ and every abelian group $H$:
    \begin{center}
        \begin{tikzcd}
            G\ar{r}{\id_G}\ar{dr}{\varphi}&G\ar[dashed]{d}[description]{\exists!}\\
            & H
        \end{tikzcd}
    \end{center}
    If such a unique map $\phi$ were to exist, then it would satisfy
    \[
    \phi \circ \id_G=\varphi \Rightarrow \phi=\varphi
    \]
    This proves uniqueness. Existence is obvious since $\phi=\varphi$ gives existence.
\end{proof}
\subsubsection{G}\label{1.5.G}
\begin{proof}
    We will take the construction given in the problem for our construction, where
    \[
    [(a,b)]+[(c,d)]=[(a+c,b+d)]
    \]
    Suppose that $(a,b)\sim (a',b')$ and $(c,d)\sim (c',d')$. Then there exists $e_1$ and $e_2\in S$ where
    \[
    a+b'+e_1=b+a'+e_1
    \]
    and where
    \[
    c+d'+e_2=d+c'+e_2
    \]
    It follows that $(a+c,b+d)\sim (a'+c',b'+d')$ because 
    \begin{align*}
        a+c+b'+d'+(e_1+e_2)=(a+b'+e_1)+(c+d'+e_2)\\
        =(b+a'+e_1)+(d+c'+e_2)=a'+c'+b+d+(e_1+e_2)
    \end{align*}
    Therefore addition is well defined. Also
    \begin{align*}
        [(a,b)]+[(c,d)]=[(a+c,b+d)]=[(c+a,d+b)]=[(c,d)]+[(a,b)]
    \end{align*}
    so addition is commutative. In a similar fashion, addition is associate because it is in $S$. Because $S$ is nonempty, there exists some $s\in S$, so we claim that $[(s,s)]$ is the identity on $H(S)$. It's clear that this identity is independent of the choice of $s\in S$ because for any $s,r\in S$,
    \[
    s+r+s=r+s+s\Rightarrow (s,s)\sim(r,r)
    \]
    To show $[(s,s)]=0$, for any $a,b\in S$, we have
    \[
    [(c,c)]+[(a,b)]=[(c+a,c+b)]=[(a,b)]
    \]
    because 
    \[
    c+a+b+a=a+c+b+a\Rightarrow (c+a,c+b)\sim (a,b)
    \]
    Also, we claim that $[(a,b)]^{-1}=[(b,a)]$. To show this,
    \[
    [(a,b)]+[(b,a)]=[(a+b,b+a)]=[(a+b,a+b)]=0
    \]
    Therefore the abelian semigroup map is
    \[
    s\mapsto [(s+s,s)]
    \]
    To show this map $\varphi$ is linear, we see
    \[
    \varphi(a+b)=[(a+b+a+b,a+b)]=[(a+a,a)]+[(b+b,b)]=\varphi(a)+\varphi(b)
    \]
    Now to show that $H$ is left-adjoint to the forgetful functor $F$, we first want to show the following diagram commutes:
    \begin{center}
        \begin{tikzcd}
            \Hom(H(A),B) \ar{r}{Hf^*} \ar{d}{\tau_{AB}}& \Hom(H(A'),B)\ar{d}{\tau_{A'B}}\\
            \Mor(A,F(B))\ar{r}{f^*}& \Mor(A', F(B))
        \end{tikzcd}
    \end{center}
     where we define $Hf([(a,b)])=[(f(a),f(b))]$ and where we define $\tau_{AB}(\phi)(a)=\phi([(a+a,a)])$ and $\tau_{AB}^{-1}(\varphi)([(a,b)])=\varphi(a)-\varphi(b)$. To show $\tau_{AB}^{-1}$ is well defined, suppose $(a,b)\sim (c,d)$. Then there exists some $e\in A$ such that
     \[
     a+d+e=c+b+e
     \]
     By linearity of $\varphi$, we get
     \[
     \varphi(a)+\varphi(d)+\varphi(e)=\varphi(c)+\varphi(b)+\varphi(e)
     \]
     Because now these are considered as objects of $B$, we have cancellation so
     \[
     \varphi(a)+\varphi(d)=\varphi(c)+\varphi(b)
     \]
     By subtraction in $B$, we get
     \[
     \varphi(a)-\varphi(b)=\varphi(c)-\varphi(d)\Rightarrow \tau_{AB}^{-1}(\varphi)([(a,b)])=\tau_{AB}^{-1}(\varphi)([(c,d)])
     \]
     Then for any $\varphi \in \Mor(A,F(B))$ and any $a\in A$, we get
     \[
     \tau_{AB}\circ \tau_{AB}^{-1}(\varphi)(a)=\tau_{AB}^{-1}(\varphi)([(a+a,a)])=\varphi(a+a)-\varphi(a)=\varphi(a)
     \]
     and on the other hand
     \[
     \tau_{AB}^{-1}\circ \tau_{AB}(\phi)([(a,b)])=\tau_{AB}(\phi)(a)-\tau_{AB}(\phi)(b)=\phi([(a+a,a)])-\phi([(b+b,b)])
     \]
     \[
     =\phi([(a+a,a)]-[(b+b,b)])=\phi([(a+a,a)]+[(b,b+b)])=\phi([(a+a+b,a+b+b)])=\phi([(a,b)])
     \]
     This proves that our $\tau_{AB}^{-1}$ is actually the inverse of $\tau_{AB}$ and that both are indeed bijections. To prove the diagram commutes, we have for any $a'\in A'$ and any $\phi\in \Hom(H(A),B)$,
     \begin{align*}
         \tau_{A'B}\circ Hf^*(\phi)(a')=Hf^*(\phi)([(a'+a',a')])=\phi\circ Hf([(a'+a',a')])\\
         =\phi([f(a')+f(a'),f(a')])
     \end{align*}
     On the other hand,
     \begin{align*}
         f^*\circ \tau_{AB}(\phi)(a')=\tau_{AB}(\phi)(f(a'))=\phi([f(a')+f(a'),f(a')])
     \end{align*}
     so this diagram does indeed commute. Now we want to show the following diagram commutes:
     \begin{center}
         \begin{tikzcd}
             \Hom(H(A),B)\ar{r}{g_*} \ar{d}{\tau_{AB}}& \Hom(H(A),B')\ar{d}{\tau_{AB'}}\\
             \Mor(A,F(B))\ar{r}{Fg_*}& \Mor(A,F(B'))
         \end{tikzcd}
     \end{center}
     Then for any $\phi \in \Hom(H(A),B)$ and any $a\in A$,
     \begin{align*}
         \tau_{AB'}\circ g_*(\phi)(a)=g_*(\phi)([(a+a,a)])=g\circ \phi([(a+a,a)])
     \end{align*}
     while along the bottom,
     \[
     Fg_*\circ \tau_{AB}(\phi)(a)=Fg\circ \tau_{AB}(\phi)(a)=Fg\circ \phi([(a+a,a)])=g\circ \phi([(a+a,a)])
     \]
     since $Fg(x)=g(x)$ for all $x\in F(B)$. Therefore both diagrams commute, which proves that $H$ is left adjoint to $F$.
\end{proof}
\subsubsection{H}\label{1.5.H}
\begin{proof}
    To show this embedding is fully faithful, it suffices to show that every morphism $f:M\to N$ in $\Mod_A$ defines a unique morphism $f:S^{-1}M\to S^{-1}N$ in $\Mod_{S^{-1}A}$ because it's clear that every $\Mod_{S^{-1}A}$ morphism defines a unique $\Mod_A$ morphism. By the universal property of $M$ and $N$, if $f:M\to N$ then we have the following commutative diagram:
    \begin{center}
        \begin{tikzcd}
            M\ar[hook, two heads]{r} \ar{dr}{f}&S^{-1}M\ar[dashed]{d}[description]{\exists!}\\
            &N
        \end{tikzcd}
    \end{center}
    because $S^{-1}M\cong M$ when $M$ is already an $S^{-1}A$ module. Also $N\cong S^{-1}N$ yields the desired unique $f':S^{-1}M\to S^{-1}N$.
    We could understand the action of the induced map $f':S^{-1}M\to S^{-1}N$ by noticing that
    \[
    1=f'(\frac{s}{s})=sf'(\frac{1}{s})=f'(\frac{1}{s})s
    \]
    so that
    \[
    f'(\frac{1}{s})=\frac{1}{s}
    \]
    which defines the desired $S^{-1}A$ module homomorphism which must act as
    \[
    f'(\frac{m}{s})=\frac{1}{s}\frac{f(m)}{1}=\frac{f(m)}{s}
    \]
    Furthermore, the forgetful functor applied to this induced homomorphism is indeed the original map $f$.\\
    If we let $L:\Mod_A\to \Mod_{S^{-1}A}$ be the localization functor, we claim that $L$ is left adjoint to the forgetful $F:\Mod_{S^{-1}A}\to \Mod_A$. For any objects $X,Y\in \Mod_A$ and $W,Z\in \Mod_{S^{-1}A}$ and any $f:Y\to X$, we have
    \begin{center}
        \begin{tikzcd}
            \Hom_{\Mod_{S^{-1}A}}(L(X),Z)\ar{d}{\tau_{XZ}} \ar{r}{Lf^*}& \Hom_{\Mod_{S^{-1}A}}(L(Y),Z)\ar{d}{\tau_{YZ}}\\
            \Hom_{\Mod_A}(X,F(Z))\ar{r}{f^*}&\Hom_{\Mod_A}(Y, F(Z))
        \end{tikzcd}
    \end{center}
    where $\tau$ is just the forgetful functor acting on homomorphisms, which commutes because
    \[
    f^*\circ \tau_{XZ}(g)(y)=\tau_{XZ}(g)\circ f(y)=g\circ f(y)
    \]
    and
    \[
    \tau_{YZ}\circ Lf^*(g)(y)=\tau_{YZ}(g\circ Lf)(y)=g\circ f(y)
    \]
    where $g\in \Hom_{\Mod_{S^{-1}A}}(L(X),Z)$ and $y\in Y$ were arbitrary. On the other hand,
    \begin{center}
        \begin{tikzcd}
            \Hom_{\Mod_{S^{-1}A}}(L(X),W)\ar{d}{\tau_{XW}} \ar{r}{g_*}& \Hom_{\Mod_{S^{-1}A}}(L(X),Z)\ar{d}{\tau_{XZ}}\\
            \Hom_{\Mod_A}(X,F(W))\ar{r}{Fg_*}&\Hom_{\Mod_A}(X, F(Z))
        \end{tikzcd}
    \end{center}
    which commutes because
    \[
    \tau_{XZ}\circ g_*(f)(x)=\tau_{XZ}(g\circ f)(x)=g\circ f(x)
    \]
    as well as
    \[
    Fg_*\circ \tau_{XY}(f)(x)=Fg\circ \tau_{XY}(f)(x)=g\circ f(x)
    \]
    where $g:W\to Z$, $f:L(X)\to W$ and $x\in X$ are arbitrary.\\
    Then indeed $L$ is left adjoint to $F$.
\end{proof}
\subsection{}
\subsubsection{A}\label{1.6.A}
\begin{proof}
    $\im f^i\xhookrightarrow{\iota^i} A^{i+1}$ by Lemma $\ref{lem:ker monic}$, so $0\rightarrow \im f^i \xrightarrow{\iota^i} A^{i+1}$ being exact is clear. Furthermore, if $\pi^i:A^{i+1}\twoheadrightarrow \cok f^i$ is the projection, $\ker \pi^i=\im \iota^i$ so $\im f^i \xrightarrow{\iota^i} A^{i+1}\xrightarrow{\pi^i} \cok f^i$ is exact. Finally, $\pi^i$ is epic by Lemma \ref{lem:cok epic}, which shows $A^{i+1}\xrightarrow{\pi^i} \cok f^i\rightarrow 0$ is exact as well, thus proving
    \[
    0\rightarrow \im f^i\xrightarrow{\iota^i} A^{i+1}\xrightarrow{\pi^i} \cok f^i \rightarrow 0
    \]
    is exact.\\
    For the second exact sequence, we first want a monomorphism $H^i(A^\bullet)\hookrightarrow \cok f^{i-1}$. For notation, let $j^i:\ker f^i\hookrightarrow A^i$ be the canonical maps for each $i$. First, we obtain the following induced morphism $\varphi^i$ from the below commutative diagram:
    \begin{center}
        \begin{tikzcd}
            &&A^{i+1}\\
            &\cok f^{i-1} \ar[dashed]{ur}[description]{\exists!}\\
            A^{i-1} \ar{ur}{0} \ar{r}{f^{i-1}}& A^i \ar[bend right]{uur}[swap]{f^i} \ar[two heads]{u}[swap]{\pi^{i-1}}
        \end{tikzcd}
    \end{center}
    Using this factorization of $f^i=\varphi^i\circ \pi^{i-1}$, we obtain another induced morphism $\phi^i$ from the following commutative diagram:
    \begin{center}
        \begin{tikzcd}
            &&A^{i+1}\\
            &\ker f^i \ar[hook]{r}{j^i} \ar{ur}{0}& A^i \ar{u}{f^i}\\
            \im d^{i-1} \ar[hook, bend right]{urr}[swap]{\iota^{i-1}} \ar[dashed]{ur}[description]{\exists!}
        \end{tikzcd}
    \end{center}
    where $f^i\circ \iota^{i-1}=0$ because $f^i=\varphi^i\circ \pi^{i-1}$ so
    \[
    f^i\circ \iota^{i-1}=\varphi^i\circ \pi^{i-1}\circ \iota^{i-1}=\varphi^i\circ 0=0
    \]
    Then we define $H^i(A^\bullet)=\ker f^i/\im f^{i-1}$ as $\cok \phi^i$, and let $\sigma^i:\ker f^i \twoheadrightarrow H^i(A^\bullet)$ be the projection. Then we obtain one last induced morphism $\chi^i$ from the following commutative diagram:
    \begin{center}
        \begin{tikzcd}
            &&\cok f^{i-1}\\
            &H^i(A^\bullet) \ar[dashed]{ur}[description]{\exists!}&\\
            \im f^{i-1} \ar{ur}{0} \ar[hook]{r}{\phi^i} \ar[bend right, hook]{rr}{\iota^{i-1}}& \ker{f^i} \ar[hook]{r}{j^i} \ar[two heads]{u}{\sigma^i}&A^i \ar[two heads]{uu}[swap]{\pi^{i-1}}
        \end{tikzcd}
    \end{center}
    where $\pi^{i-1}\circ j^i\circ \phi^i=0$ because $j^i\circ \phi^i=\iota^{i-1}$ and $\pi^{i-1}\circ \iota^{i-1}=0$.
    We claim that $\chi^i$ is the desired monomorphism. By Lemma \ref{lem:comp with monic and ker}, $\ker (\pi^{i-1}\circ j^i)=\ker \pi^{i-1}=\im f^{i-1}$, hence by commutativity of the above diagram $\ker (\chi^i \circ \sigma^i)=\im f^{i-1}$. By Lemma \ref{lem:epic iff coim is target}, since $\sigma^i$ is epic and
    \[
    \ker(\chi^i\circ \sigma^i)=\im f^{i-1}=\ker \sigma^i
    \]
    we obtain that $\chi^i$ is monic as desired. Thus $0\to H^i(A^\bullet)\xrightarrow{\chi^i}\cok f^{i-1}$ is exact.\\
    For the map $\omega^i:\cok f^{i-1}\to \im f^i$, we will let it be the induced map from the following commutative diagram:
    \begin{center}
        \begin{tikzcd}
        &&\im f^i\\
                                            &\cok f^{i-1} \ar[dashed]{ur}[description]{\exists!} \\
            A^{i-1}\ar{r}{f^{i-1}}\ar{ur}{0}&A^i\ar[two heads]{u}[swap]{\pi^{i-1}} \ar[two heads, bend right]{uur}[swap]{\tilde f^{i}}
        \end{tikzcd}
    \end{center}
    It follows from Lemma \ref{lem:comp epic then epic} that $\omega^i$ is epic, so that $\cok f^{i-1}\xrightarrow{\omega^i}\im f^i\rightarrow 0$ is exact.\\
    The last thing to show is that $\ker \omega^i=\im \chi^i$, which we can do by showing that $\cok \chi^i=\omega^i$. Because $\chi^i$ is monic, we have 
    \begin{align*}
        &\cok \chi^i=\cok f^{i-1}/H^i(A^\bullet)=\cok f^{i-1}/(\ker f^i/\im f^{i-1})\\
        &=(A^i/\im f^{i-1})/(\ker f^i/ \im f^{i-1})
    \end{align*}
    
    
    By Theorem \ref{thm:3IT} (or the 3IT) , we get that
    \[
    (A^i/\im f^{i-1})/(\ker f^i/ \im f^{i-1})=A^i/\ker f^i
    \]
    By the 1IT \ref{thm:1IT}, we have that
    \[
    A^i/\ker f^i=\im f^i
    \]
    which shows $\cok \chi^i=\im f^i$ as desired. Therefore $H^i(A^\bullet) \xrightarrow{\chi^i} \cok f^{i-1} \xrightarrow{\omega^i}$ is also exact, proving the following is exact:
    \[
    0\rightarrow H^i(A^\bullet)\xrightarrow{\chi^i}\cok f^{i-1} \xrightarrow{\omega^i} \im f^i \rightarrow 0
    \]
\end{proof}
\subsubsection{B}\label{1.6.B}
\begin{proof}
    Because $H^i(A^\bullet)=\ker d^i/\im d^{i-1}$, we get that $h^i(A^\bullet)=\dim(\ker d^i)-\dim (\im d^{i-1})$ by basic linear algebra. The rank-nullity theorem also gives us that
    \[
    \dim(\im d^i)+\dim(\ker d^i)=\dim A^i
    \]
    Therefore
    \[
    \sum (-1)^i \dim A^i=\sum (-1)^i [\dim(\im d^i)+\dim \ker(d^i)]
    \]
   We claim that the index $i$ is even if and only if $\dim(\ker d^i)$ and $\dim(\im d^i)$ have a positive sign in $\sum (-1)^i h^i(A^\bullet)$. For the $\dim(\ker d^i)$ term, this is immediate. We also notice that the $\dim(\im d^i)$ term actually comes from $h^{i+1}(A^\bullet)$, which has a factor of $(-1)^{i+1}=-1$, so that 
   \[
   (-1)^{i+1} h^{i+1}(A^\bullet)=-(\dim \ker d^{i+1}- \dim \im d^i)=\dim \im d^i-\dim \ker d^{i+1}
   \]
   so indeed the sign of the $\dim \im d^i$ is positive whenever $i$ is even. A very similar proof shows that the index $i$ is odd if and only if $\dim(\ker d^i)$ and $\dim(\im d^i)$ have a negative sign in $\sum (-1)^i h^i(A^\bullet)$. It follows that
   \[
   \sum (-1)^i \dim \im d^i=\sum (-1)^i h^i(A^\bullet)
   \]
   When $A^\bullet$ is exact, then $\ker d^i=\im d^{i-1}$ for every $i$, so in particular $\dim \ker d^i-\dim \im d^{i-1}=0$ for every $i$. By the main result, we get that
   \[
   \sum (-1)^i \dim A^i=\sum (-1)^i h^i(A^\bullet)=0
   \]
\end{proof}
\subsubsection{C}\label{1.6.C}
\begin{proof}
    We can define the addition structure of $\Hom(A^\bullet, B^\bullet)$ as $(\alpha+\beta)^i=\alpha^i+\beta^i$ for each $i$ where $\alpha,\beta\in \Mor(A^\bullet,B^\bullet)$. This gives abelian group structure to $\Hom(A^\bullet,B^\bullet)$ because for each $i$, addition commutes, associativity holds, and inverses and identities exist. This defines a morphism in $\Com_\fC$ because if for each $i$
    \[
    \alpha^{i+1}\circ f^i=g^i\circ \alpha^i
    \]
    and something similar for $\beta$, then
    \[
    (\alpha^{i+1}+\beta^{i+1})\circ f^i=\alpha^{i+1}\circ f^i+\beta^{i+1}\circ f^i=g^i\circ \alpha^i+g^i\circ \beta^i=g^i\circ (\alpha^i +\beta^i)
    \]
    because $\fC$ is an abelian category so composition distributes over addition. This shows that the sum of morphisms in $\Com_\fC$ are indeed commutative diagrams. We also need to show that addition distributes over composition. If $\alpha,\beta :B^\bullet \to C^\bullet$ in $\Com_\fC$ and $f,g:A^\bullet \to B^\bullet$, then
    \[
    [\alpha \circ(f+g)]^i=\alpha^i\circ(f^i+g^i)=\alpha^i\circ f^i+\alpha^i\circ g^i=(\alpha \circ f)^i+(\alpha \circ g)^i
    \]
    as well as
    \[
    [(\alpha+\beta)\circ f]^i=(\alpha^i\circ \beta^i)\circ f^i=\alpha^i\circ f^i+\beta^i\circ f^i=(\alpha \circ f)^i+(\beta\circ f)^i
    \]
    again by Ad1. in $\fC$.
    This shows that Ad1. holds for $\Com_\fC$.\\
    We claim that the zero object $0$ in $\Com_\fC$ is the exact sequence
    \[
    \dots \rightarrow0\rightarrow0\rightarrow0\rightarrow\dots
    \]
    We can prove that $0$ is initial because if we fix any
    \[
    \dots \rightarrow A^{i-1} \rightarrow A^i \rightarrow A^{i+1}\rightarrow \dots \in \Com_\fC
    \]
    then 
    \begin{center}
        \begin{tikzcd}
            \dots \ar{r}& 0 \ar{d} \ar{r}&0 \ar{d} \ar{r}&0 \ar{d} \ar{r}&\dots\\
            \dots \ar{r}& A^{i-1} \ar{r}& A^i \ar{r}& A^{i+1} \ar{r}& \dots
        \end{tikzcd}
    \end{center}
    clearly commutes, and because each arrow $0\to A^i$ is unique, it proves there is a unique morphism in $\Com_\fC$ from $0\to A^\bullet$ so indeed $0$ is the initial object in $\Com_\fC$. A very similar argument shows that $0$ is final in $\Com_\fC$, hence $0$ is the zero object in $\Com_\fC$. This proves that Ad2. holds in $\Com_\fC$.\\
    We define the product $A^\bullet \times B^\bullet$ as the complex where 
    \[
    (A^\bullet \times B^\bullet)^i=A^i\times B^i
    \]
    and the morphism $A^i\times B^i\rightarrow A^{i+1}\times B^{i+1}$ is given by $(A^i\rightarrow A^{i+1})\times(B^i\rightarrow B^{i+1})$, which more precisely is the induces morphism in the following commutative diagram:
    \begin{center}
        \begin{tikzcd}
            &A^i\times B^i \ar[dashed]{d}[description]{\exists!} \ar[bend right]{ddl} \ar[bend left]{ddr}\\
            &A^{i+1}\times B^{i+1} \ar{ddr} \ar{ddl}\\
            A^{i}\ar{d}&&B^i\ar{d}\\
            A^{i+1} &&B^{i+1}
        \end{tikzcd}
    \end{center}
    and the projection $A^\bullet \times B^\bullet\rightarrow A^\bullet$ is the following commutative diagram:
    \begin{center}
        \begin{tikzcd}
            \dots \ar{r}& A^{i-1}\times B^{i-1} \ar{d} \ar{r}&A^i\times B^i \ar{d} \ar{r}&A^{i+1}\times B^{i+1} \ar{d} \ar{r}&\dots\\
            \dots \ar{r}& A^{i-1} \ar{r}& A^i \ar{r}& A^{i+1} \ar{r}& \dots
        \end{tikzcd}
    \end{center}*
    which commutes by definition--the projection to $B^\bullet$ is almost defined identically. It's easy to show that this is indeed the product in $\Com_\fC$. Therefore $\Com_\fC$ satisfies Ad3., so $\Com_\fC$ is additive.\\
    To show $\Com_\fC$ is abelian, we take any $f:A^\bullet \to B^\bullet$ and claim that $\ker f$ is the complex below:
    \begin{center}
        \begin{tikzcd}
            \dots \ar{r}& \ker f^{i-1} \ar{r}&\ker f^i \ar{r}& \ker f^{i+1} \ar{r}& \dots
        \end{tikzcd}
    \end{center}
    where for each $i$ the arrow $\ker f^{i-1}\to \ker f^i$ is the one induced in the following diagram:
    \begin{center}
        \begin{tikzcd}
            &&B^i\\
            &\ker f^i \ar[hook]{r} \ar{ur}{0}& A^i \ar{u}[swap]{f^i}\\
            \ker f^{i-1} \ar[dashed]{ur}[description]{\exists!} \ar[hook]{rr}&& A^{i-1} \ar{u}
        \end{tikzcd}
    \end{center}
    where if for each $i$ we let $g^i:A^i\to A^{i+1}$ be the morphism of the complex $A^\bullet$ and $h^i:B^i\to B^{i+1}$ be the morphisms in the complex $B^\bullet$ and let $\iota^i:\ker f^i\hookrightarrow A^i$ be the inclusion, we have by definition of $f$ being a morphism in $\Com_\fC$ that the following diagram commutes:
    \begin{center}
        \begin{tikzcd}
            A^{i-1} \ar{d}{f^{i-1}} \ar{r}{g^{i-1}} &A^i\ar{d}{f^i}\\
            B^{i-1} \ar{r}{h^{i-1}}& B^i
        \end{tikzcd}
    \end{center}
    Therefore
    \[
    f^i\circ g^{i-1}\circ \iota^{i-1}=h^{i-1}\circ f^{i-1} \circ \iota^{i-1}= h^{i-1}\circ 0=0
    \]
    proving that we do indeed get the desired induced morphisms. By construction, the following diagram also commutes:
    \begin{center}
        \begin{tikzcd}
            \dots \ar{r}&A^{i-1} \ar{r}{g^{i-1}}& A^i \ar{r}{g^i}& A^i \ar{r}& \dots\\
            \dots \ar{r}& \ker f^{i-1} \ar[hook]{u}{\iota^{i-1}} \ar{r}&\ker f^{i} \ar[hook]{u}{\iota^i} \ar{r}&\ker f^{i+1} \ar[hook]{u}{\iota^{i+1}} \ar{r}&\dots
        \end{tikzcd}
    \end{center}
    We can define cokernels dually, and to be precise we should prove that these satisfy the universal property we want them to, but we shall not for brevity. It's an easy exercise if you wish.\\
    This shows that kernels and cokernels exist, and $f$ is a monic if and only if each $f^i$ are monic, in which case we get by our constructions and the fact that $\fC$ is an abelian category that $\ker \cok f=f$. Similarly, $\cok \ker f=f$ whenever $f$ is epic. This shows that indeed $\Com_\fC$ is abelian.
\end{proof}
\subsubsection{D}\label{1.6.D}
\begin{proof}
    We will deal with the special case $\Mod_A$ for ease of proof, which suffices because of the Freyd-Mitchell Theorem although in general I try not to invoke this theorem. If $h\in \Hom(A^\bullet, B^\bullet)$, then we define a map $H^i(h):H^i(A^\bullet)\to H^i(B^\bullet)$ given by
    \[
    a+\im f^{i-1}\mapsto h^i(a) +\im g^{i-1}
    \]
    where $a\in \ker f^i$. Notice that if $a\in \ker f^i$, then $h^i(a)\in \ker g^i$ because 
    \[
    g^i\circ h(a)=h^{i+1}\circ f^i(a)=h^{i+1}(0)=0
    \]
    To show $H^i(h)$ is well defined, we need to show it's constant on representatives of $\im f^{i-1}$. To do this, fix any $a\in \im f^{i-1}\subset \ker f^{i}$ and let $f^{i-1}(x)=a$ for some $x\in A^{i-1}$. Then
    \[
    H^i(h)(a)=h^i(a)+\im g^{i-1}=h^i\circ f^{i-1}(x)+\im g^{i-1}=g^{i-1}\circ h^{i-1}(x)+\im g^{i-1}=\im g^{i-1}
    \]
    so indeed $H^i(h)$ is constant on $\im f^{i-1}$ so it is well defined.\\
    Then we can define $H^i:\Com_\fC \to \fC$ to be a functor. We can do this because $H^i(\id_{A^\bullet})$ acts on elements $a+\im f^{i-1}\in H^i(A^\bullet)$ as
    \[
    a+\im f^{i-1}\mapsto \id_{A^i}(a)+\im f^{i-1}=a+\im f^{i-1}
    \]
    which shows $H^i(\id_{A^\bullet})=\id_{H^i(A^\bullet)}$. If we're given $f:A^\bullet\to B^\bullet$ and $g:B^\bullet\to C^\bullet$ as morphisms in $\Com_\fC$, then for any $a\in \ker f^i$
    \begin{align*}
        H^i(g\circ f)([a])=[g\circ f(a)]=H^i(g)([f(a)])=H^i(g)\circ H^i(f)([a])
    \end{align*}
    Then indeed $H^i$ is a covariant functor.
\end{proof}
\subsubsection{E}\label{1.6.E}
\begin{proof}
    Let $f,g:C^\bullet \to D^\bullet$ be homotopic through maps $w:C^i\to D^{i-1}$, i.e. $f-g = dw+wd$. Fixing some index $i$, we have $f^i-g^i =d_D^{i-1} w^i + w^{i+1} d_C^i$. We quickly observe $d^{i-1} w^i$ induces the trivial map on homology since we mod out the image of $d^{i-1}$. In addition, $w^{i+1} d^i$ induces the trivial map on homology since $H^i(C^\bullet) = \ker d^i/\im d^{i-1}$, so applying $d^i$ kills anything in $H^i(C^\bullet)$. Then $f^i-g^i$ induces the trivial map on homology, i.e. $H^i(f) = H^i(g)$.
\end{proof}
\subsubsection{F}\label{1.6.F}
\begin{proof}
    Suppose $A' \xrightarrow{f} A\xrightarrow{g}A''$ is exact. If $F:\fA\to \fB$ is covariant, By Lemmas \ref{lem:covariant exact and commutes with im and coim} and \ref{lem:covariant left exact commutes with ker} we have
    \begin{align*}
        \im Ff=F\im f=F\ker g=\ker Fg
    \end{align*}
    Therefore $F(A')\xrightarrow{Ff} F(A) \xrightarrow{Fg}F(A'')$ is exact as desired. \\
    If $F:\fA\to \fB$ is contravariant, By Lemmas \ref{cor:contravariant left exact and cok to ker} and \ref{lem:contravariant exact and im to coim and coim to im} we get that 
    \begin{align*}
        &\im Fg=F\coim g=F\cok \ker g=F\cok \im f=\ker F\im f=\ker \coim Ff=\ker Ff
    \end{align*}
   
\end{proof}
\subsubsection{G}\label{1.6.G}
\begin{proof}
    \begin{enumerate}[(a)]
        \item To show the localization functor $L:\Mod_A\to \Mod_{S^{-1}A}$ is left exact, suppose $0\rightarrow M' \xrightarrow{f}M \xrightarrow{g}M''$ is exact. Then
        \[
        0\rightarrow S^{-1}M'\xrightarrow{Lf} S^{-1}M
        \]
        is exact because we know $f$ is injective, which we can use to demonstrate $Lf$ is injective as follows:
        \begin{align*}
            Lf(\frac{m'}{s})=0\iff \frac{f(m')}{s}=0
        \end{align*}
        if and only if there exists some $t\in S$ such that $tf(m')=0$. But because $f$ is $A-$linear, we notice
        \[
        tf(m')=f(tm')
        \]
        Therefore $tf(m')=0$ if and only if $f(tm')=0$, and now using the fact that $f$ is injective, we get that $tm'=0$, which proves that indeed
        \begin{align*}
            \frac{m'}{s}=0
        \end{align*}
        Therefore $Lf$ is injective as desired. To show $\ker Lg=\im Lf$, fix any $\frac{m}{s}\in \im Lf$ and let $Lf(\frac{m'}{s'})=\frac{f(m')}{s'}=\frac{m}{s}$. We want to show that $Lg(\frac{m}{s})=0$. To do this, we observe
        \begin{align*}
            Lg(\frac{m}{s})=Lg(\frac{f(m')}{s'})=\frac{g\circ f(m')}{s'}=\frac{0}{s'}=0
        \end{align*}
        because $\im f=\ker g$. This shows $\im Lf \subset \ker Lg$.\\
        For the reverse inclusion, suppose $Lg(\frac{m}{s})=\frac{g(m)}{s}=0$. Then there exists some $r\in S$ such that
        \[
        rg(m)=0
        \]
        If $g(m)\ne 0$, then by $A$-linearity of $g$ we get
        \[
        g(rm)=0\Rightarrow rm\in \ker g
        \]
        Because $\ker g=\im f$, let $f(m')=rm$. Then
        \begin{align*}
            Lf(\frac{m'}{rs})=\frac{f(m')}{rs}=\frac{rm}{rs}=\frac{m}{s}
        \end{align*}
        so indeed $\frac{m}{s}\in \im Lf$. Therefore $\ker Lg\subset \im Lf$, proving the following sequence is left exact:
        \[
        0\to S^{-1}M'\xrightarrow{Lf} S^{-1}M\xrightarrow{Lg} S^{-1}M''
        \]
        To show $L$ is right exact, suppose
        \[
        M'\xrightarrow{f}M \xrightarrow{g}M'' \rightarrow 0
        \]
        is exact. By the second argument in the proof that $L$ is left exact, we get that $M'\rightarrow M\rightarrow M''$ is exact. The last thing to show is that $Lg$ is surjective given $g$ is. To do this, fix any $\frac{m''}{s}\in S^{-1}M''$. Because $m''\in M''$ and $g$ is surjective, there exists some $m\in M$ such that $g(m)=m''$. Therefore
    \[
    Lg(\frac{m}{s})=\frac{g(m)}{s}=\frac{m''}{s}
    \]
    which shows $L$ is right exact.
    \item 
    Check the solution to Exercise \ref{1.3.H}H.
    \item 
    Suppose $0\rightarrow M' \xrightarrow{f}M\xrightarrow{g}M''$ is exact. To show $f_*$ is injective, suppose $f_*(h)=0$ where $h\in \Hom(C,M')$. By definition, then $f\circ h=0$. By Lemma \ref{lem:comp with monic and ker} $\ker(f\circ h)=\ker h$
    \[
    M=\ker 0=\ker (f\circ h)=\ker h
    \]
    which implies that $h=0$. Therefore $f_*$ is indeed injective.\\
    Now suppose $h\in \ker g_*$ or equivalently $g\circ h=0$. Then we get the following induced morphism $h'$:
    \begin{center}
        \begin{tikzcd}
            &&M''\\
            &\ker g \ar[hook]{r}{f} \ar{ur}{0}&M\ar{u}{g}\\
            C\ar[dashed]{ur}[description]{\exists!} \ar[bend right]{urr}{h}
        \end{tikzcd}
    \end{center}
    because $f$ monic implies $\im f=M'$ and we know $\ker g=\im f$.
    Therefore $h=f\circ h'$, or equivalently that $f_*(h')=h$ proving that $\im f_*\subset \ker g_*$.\\
    On the other hand, if $h\in \im f_*$, then let $h=f_*(h')$ or equivalently $h=f\circ h'$. Then clearly $h\in \ker g_*$ because
    \[
    g_*(h)=g\circ h=g\circ f\circ h'=0\circ h'=0
    \]
    since $\im f=\ker g$. This shows $\im f_*=\ker g_*$, which proves the following is exact:
    \begin{center}
        \begin{tikzcd}
            0\ar{r}& \Hom(C,M') \ar{r}{f_*}& \Hom(C,M) \ar{r}{g_*}& \Hom(C,M'')
        \end{tikzcd}
    \end{center}
    If $\fC$ is an abelian category, each hom-set is an abelian group, hence $\Hom(C,\cdot)$ defines a left exact covariant functor into $\Ab$.
    \item 
    Suppose $A\xrightarrow{f}B\xrightarrow{g}C\rightarrow 0$ is exact. To show $0\rightarrow \Hom(C,M) \xrightarrow{g^*} \Hom(B,M)$ is exact, we want to show that $g^*$ has a trivial kernel. If $g^*(h)=0$, then $h\circ g=0$ But because $C=\im g\subset \ker h$, then $C=\ker h$ so $h=0$ so indeed $g^*$ has a trivial kernel.\\
    To show $\Hom(C,M)\xrightarrow{g^*} \Hom(B,M) \xrightarrow{f^*} \Hom(A,M)$ is exact, fix any $h\in \ker f^*$. Then $h\circ f=0$, and $\im f=\ker g$ implies 
    \[
    \cok f=B/\im f=B/\ker g=\coim g=C
    \]
    so we get the following commutative diagram:
    \begin{center}
        \begin{tikzcd}
            &&M\\
            & C \ar[dashed]{ur}[description]{\exists!}\\
            A \ar{ur}{0} \ar{r}{f}& B \ar[two heads]{u}{g} \ar[bend right]{uur}[swap]{h}
        \end{tikzcd}
    \end{center}
    If $h':C\to M$ is the induced morphism, we get $h=h'\circ g=g^*(h')$ so indeed $h\in \im g^*$.\\
    On the other hand, if $h\in \im g^*$, let $h=g^*(h')=h'\circ g$. It's clear then that
    \[
    f^*(h)=h\circ f=h'\circ g\circ f=h'\circ 0=0
    \]
    so then $h\in \ker f^*$, which proves along with the previous result that $\im f^*=\ker g^*$. Then indeed the following sequence is exact:
    \[
    0\rightarrow \Hom(C,M) \xrightarrow{g^*} \Hom(B,M) \xrightarrow{f^*} \Hom(A,M)
    \]
    \end{enumerate}
    
\end{proof}
\subsubsection{H}\label{1.6.H}
\begin{proof}
    By the previous exercise we have that $\Hom(\cdot,N)$ is left exact and that the localization $L:\Mod_A \to \Mod_{S^{-1}}A$ is exact. Therefore on one hand we have
    \[
    A^{\oplus q} \xrightarrow{f} A^{\oplus p}\xrightarrow{g} M\rightarrow 0
    \]
    exact implies
    \[
    0\rightarrow \Hom_A(M,N)\xrightarrow{g^*} \Hom_A(A^{\oplus p},N)\xrightarrow{f^*} \Hom_A(A^{\oplus q},N)
    \]
    is exact and so by left exactness of $L$ we get that
    \[
    0\rightarrow S^{-1}\Hom_A(M,N)\xrightarrow{L(g^*)} S^{-1}\Hom_A(A^{\oplus p},N) \xrightarrow{L(f^*)} S^{-1}\Hom_A(A^{\oplus q},N)
    \]
    is exact. On the other hand by right exactness of $L$, we have
    \[
    S^{-1}A^{\oplus q} \xrightarrow{Lf} S^{-1}A^{\oplus p} \xrightarrow{Lg} S^{-1}M \rightarrow0
    \]
    is exact so by left exactness of $\Hom$ we get
    \begin{align*}
        0\rightarrow \Hom_{S^{-1}A}(S^{-1}M,S^{-1}N) \xrightarrow{Lg^*} \Hom_{S^{-1}A}(S^{-1}A^{\oplus p},S^{-1}N)\xrightarrow{Lf^*} \Hom_{S^{-1}A}(S^{-1}A^{\oplus q},S^{-1}N)
    \end{align*}
    is exact.\\
    If $\frac{h}{s}\in \ker L(f^*)$ where $h\in \Hom_A(A^{\oplus p},N)$, then
    \begin{align*}
        0=L(f^*)(\frac{h}{s})=\frac{f^*(h)}{s}=\frac{h\circ f}{s}
    \end{align*}
    Additionally, we notice that because
    \begin{align*}
        \frac{h}{s}(\frac{x}{s'})=\frac{h(x)}{ss'}
    \end{align*}
    and
    \begin{align*}
        Lf(\frac{y}{s'})=\frac{f(y)}{s'}
    \end{align*}
    it follows that
    \begin{align*}
        Lf^*(\frac{h}{s})=\frac{h}{s}\circ Lf=\frac{h\circ f}{s}
    \end{align*}
    Then indeed $\ker L(f^*)\subset \ker Lf^*$.\\
    If now $Lf^*(h)=0$ where $h\in \Hom_{S^{-1}A}(S^{-1}A^{\oplus p},S^{-1}N)$,
    \begin{align*}
        0=Lf^*(h)=h\circ Lf
    \end{align*}
    and
    \begin{align*}
        L(f^*)(h)=L(f^*)(\frac{h}{1})=\frac{f^*(h)}{1}=\frac{h\circ f}{1}
    \end{align*}
    It's an easy exercise to verify that by $S^{-1}A$ linearity of $h$, indeed 
    \[
    h\circ Lf(\frac{x}{s})=\frac{h\circ f(x)}{s}=\frac{h\circ f}{1}(\frac{x}{s})
    \]
    for arbitrary $\frac{x}{s}\in S^{-1}A^{\oplus q}$. This proves that $\ker Lf^*\subset \ker L(f^*)$, so
    \[
    \ker Lf^*=\ker L(f^*)
    \]
    Therefore by exactness of the two sequence and by Lemma \ref{lem:monic iff im is source} applied to $L(g^*)$ and $Lg^*$,
    \begin{align*}
        \Hom_{S^{-1}A}(S^{-1}M,S^{-1}N)=\im Lg^*=\ker Lf^*=\ker L(f^*)=\im L(g^*)=S^{-1}\Hom_A(M,N)
    \end{align*}
\end{proof}
\subsubsection{I}\label{1.6.I}
\begin{proof}
    For this proof, we will use notation from Exercise \ref{1.6.A}A for the canonical and induced maps. We will also use the notation that if $f:A\to B$ is a morphism, then $f\vert^{\im} :A\to \im f$ is the induced morphism.
    \begin{enumerate}[(a)]
        \item By Exercise \ref{1.6.A}A, the following sequence is exact:
    \begin{align*}
        0\rightarrow H^i(C^\bullet) \xrightarrow{\chi^i}\cok d^{i-1} \xrightarrow{\omega^i}\im d^i\rightarrow 0
    \end{align*}
        By right exactness of $F$, the following is exact:
        \begin{align*}
            FH^i(C^\bullet) \xrightarrow{F\chi^i} F\cok d^{i-1} \xrightarrow{F \omega^i} F\im d^i \rightarrow 0
        \end{align*}
        By Lemma \ref{lem:covariant right exact commutes with cok}, we have that $F\cok d^{i-1}=\cok Fd^{i-1}$. By Exercise \ref{1.6.A}A again, the following sequence is exact as well:
        \begin{align*}
            0\rightarrow H^i(FC^\bullet) \xrightarrow{\chi}\cok Fd^{i-1} \xrightarrow{\omega}\im Fd^i\rightarrow 0
        \end{align*}
        By Exercise \ref{1.6.A}A again, we have
        \begin{align*}
            0\rightarrow \im d^i \xrightarrow{\iota^i} C^{i+1} \xrightarrow{\pi^i} \cok d^i \rightarrow 0
        \end{align*}
        is exact, so by right exactness of $F$, the following is also exact:
        \begin{align*}
            F\im d^i \xrightarrow{F\iota^{i}} FC^{i+1} \xrightarrow{F\pi^i} F\cok d^i \rightarrow 0
        \end{align*}
    We claim that the following diagram commutes:
    \begin{center}
        \begin{tikzcd}
            F\cok d^{i-1} \ar{r} {F\omega^i} \ar{d}{=}& F\im d^i \ar{d}{F\iota^i\vert^{\im}}\\
            \cok Fd^{i-1} \ar{r}{\omega}& \im Fd^i
        \end{tikzcd}
    \end{center}
   To show this, we just need to recall the definitions of our morphisms. Firstly, we have the following commutative diagram:
   \begin{center}
       \begin{tikzcd}
           &&F\im d^i \ar{r}{F\iota^i}& FC^{i+1}\\
           &F\cok d^{i-1} \ar[two heads]{ur}{F\omega^i}\\
           FC^{i-1} \ar{ur}{0}\ar{r}{Fd^{i-1}}& FC^i \ar[two heads]{u}{F\pi^{i-1}} \ar[bend right, two heads]{uur}[swap]{F(d^i\vert^{\im})} \ar[bend right=45]{uurr}[swap]{Fd^i}
       \end{tikzcd}
   \end{center}
   as well as the commutative diagram below:
   \begin{center}
       \begin{tikzcd}
           &&\im Fd^i \ar[hook]{r}{\iota}& FC^{i+1}\\
           &\cok Fd^{i-1} \ar[two heads]{ur}{\omega}\\
           FC^{i-1} \ar{ur}{0}\ar{r}{Fd^{i-1}}& FC^i \ar[two heads]{u}{F\pi^{i-1}} \ar[bend right, two heads]{uur}[swap]{(Fd^i)\vert^{\im})} \ar[bend right=45]{uurr}[swap]{Fd^i}
       \end{tikzcd}
   \end{center}
   We observe that by commutativity of the two diagrams,
   \begin{align*}
       \iota \circ \omega \circ F\pi^{i-1}=\iota \circ (Fd^i)\vert^{\im}=Fd^i=F\iota^i\circ F(d^i\vert^{\im})=F\iota^i\circ F\omega^i\circ F\pi^{i-1}
   \end{align*}
   Because $F\pi^{i-1}$ is epic, we obtain the following commutative diagram:
   \begin{center}
       \begin{tikzcd}
           F\im d^i \ar{r}{F\iota^i}& FC^{i+1}\\
           \cok Fd^{i-1} \ar[two heads]{u}{F\omega^i} \ar[two heads]{r}{\omega}& \im Fd^i \ar[hook]{u}{\iota}
       \end{tikzcd}
   \end{center}
   Using Lemma \ref{lem:covariant right exact commutes with cok} and the exactness of our sequences, we see
\begin{align*}
    \im F\iota^i=\ker F\pi^i=\ker F\cok d^i=\ker \cok F d^i=\im Fd^i
\end{align*}
   Thus the following diagram commutes:
   \begin{center}
       \begin{tikzcd}
           F\im d^i \ar{r}{F\iota^i} \ar[two heads]{dr}{F\iota^i \vert^{\im}}& FC^{i+1}\\
           \cok Fd^{i-1} \ar[two heads]{u}{F\omega^i} \ar[two heads]{r}{\omega}& \im Fd^i \ar[hook]{u}{\iota}
       \end{tikzcd}
   \end{center}
   Then by Lemma \ref{lem:ker of comp into ker of first}, we get the desired canonical inclusion $\theta:\ker F\omega^i \hookrightarrow \ker \omega$
    which shows
    \begin{align*}
        FH^i(C^\bullet)\xtwoheadrightarrow{F\chi^i\vert^{\im}} \im F\chi^i=\ker F\omega^i \xhookrightarrow{\theta} \ker \omega=\im \chi=H^i(FC^\bullet)
    \end{align*}
    \item 
    By Vakil (1.6.5.3), we have the short exact sequences below:
    \begin{align*}
        0\rightarrow \im d^{i-1}\xrightarrow{j^i} \ker d^i \xrightarrow{\phi^i} H^i(C^\bullet)\rightarrow0\\
        0\rightarrow \im Fd^{i-1}\xrightarrow{j} \ker Fd^i \xrightarrow{\phi} H^i(FC^\bullet)\rightarrow0
    \end{align*}
    By left exactness of $F$, we get from the first sequence the following exact sequence:
    \begin{align*}
        0\rightarrow F\im d^{i-1}\xrightarrow{Fj^i} F\ker d^i \xrightarrow{F\phi^i} FH^i(C^\bullet)
    \end{align*}
    and by Lemma \ref{lem:covariant left exact commutes with ker} $F\ker d^i=\ker Fd^i$. We now observe the following commutative diagram:
    \begin{center}
        \begin{tikzcd}
            && F\cok d^{i-1}\\
            &\cok Fd^{i-1} \ar[dashed]{ur}[description]{\exists!}\\
            FC^{i-1} \ar{ur}{0} \ar{r}{Fd^{i-1}}&FC^i \ar[two heads]{u}{\pi} \ar[bend right]{uur}{F\pi^{i-1}}
        \end{tikzcd}
    \end{center}
    By Lemma \ref{lem:covariant left exact commutes with ker}, we have $\ker F\pi^{i-1}= F\ker \pi^{i-1}=F\im d^{i-1}$. By Lemma \ref{lem:ker of comp into ker of first} using the factorization of $F\pi^{i-1}$ through $\pi$, we get the following commutative diagram:
    \begin{center}
        \begin{tikzcd}
            &&F\cok d^{i-1}\\
            &\ker F\pi^{i-1} \ar{ur}{0} \ar[hook]{r}{F \iota^{i-1}}&FC^i\ar{u}[swap]{F\pi^{i-1}}\\
            \ker \pi \ar[dashed]{ur}[description]{\exists!} \ar[hook,bend right]{urr}{\iota}
        \end{tikzcd}
    \end{center}
    Letting $\alpha:\im Fd^{i-1}\to F\im d^{i-1}$ be the induced monomorphism in the diagram above, we now claim that $j=Fj^i\circ \alpha$. To see this, we need to recall how we obtained $j^i$ (a nearly identical idea is used to define $j$):
    \begin{center}
        \begin{tikzcd}
            &&C^{i+1}\\
            &\cok d^{i-1}\ar[dashed]{ur}[description]{\varphi^i}\\
            C^{i-1} \ar{ur}{0} \ar{r}{d^{i-1}}& C^i \ar[two heads]{u}{\pi^{i-1}} \ar[bend right]{uur}{d^i}
        \end{tikzcd}
    \end{center}
    implies that the following diagram commutes as well
    \begin{center}
        \begin{tikzcd}
            &&C^{i+1}\\
            &\ker d^i \ar[hook]{r}{\kappa^i} \ar{ur}{0}& C^i \ar{u}{d^i}\\
            \im d^{i-1} \ar[dashed]{ur}[description]{\exists! j^i} \ar[hook, bend right]{urr}{\iota^{i-1}}
        \end{tikzcd}
    \end{center}
    So we get that $\kappa^i\circ j^i=\iota^{i-1}$ and similarly $\kappa \circ j=\iota$. Then we have
    \begin{align*}
        \iota=F\iota^{i-1}\circ \alpha =F\kappa^i\circ Fj^i \circ \alpha
    \end{align*}
    as well as
    \begin{align*}
        \iota=\kappa \circ j
    \end{align*}
    Recalling that by Lemma \ref{lem:covariant left exact commutes with ker}, $F\ker d^i=\ker Fd^i$ implies that $\kappa=F\kappa^i$. Therefore
    \begin{align*}
        \kappa \circ j=\kappa \circ Fj^i \circ \alpha
    \end{align*}
    implies, because $\kappa$ is monic, that indeed $j=Fj^i\circ \alpha$. Thus we have the following commutative diagram that is exact across rows:
    \begin{center}
        \begin{tikzcd}
            0\ar{r}& F\im d^{i-1} \ar{r}{Fj^i}& F\ker d^i \ar{r}{F\phi^i}& FH^i(C^\bullet)\\
            0\ar{r}& \im Fd^{i-1} \ar{r}{j} \ar[hook]{u}{\alpha}& \ker Fd^i \ar[<->]{u}{=} \ar{r}{\phi}& H^i(FC^\bullet)\ar{r}& 0
        \end{tikzcd}
    \end{center}
    We have two last morphisms to construct, and composing them will be the desired morphism. The first is the induced morphism $\varphi:\cok Fj^i\to F\cok j^i$ from the following commutative diagram:
        \begin{center}
        \begin{tikzcd}
            &&F\cok j^i\\
            &\cok Fj^i \ar[dashed]{ur}[description]{\exists! \varphi}\\
            F\im d^{i-1} \ar[hook]{r}{Fj^i} \ar{ur}{0}& \ker Fd^i \ar[two heads]{u}{\sigma} \ar[bend right]{uur}{F\phi^i}
        \end{tikzcd}
    \end{center}
    The second comes from Lemma \ref{lem:cok of comp onto cok of second}, where we get an epimorphism $\lambda:\cok j \twoheadrightarrow \cok Fj^i$ such that $\sigma=\lambda \circ \phi$. Then we have
    \begin{align*}
        \cok j=H^i(FC^\bullet)\xtwoheadrightarrow{\lambda} \cok Fj^i \xrightarrow{\varphi} F\cok j^i=FH^i(C^\bullet)
    \end{align*}
    \item 
    If $F$ is exact, then $\omega=F\omega^i$ because by Lemma \ref{lem:covariant exact and commutes with im and coim}. Therefore $\theta$, the canonical inclusion from $\ker F\omega^i\hookrightarrow \ker \omega$ is actually just $\id_{\ker F\omega^i}$. Additionally, $F\chi^i\vert^{\im} =\id_{FH^i(C^\bullet)}$ because by left-exactness of $F$, $F\chi^i=\ker F\omega^i$.\\
    Additionally, by right exactness of $F$, $F\phi^i=\phi$. By our constructions, we have
    \[
    F\phi^i=\varphi\circ \sigma
    \]
    and
    \[
    \sigma=\lambda \circ \phi
    \]
    Therefore
    \[
    \phi=\varphi \circ \sigma=\varphi\circ \lambda \circ \phi
    \]
    which by the fact that $\phi$ is an epimorphism shows $\varphi \circ \lambda=\id_{H^i(FC^\bullet)}$. This proves that indeed $\varphi \circ \lambda$ and $\theta \circ F\chi^i\vert^{\im}$ are inverses of each other because they are both the identity on $H^i(FC^\bullet)=FH^i(C^\bullet)$. Though it may feel strange that all of our maps just turned into the identity but they were originally going from one object to another, it's because each of the objects satisfies the definition of the other when $F$ is exact, so they are the same object.
    \end{enumerate}
\end{proof}
\subsubsection{J}\label{1.6.J}
\begin{proof}
    This exercise is a special case of the below exercise because kernels are limits of the following diagram $\fJ$:
    \begin{center}
        \begin{tikzcd}
            &\bullet \ar{d}\\
            \bullet \ar{r}&\bullet
        \end{tikzcd}
    \end{center}
where $F:\fI\times \fJ\to \fC$ is the product functor of the functors $\fI \to \fC$ and $\fJ\to \fC$ and where $h=F(\id, g)$ where $g$ is the arrow on the bottom of $\fJ$.
\end{proof}
\subsubsection{K}\label{1.6.K}
\begin{proof}
    Let $F:\fI\times \fJ\to \fC$ be a covariant functor so that $\fI\times \fJ$, the product category of $\fI$ and $\fJ$, will be the index category of the desired limits. For the rest of the proof, let $i,i'\in \fI$, $j,j'\in \fJ$, $f:i\to i'$ and $g:j\to j'$ be arbitrary. To begin with, we notice we get a natural transformation $F(f,\id):F(i,\cdot)\to F(i',\cdot)$ demonstrated below:
    \begin{center}
        \begin{tikzcd}
            F(i,j)\ar{r}{F(\id_i, g)}\ar{d}[swap]{F(f,\id_j)}& F(i,j') \ar{d}{F(f,\id_{j'})}\\
            F(i',j) \ar{r}{F(\id_{i'},g)}& F(i',j')
        \end{tikzcd}
    \end{center}
    which commutes essentially by definition of the product category. Then we obtain an induced morphism $\tilde f:\lim_j F(i,j)\to \lim_j F(i',j)$ from the commutative diagram below:
    \begin{center}
        \begin{tikzcd}
            &&\lim_j F(i,j) \ar[swap]{ddll}{p_{ij}} \ar{ddrr}{p_{ij'}} \ar[dashed]{dd}[description]{\exists!}\\
            \\
            F(i,j) \arrow[bend left=10, crossing over, pos=0.34]{rrrr}[description]{F(\id_i,g)} \ar[swap]{dr}{F(f,\id_j)}&&\lim_j F(i',j) \ar[swap]{dl}{p_{i'j}} \ar{dr}{p_{i'j'}}&& F(i,j') \ar{dl}{F(f,\id_{j'})}\\
            &F(i',j)\ar{rr}{F(\id_{i'},g)}&&F(i',j')
        \end{tikzcd}
    \end{center}
    With these induced morphisms $\Tilde{f}$, we can actually index the $\lim_j F(i,j)$'s by $I$. This gives us our definition of $\lim_i \lim_j F(i,j)$ below:
    \begin{center}
        \begin{tikzcd}
            &\lim_i \lim_j F(i,j) \ar{dl}[swap]{\tau_i} \ar{dr}{\tau_{i'}}\\
            \lim_j F(i,j) \ar{rr}{\tilde f}& &\lim_j F(i',j)
        \end{tikzcd}
    \end{center}
    Similar to above, we get an induced $\tilde g:\lim_i F(i,j)\to \lim_i(F(i,j')$ from the following commutative diagram for each $g:j\to j'$:
    \begin{center}
        \begin{tikzcd}
            &&\lim_i F(i,j) \ar[swap]{ddll}{q_{ij}} \ar{ddrr}{q_{i'j}} \ar[dashed]{dd}[description]{\exists!}\\
            \\
            F(i,j) \arrow[bend left=10, crossing over, pos=0.34]{rrrr}[description]{F(f,\id_j)} \ar[swap]{dr}{F(\id_i,g)}&&\lim_i F(i,j') \ar[swap]{dl}{q_{ij'}} \ar{dr}{q_{i'j'}}&& F(i',j) \ar{dl}{F(\id_{i'},g)}\\
            &F(i,j')\ar{rr}{F(f,\id_{j'})}&&F(i',j')
        \end{tikzcd}
    \end{center}
    Similarly, we can index the $\lim_i F(i,j)$'s by $J$ with these induced $\tilde g$'s, so we also obtain the following construction for $\lim_j \lim_i F(i,j)$ below:
    \begin{center}
        \begin{tikzcd}
            &\lim_j \lim_i F(i,j) \ar{dl}[swap]{\sigma_j} \ar{dr}{\sigma_{j'}}\\
            \lim_i F(i,j) \ar{rr}{\tilde g}& &\lim_i F(i,j')
        \end{tikzcd}
    \end{center}
    We also observe the following diagram
   \begin{center}
        \begin{tikzcd}
            &&\lim_i\lim_j F(i,j) \ar[swap]{ddll}{\tau_i} \ar{ddrr}{\tau_{i'}} \ar[dashed]{dd}[description]{\exists!}\\
            \\
            \lim_j F(i,j) \arrow[bend left=10, crossing over, pos=0.34]{rrrr}[description]{\tilde f} \ar[swap]{dr}{p_{ij}}&&\lim_i F(i,j) \ar[swap]{dl}{q_{ij}} \ar{dr}{q_{i'j}}&& \lim_jF(i',j) \ar{dl}{p_{i'j}}\\
            &F(i,j)\ar{rr}{F(f,\id_{j})}&&F(i',j)
        \end{tikzcd}
    \end{center}
    which commutes because
    \begin{align*}
        F(f,\id_j)\circ p_{ij}\circ \tau_i=p_{i'j}\circ \tilde f\circ \tau_i=p_{i'j}\circ \tau_{i'}
    \end{align*}
    Let $\varphi_j:\lim_i\lim_j F(i,j)\to \lim_i F(i,j)$ be the induced morphism above. Now we want to show that $\tilde g\circ \varphi_j=\varphi_{j'}$. To do this, consider the following diagram:
    \begin{center}
        \begin{tikzcd}
            &&\lim_i\lim_j F(i,j) \ar[swap]{ddll}{\tau_i} \ar{ddrr}{\tau_{i'}} \ar{dd}[description]{\tilde g\circ \varphi_j}\\
            \\
            \lim_j F(i,j) \ar[swap]{dr}{p_{ij'}}&&\lim_i F(i,j') \ar[swap]{dl}{q_{ij'}} \ar{dr}{q_{i'j'}}&& \lim_jF(i',j) \ar{dl}{p_{i'j'}}\\
            &F(i,j')\ar{rr}{F(f,\id_{j'})}&&F(i',j')
        \end{tikzcd}
    \end{center}
    which commutes because
    \begin{align*}
        q_{ij'}\circ \tilde g\circ \varphi_j=F(\id_{i},g)\circ q_{ij}\circ \varphi_j=F(\id_i,g)\circ p_{ij}\circ \tau_i=p_{ij'}\circ \tau_i
    \end{align*}
    as well as
    \begin{align*}
        q_{i'j'}\circ \tilde g\circ \varphi_j=F(\id_{i'},g)\circ q_{i'j}\circ \varphi_j=F(\id_{i'},g)\circ p_{i'j}\circ \tau_{i'}=p_{i'j'}\circ \tau_{i'}
    \end{align*}
    Because the diagram commutes, by uniqueness of $\varphi_{j'}$ we get that indeed $\varphi_{j'}=\tilde g\circ \varphi_j$. We claim now that $\lim_i \lim_j F(i,j)$ together with our morphisms $\varphi_j$ are universal with respect to this diagram. To prove this, suppose we have the following commutative diagram:
    \begin{center}
        \begin{tikzcd}
            &W \ar{dl}[swap]{\chi_j} \ar{dr}{\chi_{j'}}\\
            \lim_i F(i,j) \ar{rr}{\tilde g}& &\lim_i F(i,j')
        \end{tikzcd}
    \end{center}
    Then by the below commutative diagram, we get an induced $\mu_i:W\to \lim_j F(i,j)$:
    \begin{center}
        \begin{tikzcd}
            &&W \ar[swap]{ddll}{\chi_j} \ar{ddrr}{\chi_{j'}} \ar[dashed]{dd}[description]{\exists!}\\
            \\
            \lim_i F(i,j) \arrow[bend left=10, crossing over, pos=0.34]{rrrr}[description]{\tilde g} \ar[swap]{dr}{q_{ij}}&&\lim_j F(i,j) \ar[swap]{dl}{p_{ij}} \ar{dr}{p_{ij'}}&& \lim_iF(i,j') \ar{dl}{q_{ij'}}\\
            &F(i,j)\ar{rr}{F(\id_i,g)}&&F(i,j')
        \end{tikzcd}
    \end{center}
    Now we claim that $\mu_{i'}=\tilde f\circ \mu_i$. To show this, observe the following diagram:
\begin{center}
        \begin{tikzcd}
            &&W \ar[swap]{ddll}{\chi_j} \ar{ddrr}{\chi_{j'}} \ar{dd}[description]{\tilde f\circ \mu_i}\\
            \\
            \lim_i F(i,j) \arrow[bend left=10, crossing over, pos=0.34]{rrrr}[description]{\tilde g} \ar[swap]{dr}{q_{i'j}}&&\lim_j F(i',j) \ar[swap]{dl}{p_{i'j}} \ar{dr}{p_{i'j'}}&& \lim_iF(i,j') \ar{dl}{q_{i'j'}}\\
            &F(i',j)\ar{rr}{F(\id_{i'},g)}&&F(i',j')
        \end{tikzcd}
    \end{center}
which commutes because
        \begin{align*}
            p_{i'j}\circ \tilde f\circ \mu_i=F(f,\id_j)\circ p_{ij}\circ \mu_i=F(f,\id_j)\circ q_{ij}\circ \chi_j=q_{i'j}\circ \chi_j
        \end{align*}
        and
        \begin{align*}
            p_{i'j'}\circ \tilde f\circ \mu_i=F(f,\id_{j'})\circ p_{ij'}\circ \mu_i=F(f,\id_{j'})\circ q_{ij'}\circ \chi_{j'}=q_{i'j'}\circ \chi_{j'}
        \end{align*}
        By uniqueness of $\mu_{i'}$, we get that indeed $\mu_{i'}=\tilde f\circ \mu_i$. Thus we an induced $\theta:W\to \lim_i\lim_j F(i,j)$ from the following commutative diagram:
        \begin{center}
            \begin{tikzcd}
                &W \ar[bend right,swap]{ddl}{\mu_i} \ar[bend left]{ddr}{\mu_{i'}} \ar[dashed]{d}[description]{\exists!}\\
                & \lim_i \lim_j F(i,j) \ar[swap]{dl}{\tau_i} \ar{dr}{\tau_{i'}}\\
                \lim_j F(i,j) \ar{rr}{\tilde f}&& \lim_j F(i',j)
            \end{tikzcd}
        \end{center}
        Consider the following commutative diagram:
        \begin{center}
            \begin{tikzcd}
                &W \ar[bend right]{ddl} \ar[bend left]{ddr}\ar[shift right]{d}[swap]{\chi_j} \ar[shift left]{d}{\varphi_j\circ \theta} \\
                &\lim_i F(i,j)\ar{dl}[swap]{q_{ij}} \ar{dr}{q_{i'j}}\\
                F(i,j) \ar{rr}{F(f,\id_j)}&&F(i',j)
            \end{tikzcd}
        \end{center}
        We observe that 
        \begin{align*}
            q_{ij}\circ \chi_j=p_{ij}\circ \mu_i=p_{ij}\circ \tau_i \circ \theta=q_{ij}\circ \varphi_j \circ \theta
        \end{align*}
        as well as
        \begin{align*}
            q_{i'j}\circ \chi_j=p_{i'j}\circ \mu_{i'}=p_{i'j}\circ \tau_{i'}\circ \theta=q_{i'j}\circ \varphi_j\circ \theta
        \end{align*}
        This proves, by uniqueness of the arrow $W\to \lim_i F(i,j)$ that indeed $\chi_j=\varphi_j\circ \theta$. We need to consider one final commutative diagram:
        \begin{center}
            \begin{tikzcd}
                &W \ar[bend right]{ddl} \ar[bend left]{ddr}\ar[shift right]{d}[swap]{\chi_{j'}} \ar[shift left]{d}{\varphi_{j'}\circ \theta} \\
                &\lim_i F(i,j')\ar{dl}[swap]{q_{ij'}} \ar{dr}{q_{i'j'}}\\
                F(i,j') \ar{rr}{F(f,\id_{j'})}&&F(i',j')
            \end{tikzcd}
        \end{center}
        We observe that
            \begin{align*}
                q_{ij'}\circ \chi_{j'}=p_{ij'}\circ \mu_i=p_{ij'}\circ \tau_i\circ \theta=q_{ij'}\circ \varphi_{j'}\circ \theta
            \end{align*}
            as well as
            \begin{align*}
                q_{i'j'}\circ \chi_{j'}=p_{i'j'}\circ \mu_{i'}=p_{i'j'}\circ \tau_{i'}\circ \theta=q_{i'j'}\circ \varphi_{i'}\circ \theta
            \end{align*}
            Again, by uniqueness of the arrow $W\to \lim_i F(i,j')$, we get that $\chi_j=\varphi_{j'}\circ \theta$. Thus indeed $\theta$ is the unique morphism making the following diagram commute:
            \begin{center}
            \begin{tikzcd}
                &W \ar[bend right]{ddl}[swap]{\chi_j} \ar[bend left]{ddr}{\chi_{j'}} \ar[dashed]{d}[description]{\exists!} \\
                &\lim_i\lim_j F(i,j)\ar{dl}[swap]{\varphi_j} \ar{dr}{\varphi_{j'}}\\
                \lim_i F(i,j) \ar{rr}{\tilde g}&&\lim_i F(i,j')
            \end{tikzcd}
        \end{center}
        Therefore $\lim_i\lim_j F(i,j)$ satisfies the universal property of $\lim_j \lim_i F(i,j)$, proving the two are equal.
\end{proof}
\subsubsection{L}\label{1.6.L}
\begin{proof}
By Exercise \ref{1.4.F}, we know what colimits look like in $\Mod_A$. Suppose that $F,G,H: \fI\to \Mod_A$ are the index functors and we have $f\in \Nat(F,G)$ and $g\in \Nat( G,H)$ such that the following sequence is exact in the category of functors $\Mod_A^\fI$:
\begin{center}
    \begin{tikzcd}
        0 \ar{r}& F \ar{r}{f}& G \ar{r}{g}& H \ar{r}& 0
    \end{tikzcd}
\end{center}
where in particular, the $f_i$'s are from the natural transformation $f:F\to G$. We want to show that the following sequence is exact:
\begin{center}
    \begin{tikzcd}
        0 \ar{r}& \colim M_i \ar{r}{\colim f}& \colim N_i \ar{r}{\colim g}& \colim P_i \ar{r}& 0
    \end{tikzcd}
\end{center}
where $\colim f$ is induced by the following commutative diagram
\begin{center}
    \begin{tikzcd}
        &&\colim N_i \\
            \\
            N_i \ar{uurr} \arrow[bend left=10, crossing over, pos=0.34]{rrrr}&&\colim M_i \ar[dashed]{uu}[description]{\exists!}&& N_j \ar{uull}\\
            &M_i \ar{rr} \ar{ul}{f_i} \ar{ur}&&M_j \ar{ul} \ar{ur}[swap]{f_j}
    \end{tikzcd}
\end{center}
and $\colim g$ is induced by a similar one. More explicitly, the map $\colim f$ and $\colim g$ acts as
\begin{align*}
    \colim f ([m_i,i])=[f_i(m_i),i]\\
    \colim g([n_i,i])=[g_i(n_i),i]
\end{align*}
Suppose that $\colim f([m_i,i])=0$. By definition, this means $(f_i(m_i),i)\sim 0$, which, by definition of the equivalence relation means there exists some $\kappa:i\to j$ such that $G(\kappa)(f_i(m_i))=0$ for some $j\in \fI$. Then we observe the following commutative diagram, which commutes by naturality of $f$:
\begin{center}
    \begin{tikzcd}
        N_i \ar{r}{G(\kappa)}& N_j\\
        M_i \ar{u}{f_i} \ar{r}{F(\kappa)}& M_j \ar{u}{f_j}
    \end{tikzcd}
\end{center}
Therefore 
\begin{align*}
    0=G(\kappa)(f_i(m_i))=f_j(F(\kappa)(m_i)) 
\end{align*}
implies that, because $f_j$ is injective since $f$ is monic, that $F(\kappa)(m_i)=0$. By definition, this means that $[m_i,i]=0$ so indeed $\colim f$ is monic.\\
To show $\colim g$ is epic --i.e. surjective -- fix any $[p_i,i]\in \colim P_i$. Then because $g_i$ is surjective, we get some $n_i\in N_i$ such that $g_i(n_i)=p_i$. Therefore
\begin{align*}
    \colim g([n_i,i])=[g_i(n_i),i]=[p_i,i]
\end{align*}
so $\colim g$ is surjective as well.\\
The last thing to show is that $\ker \colim g=\im \colim f$. If $[n_i,i]\in \im \colim f$, then $[n_i,i]=[f_i(m_i),i]$ for some $m_i\in M_i$. Then because $\im f_i=\ker g_i$,
\begin{align*}
    \colim g([n_i,i])=[g_i(n_i),i]=[g_i\circ f_i(m_i),i]=[0,i]=0
\end{align*}
shows that $\im \colim f \subset \ker \colim g$. On the other hand, fix any $[n_i,i]\in \ker \colim g$. Because $(n_i,i)\sim 0$, there exists some $\gamma:i\to j$ such that $H(\gamma) g_i(n_i)=0$. We observe the following commutative diagram
\begin{center}
    \begin{tikzcd}
        P_i \ar{r}{H(\gamma)}& P_j\\
        N_i \ar{u}{g_i} \ar{r}{G(\gamma)}& N_j \ar{u}{g_j}
    \end{tikzcd}
\end{center}
which shows then that
\begin{align*}
    0=H(\gamma)g_i(n_i)=g_j G(\gamma)(n_i)
\end{align*}
Therefore $G(\gamma)(n_i)\in \ker g_j$. Because $\ker g_j=\im f_j$, let $f_j(m_j)=G(\gamma)(n_i)$. This shows that $(n_i,i)\sim (f_j(m_j),j)$, proving
\begin{align*}
    [n_i,i]=[f_j(m_j),j]=\colim f([m_j,j])
\end{align*}
so $\ker \colim g\subset \im \colim f$, which proves equality holds and that indeed
\begin{center}
    \begin{tikzcd}
        0 \ar{r}& \colim M_i \ar{r}{\colim f}& \colim N_i \ar{r}{\colim g}& \colim P_i \ar{r}& 0
    \end{tikzcd}
\end{center}
is exact.
\end{proof}
\subsubsection{M}
\begin{proof}
    By Exercise \ref{1.6.L}L, colimits are exact. Then we can use colimits as a functor from $\Mod_A^\fI$, and obtain by Exercise \ref{1.6.I}I that
    \begin{align*}
        H \colim C^\bullet =\colim HC^\bullet
    \end{align*}
\end{proof}
\subsubsection{N}\label{1.6.N}
\begin{proof}
    Suppose the following is exact:
    \begin{center}
        \begin{tikzcd}
            0 \ar{r} & A^\bullet \ar{r}{f} & B^\bullet \ar{r}{g}& C^\bullet \ar{r}& 0
        \end{tikzcd}
    \end{center}
    We observe that we get a morphism $\lim f$ from the commutative diagram below:
   \begin{center}
        \begin{tikzcd}
            &&\lim A_n \ar[swap]{ddll} \ar{ddrr} \ar[dashed]{dd}[description]{\exists!}\\
            \\
            A_i \arrow[bend left=10, crossing over, pos=0.34, two heads]{rrrr} \ar{dr}{f_i}&&\lim B_n \ar{dl}{f_{i+1}} \ar{dr}&& A_{i+1} \ar{dl}\\
            &B_i\ar{rr}&&B_{i+1}
        \end{tikzcd}
    \end{center}
    We get a morphism $\lim g:\lim B_n \to \lim C_n$ similar to above. We obtain that these morphisms act as follows:
    \begin{align*}
        \lim f(a_1,a_2,\dots)=(f_1(a_1),f_2(a_2),\dots)\\
        \lim g(b_1,b_2,\dots)=(g_1(b_1),g_2(b_2),\dots)
    \end{align*}
    To show $\lim f$ is injective, suppose $\lim f(a_1,a_2,\dots)=0$. Then for each $i$, $f_i(a_i)=0$, which because each $f_i$ is injective, implies each $a_i=0$ so indeed $\lim f$ has trivial kernel.\\
    To show $\lim g$ is surjective, fix any $(c_1,c_2,\dots)\in \lim C_n$. For each $i$, there exists a $b_i$ such that $g_i(b_i)=c_i$ because each $g_i$ is surjective. Then
    \[
    \lim g(b_1,b_2,\dots)=(g(b_1),g(b_2),\dots)=(c_1,c_2,\dots)
    \]
    as desired. Now to show $\ker \lim g=\im \lim f$, pick any $(f_1(a_1),f_2(a_2),\dots)\in \im \lim f$. Because $\ker g_i=\im f_i$ for each $i$, we have
    \begin{align*}
        \lim g(f_1(a_1),f_2(a_2),\dots)=(g_1f_1(a_1),g_2f_2(a_2),\dots)=(0,0,\dots)=0
    \end{align*}
    which shows $\im \lim f\subset \ker \lim g$. Now suppose $(b_1,b_2,\dots)\in \ker \lim g$, i.e. $g_i(b_i)=0$ for each $i$. Because $\ker g_i=\im f_i$, we get that $b_i=f(a_i)$ for every $i$. Then
    \begin{align*}
        \lim f (a_1,a_2,\dots)=(f_1(a_1),f_2(a_2),\dots)=(b_1,b_2,\dots)
    \end{align*}
    proves $\ker \lim g \subset \im \lim f$. This proves that indeed
    \begin{center}
        \begin{tikzcd}
            0 \ar{r}& \lim A_n \ar{r}{\lim f}& \lim B_n \ar{r}{\lim g}& \lim C_n \ar{r}& 0
        \end{tikzcd}
    \end{center}
    is exact. As a side note, I believe that we used the hypothesis that the transition maps of the left term are surjective because this makes there only be one morphism from $A_i\to A_j$ with $i>j$, though I'm not entirely sure.
\end{proof}
\subsection{}
\printbibliography
\end{document}